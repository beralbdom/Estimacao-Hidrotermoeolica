% resumo em português
\begin{resumo}
\noindent
A matriz elétrica brasileira, caracterizada pela forte dependência da geração hidráulica, apresenta vulnerabilidade a fenômenos
climáticos como o El Niño-Oscilação Sul (ENSO), cujos impactos não são diretamente considerados nos modelos de 
planejamento energético do país. Este trabalho investiga o impacto de variáveis exógenas associadas ao ENSO na estimação da geração 
das fontes hidráulica, eólica e térmica no Sistema Interligado Nacional (SIN). Para tal, foram desenvolvidos e comparados modelos de 
regressão linear, não-linear (Random Forest) e uma arquitetura de rede neural pré-treinada (Tiny Time Mixer). O modelo 
neural foi avaliado em sua aplicação direta, utilizando os pesos pré-treinados, e então foi realizado o processo de 
ajuste fino (fine-tuning) para especializar o modelo com os dados climáticos. Os resultados indicam que os modelos de regressão 
tradicionais não são capazes de capturar as dinâmicas não-lineares das fontes eólica e térmica. O modelo neural, por sua vez,
demonstrou alta capacidade de generalização, e o processo de fine-tuning provou ser a abordagem mais eficaz, resultando em uma diminuição 
do erro médio quadrático (MSE) de até 21,38\% para a fonte hidráulica. Conclui-se que a incorporação de variáveis climáticas externas 
em modelos neurais avançados aprimora significativamente a acurácia da estimação de geração, apresentando-se como uma metodologia promissora
para compor as ferramentas de planejamento energético do setor elétrico.

\vspace{0.2cm}
\noindent
Palavras-chave: Geração de energia. Clima. Planejamento energético. Machine learning.
\end{resumo}

% resumo em inglês
\begin{resumo}[Abstract]	
\begin{otherlanguage*}{english}
\noindent 
\textit{The Brazilian electrical matrix, characterized by its strong dependence on hydroelectric generation, is vulnerable to climate 
phenomena such as the El Niño-Southern Oscillation (ENSO). This work investigates the impact of exogenous variables associated with 
ENSO on the generation forecasting of hydro, wind, and thermal power sources within the National Interconnected System (SIN). To this end, 
linear regression, non-linear (Random Forest), and a pre-trained neural network architecture (Tiny Time Mixer) models were developed and compared. 
The neural network was evaluated both in its direct application and through a fine-tuning process to specialize the model with climate data. The 
results, evaluated by the Mean Squared Error (MSE), indicated that traditional regression models were inadequate for the wind and thermal sources. 
The neural approach demonstrated superiority, and the fine-tuning process proved to be the most effective, achieving an MSE reduction of up to 
21.38\% for the hydraulic source compared to the non-adjusted model. It is concluded that incorporating external climate variables into advanced 
neural networks significantly improves the accuracy of generation forecasting, validating the methodology as a promising tool to complement the 
energy planning frameworks of the electrical sector.}

\vspace{0.2cm}
\noindent
Key-words: Energy generation. Climate. Risk mitigation. Machine learning.
\end{otherlanguage*}
\end{resumo}