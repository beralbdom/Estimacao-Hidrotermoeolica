% resumo em português
\begin{resumo}
\noindent
A matriz energética brasileira é caracterizada por uma dependência significativa de fontes renováveis, especialmente a
geração hidrelétrica. Essa dependência torna o sistema elétrico vulnerável a variações climáticas, como secas 
prolongadas que podem ser intensificadas por fenômenos como o El Niño (EN) e La Niña (LN), afetando a disponibilidade 
de água nos reservatórios e, consequentemente, a geração de energia. Com o crescimento da fonte eólica, também vulnerável a 
variações climáticas, é necessário investigar o impacto de variáveis climatológicas na operação do Sistema Interligado Nacional 
(SIN), com ênfase nas fontes hidráulica e térmica. Este trabalho analisa séries históricas disponibilizadas pelo Operador Nacional 
do Sistema Elétrico (ONS) e variáveis que definem o fenômeno El Niño, obtidas do ERA5, para avaliar os efeitos desses fenômenos
na geração de energia elétrica. Diferentes modelos de regressão e aprendizado de máquina são aplicados para analisar
o impacto dessas variáveis na geração de energia.

\vspace{0.2cm}
\noindent
Palavras-chave: Geração de energia. Clima. Planejamento energético. Machine learning.
\end{resumo}

% resumo em inglês
\begin{resumo}[Abstract]	
\begin{otherlanguage*}{english}
\noindent 
\textit{The Brazilian electrical system is characterized by a significant dependence on renewable sources, especially hydropower generation. 
This dependence makes the electric system vulnerable to climatic variations, such as prolonged droughts and phenomena such as
El Niño (EN) and La Niña (LN), which can affect the availability of water in reservoirs and, consequently, energy generation. 
With the adoption of wind power, also vulnerable to climatic variations, it is necessary to investigate the impact of 
climatic variables on the operation of the National Interconnected System (SIN), with an emphasis on hydraulic, wind and 
thermal sources. This work analyzes historical series provided by the National Electric System Operator (ONS) and variables 
that define the El Niño phenomenon, obtained from the European Centre for Medium-Range Weather Forecasts (ECMWF) 
Reanalysis project (ERA5), to evaluate the effects of these phenomena on electricity generation. 
Different regression and machine learning models are applied to analyze the impact of these variables on energy generation.}

\vspace{0.2cm}
\noindent
Key-words: Energy generation. Climate. Risk mitigation. Machine learning.
\end{otherlanguage*}
\end{resumo}