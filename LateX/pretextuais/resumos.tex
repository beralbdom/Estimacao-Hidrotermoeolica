% resumo em português
\begin{resumo}
	\noindent
	Este trabalho investiga o impacto de variáveis climatológicas na operação do Sistema Interligado Nacional, com ênfase nas
	fontes hidráulica e térmica. São analisadas séries históricas disponibilizadas pelo ONS e variáveis climáticas
	fornecidas por instituições como NOAA, INPE e ECMWF, a fim de avaliar correlações e tendências ao longo do tempo. Técnicas de
	processamento intensivo, regressão e aprendizado de máquina são aplicadas para detectar padrões e propor estratégias de
	adaptação e mitigação de riscos no planejamento energético brasileiro.
	
	\vspace{0.2cm} 
	Palavras-chave: Geração de energia. Clima. Planejamento energético. Machine learning.
	\end{resumo}

% resumo em inglês
\begin{resumo}[Abstract]	
 	\begin{otherlanguage*}{english}
 	\noindent 
	\textit{This work investigates the impact of climatological variables on the operation of the Brazilian National 
	Interconnected System, emphasizing hydraulic and thermal sources. Historical time series from ONS and climate 
	data from institutions such as NOAA, INPE, and ECMWF are analyzed to identify correlations and trends over time.
	Intensive data processing, regression, and machine learning techniques are applied to detect patterns and propose 
	strategies for adapting to and mitigating risks in Brazil’s energy planning.
	} 
   \vspace{0.2cm}
 
   
    Key-words: Energy generation. Climate. Risk mitigation. Machine learning.
 	\end{otherlanguage*}
\end{resumo}