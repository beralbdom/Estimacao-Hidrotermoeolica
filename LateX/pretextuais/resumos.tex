% resumo em português
\begin{resumo}
\noindent
\sloppy
A matriz elétrica brasileira, caracterizada pela forte dependência da fonte hidráulica, é vulnerável à fenômenos
climáticos como o El Niño-Oscilação Sul (ENSO), cujos impactos não são considerados diretamente pela maior parte dos modelos 
computacionais utilizados pelo setor. Este trabalho investiga o impacto de variáveis exógenas associadas 
ao ENSO na estimação da geração das fontes hidráulica, eólica e térmica no Sistema Interligado Nacional (SIN). Para tal, foram 
aplicados e comparados modelos de regressão linear, não-linear (Random Forest) e um modelo neural baseado na arquitetura TSMixer. 
O modelo neural foi avaliado em sua aplicação direta (sem dados do ENSO), e também com o ajuste fino (com dados do ENSO). Os resultados
indicam que os modelos de regressão tradicionais não são capazes de capturar as dinâmicas não-lineares das fontes eólica e térmica. 
O modelo neural, por sua vez, demonstrou alta capacidade de generalização, e a inclusão dos dados do ENSO resultou em uma diminuição 
do erro médio quadrático (MSE) para a fonte hidráulica de 21,38\% para a janela de previsão de 96 dias e 5,8\% para a janela de previsão 
de 30 semanas. As fontes eólica e térmica também apresentaram melhoria. Sendo assim, a incorporação de variáveis climáticas externas em 
modelos neurais avançados aprimora significativamente a acurácia da estimação de geração, apresentando-se como uma metodologia promissora 
para compor as ferramentas de planejamento energético do setor elétrico.

\vspace{0.2cm}
\noindent
Palavras-chave: Geração de energia. Clima. Planejamento energético. Machine learning.
\end{resumo}

% resumo em inglês
\begin{resumo}[Abstract]	
\begin{otherlanguage*}{english}
\noindent
\sloppy
\textit{The brazilian electrical matrix, which is historically dependent on the hydroelectric generation, is vulnerable to 
external climatological phenomena such as El Niño-Southern Oscilation (ENSO). These external effects are not directly considered
on the main computational models used by the electric sector. This work investigates the impact of these variables on the
hydro, eolic and thermal electricity generation estimation. Different models were implemented to measure the impact of these variables
on the accuracy of the results: multivariate linear regression and random forest regressor as baselines, and a neural model based on the 
TSMixer architecture. The neural model was evaluated on its oneshot (no ENSO data) and finetune (with ENSO data) approaches.
The results show that the introduction of ENSO data as exogenous variables on the finetuned neural model resulted in a
21,38\% smaller mean squared error (MSE) compared to the oneshot implementation for the hydro source, considering the prediction
window of 96 days. For the 30 week prediction window, the observed improvement was 5,8\%. The eolic and thermal sources
also showed improvement on the MSE. Therefore, the usage of climatological variables of external phenomena, such
as ENSO, is a valid approach to improve the accuracy of the estimations produced by advanced regression models, contributing to the tools already available to the electric sector. }

\vspace{0.2cm}
\noindent
Key-words: Energy generation. Climate. Risk mitigation. Machine learning.
\end{otherlanguage*}
\end{resumo}