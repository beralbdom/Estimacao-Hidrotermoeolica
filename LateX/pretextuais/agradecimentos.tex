\begin{agradecimentos}
\sloppy	

Agradeço aos meus pais, Débora e Gilvan, por sempre me apoiarem nas minhas escolhas, terem me ajudado a superar os obstáculos 
que surgiram no caminho e sempre estarem presentes nos momentos mais importantes. Obrigado por acreditarem em mim, por 
terem abdicado de tantas coisas para me proporcionar uma educação de qualidade e me ensinado a importância da ética, esforço, 
estudo e trabalho. Serei eternamente grato pelos esforços e sacrifícios que fizeram por mim. Vocês sempre serão os meus 
maiores exemplos na vida.

À minha irmã, Letícia, por ser meu alívio cômico por todos esses anos. À minha família, por compreender minhas ausências
e por todos os aprendizados que me proporcionaram.

À minha companheira, Juliana, por ter me apoiado, incentivado e compreendido durante essa trajetória. Por ter me 
ajudado a manter a calma e acreditar que eu era capaz de superar qualquer obstáculo. Obrigado por ser a
minha maior incentivadora e por nunca me deixar desistir nos momentos de dificuldade. Seu amor e paciência foram
essenciais para que eu pudesse concluir este trabalho.

Ao Diego e demais amigos que fiz durante a graduação, que sem dúvidas espero levar para a vida toda. Sem vocês o caminho
teria sido muito mais difícil. À Faraday E-Racing, que representou um marco na minha trajetória acadêmica e me proporcionou 
experiências essenciais para a minha formação. À equipe de Tecnologias da Superintendência de Geração de 
Energia (SGR) da Empresa de Pesquisa Energética (EPE), por todo o aprendizado e pelas oportunidades durante o meu período de estágio 
que possibilitaram o desenvolvimento deste trabalho. 

Ao professor André Pinho, pela sua excelência, ética, maestria em ensinar e por ter me orientado de maneira exemplar durante 
o desenvolvimento deste projeto. Agradeço também aos professores Flávio Martins, Felipe Sass e Marcio Guimaraens, por me 
lembrarem em cada aula do motivo pelo qual escolhi a Engenharia Elétrica.

\end{agradecimentos}