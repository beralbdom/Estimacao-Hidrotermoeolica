\chapter{Resultados}
Esse capítulo apresentará os resultados das implementações dos modelos considerando os dados de geração, carga e ENSO.
Para todos os casos, o conjunto de variáveis dependentes foi constituído pelas séries de geração por fonte e subsistema,
e o conjunto de variáveis independentes foi constituído pelas séries de carga e ENSO. Serão apresentados os gráficos
de geração total das fontes. Demais gráficos dos resultados para os subsistemas estão disponíveis no Apêndice B.

\section{Modelo Linear}
\subsection{Fonte Hidráulica}
A Figura \ref{fig:linear_hidraulica} apresenta os resultados da regressão linear para a fonte hidráulica. Para esse caso,
foram consdierados os dados de 2010 em diante. O conjunto de treino foi constituído pelos dados de 2010 a 2021, e o 
conjunto de teste pelos dados restantes. 

\begin{figure}[!ht]
  \IBGEtab{\caption{Regressão linear para fonte hidráulica (carga + ENSO) em base diária}
       \label{fig:linear_hidraulica}}
  {\includesvg[scale=.9]{figuras/linear_Hidráulica}}
  {\fonte{o autor.}}
\end{figure}
Fica evidente que, para a geração hidráulcia total, o modelo é capaz de estimar os valores de geração do conjunto de teste
com um valor de R² de 0,593, o que indica uma correlação moderada entre os valores estimados e os valores reais, considerando
as limitações do modelo linear. 

A fim de validar a utilidade dos dados do ENSO, uma nova regressão foi realizada, dessa vez
considerando apenas a carga como variável independente, conforme a Figura \ref{fig:linear_hidraulica_carga}. Observa-se 
uma queda de 0,593 para 0,525 no valor de R², indicando que a inclusão do ENSO melhora a capacidade preditiva do modelo.
Além disso, o resultado é compatível com a fundamentação teórica apresentada no Capítulo 3, na qual a influência do ENSO
na geração hidráulica é direta e significativa.

\begin{figure}[!ht]
  \IBGEtab{\caption{Regressão linear para fonte hidráulica (apenas carga) em base diária}
       \label{fig:linear_hidraulica_carga}}
  {\includesvg[scale=.9]{figuras/linear_Hidráulica_CARGA}}
  {\fonte{o autor.}}
\end{figure}

A fim de melhor avaliar os resultados, os demais gráficos serão apresentados em base temporal mensal, indicando a média 
dos valores de geração para cada mês em MWmed, consdierando os valores médios das variáveis independentes para o mesmo período. 
A Figura \ref{fig:linear_hidraulica_mensal} apresenta os resultados mensais da regressão linear para a fonte hidráulica, 
considerando a carga e o ENSO. 

\begin{figure}[!ht]
  \IBGEtab{\caption{Regressão linear para fonte hidráulica (carga + ENSO) em base mensal}
       \label{fig:linear_hidraulica_mensal}}
  {\includesvg[scale=.9]{figuras/linear_Hidráulica_mensal}}
  {\fonte{o autor.}}
\end{figure}
\begin{figure}[!ht]
  \IBGEtab{\caption{Regressão linear para fonte hidráulica (apenas carga) em base mensal}
       \label{fig:linear_hidraulica_mensal_carga}}
  {\includesvg[scale=.9]{figuras/linear_Hidráulica_mensal_CARGA}}
  {\fonte{o autor.}}
\end{figure}

Novamente, é realizado uma avaliação do modelo considerando apenas a carga, conforme a Figura 
\ref{fig:linear_hidraulica_mensal_carga}. Observa-se que, para a regressão mensal, a queda na performance do modelo é ainda
maior: o valor de R² cai de 0,519 para 0,420. Isso pode indicar que a inclusão de variáveis do ENSO é ainda mais significante 
para a base temporal mensal.

\subsection{Fonte Eólica e Térmica}
Para as fontes eólica e térmica, o modelo linear não foi capaz de apresentar resultados satisfatórios, o que pode indicar
que a relação entre as variáveis independentes e dependentes é não linear.

\begin{figure}[!ht]
  \IBGEtab{\caption{Regressão linear para fonte eólica (carga + ENSO) em base mensal}
       \label{fig:linear_eolica_mensal}}
  {\includesvg[scale=.9]{figuras/linear_Eólica_mensal}}
  {\fonte{o autor.}}
\end{figure}
\begin{figure}[!ht]
  \IBGEtab{\caption{Regressão linear para fonte térmica (carga + ENSO) em base mensal}
       \label{fig:linear_termica_mensal}}
  {\includesvg[scale=.9]{figuras/linear_Térmica_mensal}}
  {\fonte{o autor.}}
\end{figure}

A regressão para a fonte eólica resultou em um valor de R² de 0,004, o que indica que o modelo não é capaz de explicar a 
variabilidade dos dados, enquanto que para a fonte térmica, o valor de R² foi negativo, indicando que o modelo é pior do 
que uma média simples dos dados. Esses resultados corroboram a hipótese de que a relação entre as variáveis independentes 
e dependentes é não linear, o que justifica a escolha de modelos mais complexos para essas fontes.


\section{Modelo Não Linear}
\subsection{Fonte Hidráulica}
\subsection{Fonte Eólica}
\subsection{Fonte Térmica}


\section{Modelo Neural}
\subsection{Baseline: Previsão com Modelo Pré-treinado}
Ao fazer a implementação \textit{oneshot}, todos os pesos do \textit{backbone}, as camadas escondidas, permanecem inalterados
em relação ao modelo pré treinado, e apenas a \textit{head}, ou cabeça de previsão, é ajustada para gerar a saída desejada,
considerando o número de variáveis dependentes.

A seguir serão apresentados os resultados da previsão \textit{oneshot} para cada fonte de geração, considerando os dados de carga e ENSO
como variáveis independentes. Para cada fonte, serão apresentados os resultados considerando as janelas de contexto e previsão
de 90/30 semanas e 512/96 dias. Demais gráficos dos resultados para os subsistemas estão disponíveis no Apêndice B.

\subsubsection{Fonte Hidráulica}
\begin{figure}[!ht]
  \IBGEtab{\caption{Oneshot para fonte hidráulica (contexto/previsão: 90/30 semanas)}
       \label{fig:oneshot_hidraulica_w90-30}}
  {\includesvg[scale=.9]{figuras/Hidráulica_W90-30}}
  {\fonte{o autor.}}
\end{figure}

\begin{figure}[!ht]
  \IBGEtab{\caption{Oneshot para fonte hidráulica (contexto/previsão: 512/96 dias)}
       \label{fig:oneshot_hidraulica_d512-96}}
  {\includesvg[scale=.9]{figuras/Hidráulica_D512-96}}
  {\fonte{o autor.}}
\end{figure}

\subsubsection{Fonte Eólica}
\begin{figure}[!ht]
  \IBGEtab{\caption{Oneshot para fonte eólica (contexto/previsão: 90/30 semanas)}
       \label{fig:oneshot_eolica_w90-30}}
  {\includesvg[scale=.9]{figuras/Eólica_W90-30}}
  {\fonte{o autor.}}
\end{figure}

\begin{figure}[!ht]
  \IBGEtab{\caption{Oneshot para fonte eólica (contexto/previsão: 512/96 dias)}
       \label{fig:oneshot_eolica_d512-96}}
  {\includesvg[scale=.9]{figuras/Eólica_D512-96}}
  {\fonte{o autor.}}
\end{figure}

\subsubsection{Fonte Térmica}
\begin{figure}[!ht]
  \IBGEtab{\caption{Oneshot para fonte térmica (contexto/previsão: 90/30 semanas)}
       \label{fig:oneshot_termica_w90-30}}
  {\includesvg[scale=.9]{figuras/Térmica_W90-30}}
  {\fonte{o autor.}}
\end{figure}

\begin{figure}[!ht]
  \IBGEtab{\caption{Oneshot para fonte térmica (contexto/previsão: 512/96 dias)}
       \label{fig:oneshot_termica_d512-96}}
  {\includesvg[scale=.9]{figuras/Térmica_D512-96}}
  {\fonte{o autor.}}
\end{figure}


\subsection{Impacto e Análise do Ajuste Fino}
\subsubsection{Fonte Hidráulica}
\subsubsection{Fonte Eólica}
\subsubsection{Fonte Térmica}