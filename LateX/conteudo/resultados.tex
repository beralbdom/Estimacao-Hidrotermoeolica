\chapter{Resultados}
Este capítulo apresenta os resultados obtidos na estimação da geração das fontes hidráulica, eólica e térmica. A análise 
se inicia com modelos de regressão linear e não linear (Random Forest) para estabelecer um baseline de performance e 
avaliar a linearidade das relações (Seções 5.1 e 5.2). Em seguida, na Seção 5.3, avalia-se o desempenho do modelo 
neural (\textit{Tiny Time Mixer}) em sua aplicação direta. Por fim, na Seção 5.3.2, demonstra-se o impacto e a 
eficácia do ajuste fino (\textit{finetuning}) deste modelo com a inclusão de variáveis do fenômeno ENSO, quantificando o ganho de 
acurácia obtido com a especialização do modelo. Serão apresentados os gráficos de geração total das fontes. Demais gráficos 
dos resultados para os subsistemas estão disponíveis no Apêndice B.

\section{Modelo Linear}
\subsection{Fonte Hidráulica}
A Figura \ref{fig:linear_hidraulica} apresenta os resultados da regressão linear para a fonte hidráulica. Para esse caso,
foram consdierados os dados de 2010 em diante. O conjunto de treino foi constituído pelos dados de 2010 a 2021, e o 
conjunto de teste pelos dados restantes. 

\begin{figure}[!ht]
  \IBGEtab{\caption{Regressão linear para fonte hidráulica (carga + ENSO) em base diária}
       \label{fig:linear_hidraulica}}
  {\includesvg[scale=.9]{figuras/linear_Hidráulica}}
  {\fonte{o autor.}}
\end{figure}
Fica evidente que, para a geração hidráulcia total, o modelo é capaz de estimar os valores de geração do conjunto de teste
com um valor de R² de 0,593, o que indica uma correlação moderada entre os valores estimados e os valores reais, considerando
as limitações do modelo linear. 

Para verificar a utilidade dos dados do ENSO, uma nova regressão foi realizada, dessa vez
considerando apenas a carga como variável independente, conforme a Figura \ref{fig:linear_hidraulica_carga}. Observa-se 
uma queda no valor de R² de 0,593 para 0,525, uma redução de 11,5\% na capacidade explicativa do modelo, indicando que a 
inclusão do ENSO melhora significativamente a predição. Além disso, o resultado é compatível com a fundamentação teórica 
apresentada no Capítulo 3, na qual a influência do ENSO na geração hidráulica é direta e significativa.

\begin{figure}[!ht]
  \IBGEtab{\caption{Regressão linear para fonte hidráulica (apenas carga) em base diária}
       \label{fig:linear_hidraulica_carga}}
  {\includesvg[scale=.9]{figuras/linear_Hidráulica_CARGA}}
  {\fonte{o autor.}}
\end{figure}

Embora o modelo linear atinja um R² de 0,593, o gráfico de dispersão (à esquerda) mostra que os pontos se 
afastam da linha ideal especialmente nos extremos. Nota-se que o modelo tende a subestimar os picos de geração 
(pontos acima da linha) e a superestimar os vales (pontos abaixo da linha), uma limitação característica da regressão 
linear ao tentar capturar a amplitude total de uma série com forte sazonalidade.

A fim de melhor avaliar os resultados, os demais gráficos serão apresentados em base temporal mensal, indicando a média 
dos valores de geração para cada mês em MWmed, consdierando os valores médios das variáveis independentes para o mesmo período. 
A Figura \ref{fig:linear_hidraulica_mensal} apresenta os resultados mensais da regressão linear para a fonte hidráulica, 
considerando a carga e o ENSO. 

\begin{figure}[!ht]
  \IBGEtab{\caption{Regressão linear para fonte hidráulica (carga + ENSO) em base mensal}
       \label{fig:linear_hidraulica_mensal}}
  {\includesvg[scale=.9]{figuras/linear_Hidráulica_mensal}}
  {\fonte{o autor.}}
\end{figure}
\begin{figure}[!ht]
  \IBGEtab{\caption{Regressão linear para fonte hidráulica (apenas carga) em base mensal}
       \label{fig:linear_hidraulica_mensal_carga}}
  {\includesvg[scale=.9]{figuras/linear_Hidráulica_mensal_CARGA}}
  {\fonte{o autor.}}
\end{figure}

Novamente, é realizado uma avaliação do modelo considerando apenas a carga, conforme a Figura 
\ref{fig:linear_hidraulica_mensal_carga}. Observa-se que, para a regressão mensal, a queda na performance do modelo é ainda
maior: o valor de R² cai de 0,519 para 0,420. Isso pode indicar que a inclusão de variáveis do ENSO é ainda mais significante 
para a base temporal mensal.

\subsection{Fonte Eólica e Térmica}
As Figuras \ref{fig:linear_eolica_mensal} e \ref{fig:linear_termica_mensal} apresentam os resultados para as fontes eólica 
e térmica, nos quais pode-se observar que o modelo não foi capaz de apresentar resultados satisfatórios, sugerindo
que a relação entre as variáveis independentes e dependentes pode ser não linear, validando a hipótese levantada nos capítulos
anteriores.

\begin{figure}[!ht]
  \IBGEtab{\caption{Regressão linear para fonte eólica (carga + ENSO) em base mensal}
       \label{fig:linear_eolica_mensal}}
  {\includesvg[scale=.9]{figuras/linear_Eólica_mensal}}
  {\fonte{o autor.}}
\end{figure}
\begin{figure}[!ht]
  \IBGEtab{\caption{Regressão linear para fonte térmica (carga + ENSO) em base mensal}
       \label{fig:linear_termica_mensal}}
  {\includesvg[scale=.9]{figuras/linear_Térmica_mensal}}
  {\fonte{o autor.}}
\end{figure}

A regressão para a fonte eólica resultou em um valor de R² de 0,004, o que indica que o modelo não é capaz de explicar a 
variabilidade dos dados, enquanto que para a fonte térmica, o valor de R² foi negativo, indicando que o modelo é pior do 
que uma média simples dos dados. Esses resultados corroboram a hipótese de que a relação entre as variáveis independentes 
e dependentes é não linear, o que justifica a escolha de modelos mais complexos para essas fontes.


\section{Modelo Não Linear}
\subsection{Fonte Hidráulica}
\begin{figure}[!ht]
  \IBGEtab{\caption{Random forest para fonte hidráulica}
       \label{fig:rf_hidraulica}}
  {\includesvg[scale=.9]{figuras/nlinear/rf_Hidráulica}}
  {\fonte{o autor.}}
\end{figure}
A Figura \ref{fig:rf_hidraulica} apresenta os resultados do modelo não linear para a fonte hidráulica, considerando os dados 
de carga e ENSO como variáveis independentes. O dataset utilizado contém dados a partir de 2010. Os resultados são
ligeiramente superiores aos do modelo de regressão linear, com um valor de R² de 0,525.

\subsection{Fonte Eólica e Térmica}
\begin{figure}[!ht]
  \IBGEtab{\caption{Random forest para fonte eólica}
       \label{fig:rf_eolica}}
  {\includesvg[scale=.9]{figuras/nlinear/rf_Eólica}}
  {\fonte{o autor.}}
\end{figure}
\begin{figure}[!ht]
  \IBGEtab{\caption{Random forest para fonte térmica}
       \label{fig:rf_termica}}
  {\includesvg[scale=.9]{figuras/nlinear/rf_Térmica}}
  {\fonte{o autor.}}
\end{figure}

As Figuras \ref{fig:rf_eolica} e \ref{fig:rf_termica} apresentam os resultados do modelo não linear para as fontes eólica 
e térmica, respectivamente. De maneira similar ao modelo linear, o modelo não linear também não é capaz de produzir resultados
satisfatórios para essas fontes.

Ainda que o modelo Random Forest apresente uma capacidade superior para modelar não-linearidades, os resultados para as 
fontes eólica e térmica permaneceram insatisfatórios. Essa dificuldade sugere que as relações temporais são complexas e 
podem ser mais bem capturadas por modelos como o avaliado a seguir.

\section{Modelo Neural}
\subsection{Modelo Pré-treinado}
Ao fazer a implementação \textit{oneshot}, todos os pesos do \textit{backbone}, as camadas escondidas, permanecem inalterados
em relação ao modelo pré treinado, e apenas a \textit{head}, ou cabeça de previsão, é ajustada para gerar a saída desejada,
considerando o número de variáveis dependentes.

A seguir serão apresentados os resultados da previsão \textit{oneshot} para cada fonte de geração, considerando os dados de carga e ENSO
como variáveis independentes. Para cada fonte, serão apresentados os resultados considerando as janelas de contexto e previsão
de 90/30 semanas e 512/96 dias. Demais gráficos dos resultados para os subsistemas estão disponíveis no Apêndice B.

\subsubsection{Fonte Hidráulica}
\begin{figure}[!ht]
  \IBGEtab{\caption{Oneshot para fonte hidráulica (contexto/previsão: 90/30 semanas)}
       \label{fig:oneshot_hidraulica_w90-30}}
  {\includesvg[scale=.9]{figuras/oneshot/os_Hidráulica_W90-30}}
  {\fonte{o autor.}}
\end{figure}
\begin{figure}[!ht]
  \IBGEtab{\caption{Oneshot para fonte hidráulica (contexto/previsão: 512/96 dias)}
       \label{fig:oneshot_hidraulica_d512-96}}
  {\includesvg[scale=.9]{figuras/oneshot/os_Hidráulica_D512-96}}
  {\fonte{o autor.}}
\end{figure}
As Figuras \ref{fig:oneshot_hidraulica_w90-30} e \ref{fig:oneshot_hidraulica_d512-96} apresentam os resultados da previsão
\textit{oneshot} para a fonte hidráulica, considerando as janelas de contexto e previsão de 90/30 semanas e 512/96 dias, respectivamente.
Observa-se que as métricas de avaliação foram superiores para o caso de 512/96 dias. Isso pode ser explicado devido à
taxa de amostragem diária, que permite uma maior quantidade de dados para o treinamento do modelo e maior número de 
janelas de contexto e previsão.

Para o caso de 90/30 semanas, o valor de R² foi de 0,573, enquanto que para o caso de 512/96 dias, o valor de R² foi de 0,581.
Esses valores são razoáveis, considerando a implementação \textit{oneshot}, sem ajuste fino para o dataset. Além disso, os
resultados são comparáveis aos resultados do modelo linear, o que sugere que a relação da geração hidráulica com as variáveis 
consideradas pode ser mais linear do que as demais fontes de geração.

\subsubsection{Fonte Eólica}
\begin{figure}[!ht]
  \IBGEtab{\caption{Oneshot para fonte eólica (contexto/previsão: 90/30 semanas)}
       \label{fig:oneshot_eolica_w90-30}}
  {\includesvg[scale=.9]{figuras/oneshot/os_Eólica_W90-30}}
  {\fonte{o autor.}}
\end{figure}
\begin{figure}[!ht]
  \IBGEtab{\caption{Oneshot para fonte eólica (contexto/previsão: 512/96 dias)}
       \label{fig:oneshot_eolica_d512-96}}
  {\includesvg[scale=.9]{figuras/oneshot/os_Eólica_D512-96}}
  {\fonte{o autor.}}
\end{figure}
As Figuras \ref{fig:oneshot_eolica_w90-30} e \ref{fig:oneshot_eolica_d512-96} apresentam os resultados da previsão
\textit{oneshot} para a fonte eólica, considerando as janelas de contexto e previsão de 90/30 semanas e 512/96 dias, respectivamente.
Observa-se que, para o caso de 90/30 semanas, o valor de R² foi de 0,683, enquanto que para o caso de 512/96 dias, o valor de R² foi de 0,826.

Esses valores são superiores aos resultados da fonte hidráulica, sugerindo que o modelo consegue capturar melhor a relação
entre as variáveis. Além disso, as métricas de avaliação foram consideravelmente superiores em comparação com os modelos
linear e não linear, sugerindo que, de fato, a relação entre as variáveis é extremamente não linear.

\subsubsection{Fonte Térmica}
\begin{figure}[!ht]
  \IBGEtab{\caption{Oneshot para fonte térmica (contexto/previsão: 90/30 semanas)}
       \label{fig:oneshot_termica_w90-30}}
  {\includesvg[scale=.9]{figuras/oneshot/os_Térmica_W90-30}}
  {\fonte{o autor.}}
\end{figure}
\begin{figure}[!ht]
  \IBGEtab{\caption{Oneshot para fonte térmica (contexto/previsão: 512/96 dias)}
       \label{fig:oneshot_termica_d512-96}}
  {\includesvg[scale=.9]{figuras/oneshot/os_Térmica_D512-96}}
  {\fonte{o autor.}}
\end{figure}
As Figuras \ref{fig:oneshot_termica_w90-30} e \ref{fig:oneshot_termica_d512-96} apresentam os resultados da previsão
\textit{oneshot} para a fonte térmica, considerando as janelas de contexto e previsão de 90/30 semanas e 512/96 dias, respectivamente.
A geração térmica apresentou o pior resultado entre as fontes, com um valor de R² de 0,183 para o caso de 90/30 semanas
e 0,531 para o caso de 512/96 dias. Ainda assim, os valores são consideravelmente superiores em comparação com os os
modelos linear e não linear.

\subsection{Impacto e Análise do Ajuste Fino}
\subsubsection{Fonte Hidráulica}
\begin{figure}[!ht]
  \IBGEtab{\caption{Finetune para fonte hidráulica (contexto/previsão: 90/30 semanas)}
       \label{fig:finetune_hidraulica_w90-30}}
  {\includesvg[scale=.9]{figuras/finetune/fn_Hidráulica_W90-30}}
  {\fonte{o autor.}}
\end{figure}
\begin{figure}[!ht]
  \IBGEtab{\caption{Finetune para fonte hidráulica (contexto/previsão: 512/96 dias)}
       \label{fig:finetune_hidraulica_d512-96}}
  {\includesvg[scale=.9]{figuras/finetune/fn_Hidráulica_D512-96}}
  {\fonte{o autor.}}
\end{figure}
As Figuras \ref{fig:finetune_hidraulica_w90-30} e \ref{fig:finetune_hidraulica_d512-96} apresentam os resultados do ajuste fino
para a fonte hidráulica, considerando as janelas de contexto e previsão de 90/30 semanas e 512/96 dias, respectivamente.

Observa-se que o ajuste fino melhorou em 4,53\% o valor de R² para o caso de 90/30 semanas, passando de 0,573 para 0,599.
Para o caso de 512/96 dias, o ajuste fino melhorou em 15,49\% o valor de R², passando de 0,581 para 0,671. Os resultados 
indicam que o processo de ajuste dos pesos do modelo é eficaz em melhorar a performance do modelo.

\subsubsection{Fonte Eólica}
\begin{figure}[!ht]
  \IBGEtab{\caption{Finetune para fonte eólica (contexto/previsão: 90/30 semanas)}
       \label{fig:finetune_eolica_w90-30}}
  {\includesvg[scale=.9]{figuras/finetune/fn_Eólica_W90-30}}
  {\fonte{o autor.}}
\end{figure}
\begin{figure}[!ht]
  \IBGEtab{\caption{Finetune para fonte eólica (contexto/previsão: 512/96 dias)}
       \label{fig:finetune_eolica_d512-96}}
  {\includesvg[scale=.9]{figuras/finetune/fn_Eólica_D512-96}}
  {\fonte{o autor.}}
\end{figure}
As Figuras \ref{fig:finetune_eolica_w90-30} e \ref{fig:finetune_eolica_d512-96} apresentam os resultados do ajuste fino
para a fonte eólica, considerando as janelas de contexto e previsão de 90/30 semanas e 512/96 dias, respectivamente.

Observa-se que o ajuste fino melhorou em 19,76\% o valor de R² para o caso de 90/30 semanas, passando de 0,683 para 0,818.
Para o caso de 512/96 dias, o ajuste fino melhorou em 1,09\% o valor de R², passando de 0,826 para 0,835. Os resultados
também indicam que o processo de ajuste dos pesos foi eficaz para a fonte eólica, em especial para o caso de 90/30 semanas.

\subsubsection{Fonte Térmica}
\begin{figure}[!ht]
  \IBGEtab{\caption{Finetune para fonte térmica (contexto/previsão: 90/30 semanas)}
       \label{fig:finetune_termica_w90-30}}
  {\includesvg[scale=.9]{figuras/finetune/fn_Térmica_W90-30}}
  {\fonte{o autor.}}
\end{figure}
\begin{figure}[!ht]
  \IBGEtab{\caption{Finetune para fonte térmica (contexto/previsão: 512/96 dias)}
       \label{fig:finetune_termica_d512-96}}
  {\includesvg[scale=.9]{figuras/finetune/fn_Térmica_D512-96}}
  {\fonte{o autor.}}
\end{figure}
As Figuras \ref{fig:finetune_termica_w90-30} e \ref{fig:finetune_termica_d512-96} apresentam os resultados do ajuste fino
para a fonte térmica, considerando as janelas de contexto e previsão de 90/30 semanas e 512/96 dias, respectivamente.

A fonte térmica apresentou melhoria de 37,15\% no valor de R² para o caso de 90/30 semanas, passando de 0,183 para 0,251.
Para o caso de 512/96 dias, o ajuste fino não apresentou melhoria significativa. 

Os resultados demonstram uma clara hierarquia de performance entre os modelos testados. Foi evidenciado que modelos 
lineares e não-lineares tradicionais são insuficientes para capturar a complexa dinâmica das gerações eólica e térmica. 

O modelo neural pré-treinado (TinyTimeMixer) representou um salto de qualidade, especialmente para a fonte eólica 
(R² de 0,826). Contudo, a análise demonstrou que a maior acurácia em todas as fontes foi atingida através do processo 
de ajuste fino (fine-tuning), que especializou o modelo com variáveis do ENSO, resultando em melhorias de performance 
significativas no R². Estes achados validam a hipótese central do trabalho: a incorporação de variáveis 
climáticas externas, aliada a uma arquitetura neural adequada, aprimora significativamente a estimação da geração energética.