% \chapter{Trabalhos Relacionados}
% Os impactos do fenômeno ENSO vêm sendo estudados extensamente em diversas áreas. Segundo \citeonline{Cirino2015}, eventos
% de El Niño e La Niña influenciam significativamente a agricultura brasileira, especialmente nas regiões Sul e Nordeste.

% Um estudo de \citeonline{Pirani2024} demonstrou que a ocorrência do fenômeno El Niño está associada a uma maior taxa de 
% casos de dengue no estado de São Paulo, devido a um aumento na temperatura e precipitação, condições
% favoráveis para a proliferação do mosquito Aedes aegypti.

% Segundo \citeonline{Capozzoli2017}, o fenômeno ENSO tem uma relação direta sobre a disponibilidade de recursos hídricos
% nas bacias hidrográficas brasileiras, o que sugere um impacto direto sobre a geração hidrelétrica no Brasil. Além disso,
% os resultados são coerentes com a literatura, indicando impactos distintos nas diferentes regiões do país.

% Naturalmente, o fenômeno ENSO também vem sendo estudado no contexto do setor elétrico brasileiro. Segundo um
% estudo da \citeonline{epe2023}, a ocorrência do fenômeno La Niña em 2021 foi um fator determinante para a crise hídrica
% que afetou o Brasil nesse período, em decorrência da redução das afluências, ou seja, a quantidade de água que chega aos
% reservatórios das usinas hidrelétricas.

% De acordo com \citeonline{Sanchez2006}, o recurso associado à fonte eólica é diretamente afetado por variáveis climáticas,
% como densidade local do ar, precipitação, temperatura e cobertura de nuvens. Nesse sentido, o trabalho apresenta um sistema de 
% previsão estatística para a produção de energia eólica de curto prazo, destacando a necessidade de adaptabilidade 
% para lidar com as relações não lineares entre as variáveis climáticas e a geração eólica.

% Segundo o relatório da Comissão Permanente para Análise de Metodologias e Programas Computacionais do Setor Elétrico (CPAMP),
% constituída por instituições do setor elétrico como EPE, ONS e CEPEL, a incorporação de dados de variáveis climáticas,
% como o fenômeno ENSO, aos modelos computacionais é ativamente discutida devido a relação entre as séries históricas de 
% vazões e dos ciclos de índices climáticos. \cite{cpamp2019}

% De acordo com \citeonline{Resende2018}, o uso de modelos de aprendizado de máquina para previsão de carga do SIN tem o
% potencial de aprimorar o resultado das previsões, reduzindo os desvios de previsão de carga e, consequentemente, uma redução 
% significativa dos custos de operação do sistema elétrico. Infere-se, portanto, que essa abordagem também poderia
% ser aplicada para estimar outros parâmetros, como a geração futura.

% Para isso, é essencial selecionar modelos que sejam capazes de capturar as relações potencialmente complexas entre os
% dados de geração e os fenômenos climáticos. Considerando que a literatura sugere que essa relação seja altamente não-linear,
% modelos neurais seriam uma escolha natural, mas não necessariamente os modelos neurais mais avançados seriam os mais adequados.

% Segundo \citeonline{Zeng2022}, modelos de previsão de séries temporais baseados na arquitetura \textit{Transformer}, introduzida 
% por \citeonline{Vaswani2017}, podem produzir resultados inferiores quando comparados a modelos mais simples. Nesse contexto,
% surge então a arquitetura \textit{TSMixer}, proposta por \citeonline{Chen2023}. Essa arquitetura, embora mais simples, produz resultados 
% consideravelmente superiores com uma fração do custo computacional. 


% ----------------------------------------------------------------------------------------------------------------------
\chapter{Fundamentação Teórica}
Neste capítulo, a base teórica do trabalho será apresentada, abordando os principais conceitos neceessários para a compreensão da
metodologia e dos resultados obtidos.
\section{Impacto do ENSO na Geração de Energia Elétrica}
O ENSO é um fenômeno que ocorre no Oceano Pacífico Equatorial, caracterizado por variações na temperatura da superfície
do mar (SST) em regiões específicas, como ilustrado na Figura \ref{fig:regioes_enso}. O fenômeno é um dos principais fatores 
que influenciam os padrões de vento e precipitação em diversas regiões da América do Sul e seus efeitos se estendem por 
todas as regiões do Brasil. \cite{Andreoli2016}

\begin{figure}[!ht]
	\IBGEtab{\caption{Regiões do fenômeno El Niño-Oscilação Sul (ENSO)}
			 \label{fig:regioes_enso}}
	{\includesvg[scale=1]{figuras/regioes_enso_global}}
	{\fonte{o autor, com base em {\citeonline{TheDefinitionofElNio}}}}
\end{figure}

Os impactos em cada região estão resumidos a seguir, de acordo com \citeonline{Capozzoli2017}:
\begin{itemize}
	\item \textbf{Sul:} a região Sul é uma das mais consistentemente afetadas. Eventos de El Niño tendem a causar precipitação 
acima da média, particularmente durante a primavera e o verão, enquanto eventos de La Niña estão associados à condições de seca.
	\item \textbf{Sudeste:} a região Sudeste apresenta uma resposta mais complexa, sendo consdierada uma zona de transição. 
A bacia do Rio Paraná, em especial, apresenta sensibilidade aos fenômenos do ENSO, tendo apresentado tendência de aumento de 
vazão durante alguns eventos de El Niño.
	\item \textbf{Norte/Nordeste:} para as regiões Norte e Nordeste, eventos de El Niño estão associados a períodos de seca, enquanto eventos de
La Niña tendem a trazer chuvas acima da média. No entanto, é importante destacar que outros fenômenos atmosféricos podem
interferir com esses padrões, modulando os efeitos do ENSO.
\end{itemize}

Sendo assim, verifica-se que as variações induzidas pelos fenômenos do ENSO traduzem-se diretamente em variações nas
vazões dos rios que alimentam as bacias, que por sua vez impactam o potencial de geração da fonte hidráulica.


% ----------------------------------------------------------------------------------------------------------------------
\section{O Modelo NEWAVE}
\label{sec:newave}

Desenvolvido e mantido pelo Centro de Pesquisas de Energia Elétrica (CEPEL) e amplamente utilizado pelo setor elétrico 
brasileiro para definição de estratégias e tomada de decisão, o NEWAVE é um modelo de otimização que busca minimizar os 
custos de operação do sistema, considerando a incerteza das afluências futuras e a operação de um sistema 
hidro-térmico-eólico interligado. O modelo é utilizado para estudos como:
\begin{itemize}
	\item Elaboração do Plano Decenal de Expansão de Energia (PDE), pela EPE;
	\item Elaboração do Programa Mensal de Operação (PMO) e Plano de Operação Energética (PEN), pelo ONS;
	\item Formação de preços, como no cálculo do Preço de Liquidação das Diferenças (PLD) pelo CCEE;
	\item Cálculo de Garantia Física e da Energia Assegurada de empreendimentos de geração, pela EPE;
	\item Elaboração de diretrizes para os leilões de energia, pela EPE.
\end{itemize}

Em resumo, o modelo emprega a Programação Dinâmica Dual Estocástica (PDDE), uma técnica de otimização que permite lidar 
com as incertezas ligadas às afluências futuras sem que o modelo se torne computacionalmente impraticável, considerando 
múltiplos reservatórios, interconexões e o horizonte temporal de médio e longo prazos. \cite{Pereira1991}

\subsection{Representação das Usinas}
Para a fonte hidráulica, o NEWAVE modela o sistema de geração em Reservatórios Equivalentes de Energia (REEs), que são grupos de
usinas associadas a um subsistema ou submercado de energia. Cada subsistema pode conter mais de um REE, possibilitando
diferenciar bacias hidrográficas com regimes distintos, ainda que pertençam a um mesmo subsistema. 

Além disso, cada REE é definido por um conjunto de parâmetros que são calculados a partir das características individuais 
de cada usina. Nas versões mais recentes do modelo, também é possível considerar todas as usinas indivudalmente ou operar
de maneira híbrida, ou seja, considerando alguns REEs e outras usinas individualmente.

As usinas termelétricas são representadas no modelo através de classes térmicas. Cada classe agrupa usinas com custos 
semelhantes e está associada a um subsistema. Cada classe também é definida por um conjunto de parâmetros calculados
a partir das características individuais de cada usina.

Nas versões mais recentes do modelo, a fonte eólica também é modelada. De maneira similar, os parques eólicos são agrupados
em Parques Eólicos Equivalentes (PEE). O agrupamento é feito a partir de dados de cadastro de cada prque eólico, estado,
submercado, função de produção (curva relacionando a velocidade do vento com a potência gerada), dados sobre torres de
medição e séries históricas de velocidade do vento.

\subsection{Dados de Entrada}
O modelo requer um conjunto de dados de entrada que inclui as características das usinas, dados dos subsistemas, demanda,
séries históricas de vazões e ventos, cronogramas de expansão, restrições operativas, dentre outros. Observa-se
que todos os dados de entrada são locais e, portanto, o modelo não considera variáveis externas, como fenômenos climáticos
como o EN e LN, que podem impactar a geração de energia elétrica. 

Ainda que as últimas versões do modelo apresentem campos previstos para a entrada de dados do ENSO, esses campos
estão marcados como "não implementados". Dessa forma, entende-se que o modelo não considera diretamente o impacto 
dessas variáveis. No entanto, vale destacar que essas variáveis externas podem ser utilizadas para elaborar as séries históricas de 
vazões e velocidade de ventos utilizadas como dados de entrada. 


% ----------------------------------------------------------------------------------------------------------------------
\newpage
\section{Modelos Linear e Não-linear}
Neste trabalho, foram utilizados modelos de regressão linear e não linear a fim de se obter uma \textit{baseline} para
comparação com o modelo neural.

\subsection{Modelo de Regressão Linear Múltipla}
O modelo linear consiste na regressão linear múltipla e aplicação do método dos mínimos quadrados ordinários para determinar 
os coeficientes da equação linear que melhor se ajusta aos dados, conforme ilustrado na Figura \ref{fig:modelo_linear}.
\begin{figure}[!ht]
	\IBGEtab{\caption{Regressão linear de uma senoide}
			 \label{fig:modelo_linear}}
	{\includesvg[scale=1]{figuras/modelo_linear_exemplo}}
	{\fonte{o autor.}}
\end{figure}

Naturalmente, esse tipo de modelo é utilizado quando se espera que a variável de interesse (dependente) seja uma combinação linear
das variáveis de entrada (independentes), conforme a equação \ref{eq:regressao_linear},
\begin{equation}
\label{eq:regressao_linear}
y = \beta_0 + \beta_1 x_1 + \beta_2 x_2 + ... + \beta_n x_n
\end{equation}
em que $y$ é a variável de interesse, 
$\beta_0$ é o intercepto, $\beta_1, \beta_2, ..., \beta_n$ são os coeficientes e $x_1, x_2, ..., x_n$ são as variáveis de entrada.

O método de mínimos quadrados busca minimizar a soma dos quadrados dos resíduos, ou seja, a diferença entre os valores
observados e os valores previstos pelo modelo. Matematicamente, isso é representado pela equação \ref{eq:minimos_quadrados},
\begin{equation}
\label{eq:minimos_quadrados}
\min_{\beta} \sum_{i=1}^{m} \left(y_i - (\beta_0 + \beta_1 x_{i1} + \beta_2 x_{i2} + ... + \beta_n x_{in})\right)^2
\end{equation}
em que $m$ é o número de observações, $y_i$ é o valor observado da variável de interesse para a $i$-ésima observação,
$x_{ij}$ é o valor da $j$-ésima variável de entrada para a $i$-ésima observação e $\beta$ é o vetor de coeficientes
do modelo.

Ainda que se espere que as relações entre os dados seja altamente não linear, o método é capaz de capturar algumas
das relações lineares entre as variáveis, além de servir como ponto de partida para as próximas análises.


% ----------------------------------------------------------------------------------------------------------------------
\subsection{Modelo de Regressão por Árvores de Decisão}
O modelo funciona a partir de árvores de decisão construídas a partir dos dados de entrada. Cada árvore é treinada em um 
\textit{subset} aleatório dos dados e então a árvore cresce por meio da bifurcação. A estrutura do modelo é ilustrada na 
Figura \ref{fig:modelo_random_forest}.

\begin{figure}[!ht]
	\IBGEtab{\caption{Modelo com 2 árvores de decisão}
			 \label{fig:modelo_random_forest}}
	{\includesvg[scale=1]{figuras/randomforest}}
	{\fonte{o autor.}}
\end{figure}
Durante a bifurcação, o modelo estabelece o melhor caminho através da combinação de subconjuntos das variáveis de entrada, 
de forma a minimizar o erro na previsão da variável de interesse. A previsão final é obtida através da média das previsões 
de todas as árvores, o que ajuda a reduzir o \textit{overfitting} e melhora a generalização do modelo \cite{Mansour2001}. A saída do modelo
é dada por:
\begin{equation}
\label{eq:regressao_random_forest}
\hat{y} = \frac{1}{T} \sum_{t=1}^{T} y_t(x)
\end{equation}
em que $T$ é o número de árvores, $y_t(x)$ é a previsão da $t$-ésima árvore e $\hat{y}$ é a previsão final do modelo.


% ----------------------------------------------------------------------------------------------------------------------
\newpage
\section{Modelo Neural}
Antes de apresentar o modelo neural, será feita uma breve introdução a respeito de alguns conceitos importantes,
como o \textit{Perceptron} e os \textit{Multi Layer Perceptrons} (MLPs). Esses conceitos são fundamentais para compreender
a arquitetura do modelo utilizado.
                  
\subsection{O Perceptron}
O \textit{Perceptron} é um modelo de rede neural artificial proposto por \citeonline{Rosenblatt1958}, inspirado no
funcionamento de neurônios biológicos. A figura \ref{fig:perceptron} ilustra a estrutura básica de um Perceptron.

Um perceptron recebe um conjunto $ X = [x_1, x_2, ..., x_n] $ de entradas,
com cada $ x_i $ associada a um peso aleatório $ w_i $. O Perceptron então calcula a soma ponderada das entradas e aplica uma 
função de ativação para produzir a saída. Matematicamente, isso é dado por:
\begin{equation}
\label{eq:perceptron}
\hat{y}(X) = f\left(\sum_{i=1}^{n} w_i x_i + b\right)
\end{equation}
em que $ \hat{y} $ é a saída do Perceptron e $ f $ a função de ativação que, para esse caso, usa-se a função degrau $ u(t) $.

\begin{figure}[!ht]
	\IBGEtab{\caption{Estrutura do Perceptron}
			 \label{fig:perceptron}}
	{\includesvg[scale=1]{figuras/perceptron}}
	{\fonte{o autor.}}
\end{figure}

O Perceptron é limitado a resolver problemas de classificação linearmente separáveis \cite{Cybenko1989}. Ou seja, problemas nos quais é
possível traçar uma linha (ou hiperplano) que separe as classes de forma clara. Para essas aplicações, o Perceptron
atualiza os pesos de modo iterativo durante o treinamento, através da taxa de aprendizado $ r $, conforme a equação 
\ref{eq:atualizacao_perceptron}, que demonstra a atualização dos pesos no tempo:
\begin{equation}
\label{eq:atualizacao_perceptron}
w_i(t+1) \leftarrow w_i(t) + r (y(t) - \hat{y}(t)) x_i
\end{equation}
em que $ y $ é o valor real e $ \hat{y} $ a saída do Perceptron. A taxa de aprendizado representa o quão rápido
os pesos são atualizados durante o treinamento. A atualização é feita de forma a minimizar o erro entre a saída prevista
e a saída real.


% ----------------------------------------------------------------------------------------------------------------------
\subsection{Multi Layer Perceptrons (MLPs)}
MLPs surgiram como uma evolução dos \textit{Perceptrons} simples, com a finalidade de permitir a modelagem de relações
não lineares, e são a base para o \textit{deep learning}, metodologia que utiliza redes neurais com inúmeras camadas para
resolução de tarefas complexas. A figura \ref{fig:mlp_1} ilustra a estrutura básica de um MLP de dupla camada.

\begin{figure}[!ht]
	\IBGEtab{\caption{MLP de dupla camada}
			 \label{fig:mlp_1}}
	{\includesvg[scale=1]{figuras/mlp_1}}
	{\fonte{o autor.}}
\end{figure}

Essas estruturas são compostas por neurônios (perceptrons) interconectados e organizados em camadas. Cada neurônio de uma 
camada está conectado a todos os outros das camadas adjacentes. A primeira e última camadas são 
chamadas de camada de entrada e camada de saída, respectivamente, enquanto as camadas intermediárias são chamadas de 
camadas ocultas.

MLPs são treinadas utilizando o algoritmo de \textit{backpropagation}, ou retropropagação, que ajusta iterativamente os 
pesos e vieses da rede de forma a minimizar o erro entre a saída prevista e a saída real \cite{Rumelhart1988}. De acordo com 
\citeonline{Goodfellow-et-al-2016}, o algoritmo funciona em um ciclo de quatro etapas:

\newpage
\begin{enumerate}
\item \textbf{Forward Pass:} os dados de entrada são propagados pelos neurônios até a camada de saída. A saída de cada 
neurônio é calculada em duas fases. Primeiro, a entrada ponderada $ z_i^{(l)} $ para o neurônio $ i $ da camada
$ l $ é a soma das saídas da camada anterior, $ v_j^{(l-1)} $, multiplicadas pelos seus respectivos pesos $ w_{ij}^{(l)} $, 
mais um viés $ b_i^{(l)} $:
\begin{equation}
\label{eq:soma_ponderada_u}
z_i^{(l)} = \sum_{j} (w_{ij}^{(l)} v_j^{(l-1)}) + b_i^{(l)}
\end{equation}
Em seguida, a saída ativada $ v_i^{(l)} $ é obtida aplicando-se a função de ativação $ \sigma $:
\begin{equation}
\label{eq:ativacao_v}
v_i^{(l)} = \sigma(z_i^{(l)})
\end{equation}

\item \textbf{Cálculo do Erro:} uma função de perda $ E $ é usada para quantificar o erro entre a saída $ \hat{y} $ da 
rede e o valor real $ y $. Para tarefas de regressão, costuma-se usar o Erro Quadrático Médio (MSE):
\begin{equation}
\label{eq:mse_v}
E = \frac{1}{n} \sum_{i=1}^{n} (y_{i} - \hat{y}_i)^2
\end{equation}
em que $ n $ é o número de neurônios na camada de saída.

\item \textbf{Backward Pass:} o erro $ E $ é propagado de volta pela rede até a primeira camada. Para isso, a regra da cadeia 
é utilizada para determinar o gradiente da função de perda em relação a cada peso e viés da rede \cite{KELLEY1960}. Para a camada 
$ L $ de saída, o erro $ \delta_i^{(L)} $ é calculado diretamente para cada neurônio $ i $:
\begin{equation}
\label{eq:erro_saida_v}
\delta_i^{(L)} = \frac{\partial E}{\partial v_i^{(L)}} \cdot \sigma'(z_i^{(L)})
\end{equation}
em que $\frac{\partial E}{\partial v_i^{(L)}}$ é a derivada da perda em relação à saída do neurônio $i$ e $\sigma'$ é a 
derivada da função de ativação. Caso seja utilizado o MSE e sabendo que $ v_i^{(L)} = \hat{y_i} $, a equação 
\ref{eq:erro_saida_v} torna-se:
\begin{equation}
\label{eq:erro_saida_mse_v}
\delta_i^{(L)} = (\hat{y}_i - y_i) \cdot \sigma'(z_i^{(L)})
\end{equation}

Para as camadas ocultas, o erro $ \delta_i^{(l)} $ para um neurônio $ i $ na camada $ l $ é calculado recursivamente, 
com base nos erros da camada $ l+1 $ seguinte:
\begin{equation}
\label{eq:erro_oculta_v}
\delta_i^{(l)} = \left( \sum_{j} \delta_j^{(l+1)} w_{ji}^{(l+1)} \right) \cdot \sigma'(z_i^{(l)})
\end{equation}
em que a soma percorre todos os neurônios $j$ da camada seguinte, ponderando seus erros $\delta_j^{(l+1)}$ pelos 
pesos $w_{ji}^{(l+1)}$ que os conectam ao neurônio $i$ da camada atual.

Com o termo $\delta_i^{(l)}$ para cada neurônio, os gradientes dos pesos e vieses são encontrados:
\begin{equation}
\label{eq:gradiente_pesos_v}
\frac{\partial E}{\partial w_{ij}^{(l)}} = \delta_i^{(l)} v_j^{(l-1)}
\end{equation}
\begin{equation}
\label{eq:gradiente_bias_v}
\frac{\partial E}{\partial b_i^{(l)}} = \delta_i^{(l)}
\end{equation}
em que $v_j^{(l-1)}$ é a saída do neurônio $j$ da camada anterior $l-1$.

% \newpage
\item \textbf{Atualização dos Pesos:} finalmente, cada peso e viés da rede é atualizado na direção oposta à do seu gradiente, 
de modo a reduzir o erro na próxima iteração:
\begin{equation}
\label{eq:atualizacao_mlp_v}
w_{ij}^{(l)}(t+1) \leftarrow w_{ij}^{(l)}(t) - r \frac{\partial E}{\partial w_{ij}^{(l)}}
\end{equation}
\begin{equation}
\label{eq:atualizacao_bias_v}
b_i^{(l)}(t+1) \leftarrow b_i^{(l)}(t) - r \frac{\partial E}{\partial b_i^{(l)}}
\end{equation}
em que $t$ representa a iteração atual e $r$ é a taxa de aprendizado, que controla a magnitude da atualização.
\end{enumerate}

Em resumo, o processo de aprendizado de um MLP envolve a propagação dos dados de entrada pela rede, o cálculo do erro entre 
a saída prevista e a real, a retropropagação do erro para calcular os gradientes e a atualização dos pesos e vieses para 
minimizar o erro. Esse processo é repetido até que a rede atinja um nível satisfatório de desempenho ou até que um critério
de parada seja atingido, como um número máximo de iterações ou uma tolerância de erro.	


% ----------------------------------------------------------------------------------------------------------------------
\subsection{Arquitetura TSMixer}
A arquitetura proposta por \citeonline{Chen2023} é uma abordagem inovadora para previsão de séries temporais. Ela utiliza
MLPs em cascata, denominadas de \textit{Mixing Layers}, para capturar as relações na dimensão temporal, bem como na dimensão das 
características (variáveis).

A mistura no domínio do tempo permite ao modelo capturar os padrões temporais da série. Essa abordagem se mostrou 
eficaz para aprender padrões temporais como sazonalidades e tendências, sem a necessidade de mecanismos de atenção, 
como os utilizados na arquitetura \textit{Transformer}, que possui mecanismos de atenção que são computacionalmente taxantes.

A mistura no domínio das características permite ao modelo capturar as relações entre as variáveis em cada instante de tempo.
Ou seja, o modelo é capaz de aprender como as variáveis interagem entre si ao longo da série temporal. Essa abordagem é 
eficaz para capturar correlações e dependências, sem a necessidade de mecanismos de atenção.

A arquitetura emprega as técnicas de \textit{dropout} e resíduo durante o treinamento. O objetivo é evitar \textit{overfitting}
e melhorar a generalização do modelo. A cada passo de treinamento (época), o \textit{dropout} desliga aleatoriamente um
número de neurônios em uma camada. Dessa forma, a rede é forçada a não depender de um conjunto pequeno de neurônios,
fazendo com que o aprendizado seja distribuído entre entre os demais neurônios da camada. Essa abordagem diminui a chance
do modelo apenas memorizar os dados de treinamento (\textit{overfitting}).

A técnica de resíduo é usada para permitir que a entrada de uma camada seja somada à sua saída. O objetivo é manter o
processo de aprendizado eficiente, evitando o efeito de \textit{vanishing gradient}, no qual o gradiente do erro se torna
muito pequeno ao ser propagado às camadas iniciais pelo algoritmo de \textit{backpropagation}. Esse efeito pode ocorrer 
em redes com muitas camadas quando as derivadas parciais da função de erro podem ser muito pequenas para as camadas iniciais.

\begin{figure}[!ht]
	\IBGEtab{\caption{Arquitetura do modelo neural usado}
			 \label{fig:arquitetura_modelo_neural}}
	{\includesvg[scale=1]{figuras/tsmixer_1}}
	{\fonte{o autor, adaptado de \citeonline{Chen2023}.}}
\end{figure}

A figura \ref{fig:arquitetura_modelo_neural} ilustra a arquitetura do modelo. As colunas das entradas representam diferentes
variáveis e as linhas são os instantes de tempo. As operações de mistura são realizadas linha a linha. 

As MLPs no domínio do tempo tem seus pesos compartilhados entre todas a todas as variáveis, enquanto as MLPs no domínio das 
características tem seus pesos compartilhados entre todos os instantes de tempo. Para a projeção final, uma camada totalmente 
conectada é usada para mapear o tamanho da entrada (contexto) para o tamanho da saída (previsão). As etapas estão descritas a seguir:

\begin{enumerate}
	\item \textbf{Mixing Temporal}: modela padrões temporais da série. Constitui uma única camada de neurônios totalmente
conectada (MLP de camada única), seguida de uma função de ativação e \textit{dropout}. A entrada é transposta para aplicar 
as conexões no domínio do tempo, ou seja, as entradas $ X_i = [x_i(t_1), x_i(t_2), \dots x_i(t_n)] $ são cada variável em todos 
os instantes de tempo, permitindo que o modelo capture os padrões temporais de cada variável utilizando os mesmos pesos.
A saída é transposta novamente para manter a forma original da entrada.

	\item \textbf{Mixing de Características}: modela relações entre as variáveis em cada instante de tempo. Constitui
duas camadas totalmente conectadas (MLPs de camada dupla), também com uma função de ativação e \textit{dropout}. As conexões
são aplicadas no domínio das características, ou seja, as entradas $ T_i = [t_i(x_1), t_i(x_2), \dots t_i(x_n)] $ são
todas as variáveis em um instante de tempo, permitindo ao modelo capturar as relações entre elas, também utilizando os
mesmos pesos.

	\item \textbf{Projeção Temporal}: projetam a saída, mapeando o tamanho da janela de entrada (contexto) para o tamanho 
da janela de saída (previsão). Constitui uma camada totalmente conectada aplicada no domínio do tempo.
\end{enumerate}

\citeonline{Chen2023} também propõem uma variante (\textit{TSMixer-Ext}) para uso de variáveis exógenas, que são variáveis auxiliares que podem
explicar o comportamento da variável de interesse. Considerando que essa variante ainda não foi disponibilizada publicamente,
para esse projeto foi utilizada uma implementação da arquitetura TSMixer com suporte a variáveis exógenas,
disponibilizada por \citeonline{Ekambaram2024}, denominada de \textit{Tiny Time Mixer} (TTM).

\subsubsection{Extensão para Variáveis Exógenas (TSMixer-Ext)}
Essa variante da arquitetura é capaz de trabalhar com recursos estáticos, que não variam ao longo do tempo, como
localização e características da usina; e também recursos variáveis no tempo, como índices climáticos. A TSMixer-Ext 
introduz os processos de Alinhamento e Mistura, antes de aplicar as camadas de mistura principais:

\begin{enumerate}
\item \textbf{Alinhamento:} o objetivo é fazer com que todas as entradas (históricas, futuras e estáticas) tenham o mesmo 
comprimento de sequência para que possam ser combinadas. Recursos estáticos, por exemplo, são repetidos ao longo da 
dimensão do tempo para se alinharem ao horizonte de previsão. Após o alinhamento, todas as variáveis são concatenadas 
em uma única matriz de características.
\item \textbf{Mistura:} a matriz de características combinada é então processada pelas mesmas Mixing Layers 
(temporal e de características) descritas anteriormente. Essa abordagem permite que o modelo aprenda não apenas os 
padrões da série principal, mas também as complexas interações e influências que as variáveis exógenas exercem sobre ela.
\end{enumerate}

No contexto deste projeto, as variáveis exógenas são os dados referentes ao fenômeno ENSO, ou seja, os dados de SST.