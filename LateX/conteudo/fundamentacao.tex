\chapter{Trabalhos Relacionados}
Os impactos do fenômeno ENSO vêm sendo estudados extensamente em diversas áreas. Segundo \citeonline{Cirino2015}, eventos
de El Niño e La Niña influenciam significativamente a agricultura brasileira, especialmente nas regiões Sul e Nordeste.

Um estudo de \citeonline{Pirani2024} demonstrou que a ocorrência do fenômeno El Niño está associada a uma maior taxa de 
casos de dengue no estado de São Paulo, devido a um aumento na temperatura e precipitação, condições
favoráveis para a proliferação do mosquito Aedes aegypti.

Segundo \citeonline{Capozzoli2017}, o fenômeno ENSO tem uma relação direta sobre a disponibilidade de recursos hídricos
nas bacias hidrográficas brasileiras, o que sugere um impacto direto sobre a geração hidrelétrica no Brasil. Além disso,
os resultados são coerentes com a literatura, indicando impactos distintos nas diferentes regiões do país.

Naturalmente, o fenômeno ENSO também vem sendo estudado no contexto do setor elétrico brasileiro. Segundo um
estudo da \citeonline{epe2023}, a ocorrência do fenômeno La Niña em 2021 foi um fator determinante para a crise hídrica
que afetou o Brasil nesse período, em decorrência da redução das afluências, ou seja, a quantidade de água que chega aos
reservatórios das usinas hidrelétricas.

Segundo o relatório da Comissão Permanente para Análise de Metodologias e Programas Computacionais do Setor Elétrico (CPAMP),
constituída por instituições do setor elétrico como EPE, ONS e CEPEL, a incorporação de dados de variáveis climáticas,
como o fenômeno ENSO, aos modelos computacionais é ativamente discutida devido a relação entre as séries históricas de 
vazões e dos ciclos de índices climáticos. \cite{cpamp2019}

De acordo com \citeonline{Resende2018}, o uso de modelos de aprendizado de máquina para previsão de carga do SIN tem o
potencial de aprimorar o resultado das previsões, reduzindo os desvios de previsão de carga e, consequentemente, uma redução 
significativa dos custos de operação do sistema elétrico. Infere-se, portanto, que essa abordagem também poderia
ser aplicada para estimar outros parâmetros, como a geração futura.

Para isso, é essencial selecionar modelos que sejam capazes de capturar as relações potencialmente complexas entre os
dados de geração e os fenômenos climáticos. Considerando que a literatura sugere que essa relação seja altamente não-linear,
modelos neurais seriam uma escolha natural, mas não necessariamente os modelos neurais mais avançados seriam os mais adequados.

Segundo \citeonline{Zeng2022}, modelos de previsão de séries temporais baseados na arquitetura \textit{Transformer}, introduzida 
por \citeonline{Vaswani2017}, podem produzir resultados inferiores quando comparados a modelos mais simples. Nesse contexto,
surge então a arquitetura \textit{TSMixer}, proposta por \citeonline{Chen2023}. Essa arquitetura, embora mais simples, produz resultados 
consideravelmente superiores com uma fração do custo computacional. 

% ----------------------------------------------------------------------------------------------------------------------
\chapter{Fundamentação Teórica}
\section{Impacto do ENSO na Geração de Energia Elétrica}
O ENSO é um fenômeno que ocorre no Oceano Pacífico Equatorial, caracterizado por variações na temperatura da superfície
do mar (TSM) em regiões específicas, como ilustrado na Figura \ref{fig:regioes_enso}. O fenômeno é um dos principais fatores 
que influenciam os padrões de vento e precipitação em diversas regiões da América do Sul e seus efeitos se estendem por 
todas as regiões do Brasil. \cite{Andreoli2016}

\begin{figure}[!ht]
	\IBGEtab{\caption{Regiões do fenômeno El Niño-Oscilação Sul (ENSO)}
			 \label{fig:regioes_enso}}
	{\includesvg[scale=1]{figuras/regioes_enso_global}}
	{\fonte{o autor.}}
\end{figure}

Os impactos em cada região estão resumidos a seguir, de acordo com \citeonline{Capozzoli2017}:
\begin{itemize}
	\item \textbf{Sul:} A região Sul é uma das mais consistentemente afetadas. Eventos de El Niño tendem a causar precipitação acima da média,
particularmente durante a primavera e o verão, enquanto eventos de La Niña estão associados à condições de seca.
	\item \textbf{Sudeste:} A região Sudeste apresenta uma resposta mais complexa, sendo consdierada uma zona de transição. A bacia do Rio Paraná,
em especial, apresenta sensibilidade aos fenômenos do ENSO, tendo apresentado tendência de aumento de vazão durante alguns
eventos de El Niño.
	\item \textbf{Norte/Nordeste:} Para as regiões Norte e Nordeste, eventos de El Niño estão associados a períodos de seca, enquanto eventos de
La Niña tendem a trazer chuvas acima da média. No entanto, é importante destacar que outros fenômenos atmosféricos podem
interferir com esses padrões, modulando os efeitos do ENSO.
\end{itemize}

Sendo assim, verifica-se que as variações induzidas pelos fenômenos do ENSO traduzem-se diretamente em variações nas
vazões dos rios que alimentam as bacias, que por sua vez impactam o potencial de geração da fonte hidráulica.

% ----------------------------------------------------------------------------------------------------------------------
\section{O Modelo NEWAVE}
\label{sec:newave}

Desenvolvido e mantido pelo Centro de Pesquisas de Energia Elétrica (CEPEL) e amplamente utilizado pelo setor elétrico 
brasileiro para definição de estratégias e tomada de decisão, o NEWAVE é um modelo de otimização que busca minimizar os 
custos de operação do sistema, considerando a incerteza das afluências futuras e a operação de um sistema 
hidro-térmico-eólico interligado. O modelo é utilizado para estudos como:
\begin{itemize}
	\item Elaboração do Plano Decenal de Expansão de Energia (PDE), pela EPE;
	\item Elaboração do Programa Mensal de Operação (PMO) e Plano de Operação Energética (PEN), pelo ONS;
	\item Formação de preços, como no cálculo do Preço de Liquidação das Diferenças (PLD) pelo CCEE;
	\item Cálculo de Garantia Física e da Energia Assegurada para empreendimentos de geração participantes nos leilões 
    de energia elétrica, pela EPE;
	\item Elaboração de diretrizes para os leilões de energia, pela EPE.
\end{itemize}

Em resumo, o modelo emprega a Programação Dinâmica Dual Estocástica (PDDE), uma técnica de otimização que permite lidar 
com as incertezas ligadas às afluências futuras sem que o modelo se torne computacionalmente impraticável, considerando 
múltiplos reservatórios, interconexões e o horizonte temporal de médio e longo prazos.

\subsection{Representação das Usinas}
O NEWAVE modela o sistema de geração hidrelétrico em Reservatórios Equivalentes de Energia (REEs), que são grupos de
usinas associadas a um subsistema ou submercado de energia. Cada subsistema pode conter mais de um REE, possibilitando
diferenciar bacias hidrográficas com regimes distintos, ainda que pertençam a um mesmo subsistema. 

Além disso, cada REE é definido por um conjunto de parâmetros que são calculados a partir das características indivuduais 
de cada usina. Nas versões mais recentes do modelo, também é possível considerar todas as usinas indivudalmente ou operar
de maneira híbrtida, ou seja, considerando alguns REEs e outras usinas individualmente.

As usinas termelétricas são representadas no modelo através de classes térmicas. Cada classe agrupa usinas com custos 
semelhantes e está associada a um subsistema. Cada classe também é definida por um conjunto de parâmetros calculados
a partir das características individuais de cada usina.

Nas versões mais recentes do modelo, a fonte eólica também é modelada. De maneira similar, os parques eólicos são agrupados
em Parques Eólicos Equivalentes (PEE). O agrupamento é feito a aprtir de dados de cadastro de cada prque eólico, estado,
submercado, função de produção (curva relacionando a velocidade do vento com a potência gerada), dados sobre torres de
medição e séries históricas de velocidade do vento.

\subsection{Dados de Entrada}
O modelo requer um conjunto de dados de entrada que inclui as características das usinas, dados dos subsistemas, demanda,
séries históricas de vazões e ventos, cronogramas de expansão, restrições operativas, dentre outros. Observa-se, portanto, 
que todos os dados de entrada são locais e, portanto, o modelo não considera variáveis externas, como fenomênos climáticos
como o EN e LN, que podem impactar a geração de energia elétrica. 

Ainda que as últimas versões do modelo apresentem campos previstos para a entrada de dados do ENSO, esses campos
estão marcados como "não implementados". Dessa forma, entende-se que o modelo não considera diretamente o impacto 
dessas variáveis. No entanto, vale destacar que essas variáveis externas podem ser utilizadas para elaborar as séries históricas de 
vazões e velocidade de ventos utilizadas como dados de entrada. 

% ----------------------------------------------------------------------------------------------------------------------
\section{Modelos Linear e Não-linear}
Nesse projeto, foram utilizados modelos de regressão linear e não linear a fim de se obter uma \textit{baseline} para
comparação com o modelo neural. Para ambos os casos, a biblioteca \textit{scikit-learn} foi utilizada.

\subsection{Modelo Linear}
O modelo linear utilizado foi o \textit{LinearRegression} da biblioteca \textit{scikit-learn}. Ele consiste na
aplicação do método dos mínimos quadrados para determinar os coeficientes da equação linear que melhor se ajusta aos dados,
conforme ilustrado na Figura \ref{fig:modelo_linear}.
\begin{figure}[!ht]
	\IBGEtab{\caption{Regressão Linear de uma Senoide}
			 \label{fig:modelo_linear}}
	{\includesvg[scale=1]{figuras/modelo_linear_exemplo}}
	{\fonte{o autor.}}
\end{figure}

Naturalmente, esse tipo de modelo é utilizado quando se espera que a variável de interesse (dependente) seja uma combinação linear
das variáveis de entrada (independentes), conforme a equação \ref{eq:regressao_linear},
\begin{equation}
\label{eq:regressao_linear}
y = \beta_0 + \beta_1 x_1 + \beta_2 x_2 + ... + \beta_n x_n
\end{equation}
em que $y$ é a variável de interesse, 
$\beta_0$ é o intercepto, $\beta_1, \beta_2, ..., \beta_n$ são os coeficientes e $x_1, x_2, ..., x_n$ são as variáveis de entrada.

O método de mínimos quadrados busca minimizar a soma dos quadrados dos resíduos, ou seja, a diferença entre os valores
observados e os valores previstos pelo modelo. Matematicamente, isso é representado pela equação \ref{eq:minimos_quadrados},
\begin{equation}
\label{eq:minimos_quadrados}
\min_{\beta} \sum_{i=1}^{m} \left(y_i - (\beta_0 + \beta_1 x_{i1} + \beta_2 x_{i2} + ... + \beta_n x_{in})\right)^2
\end{equation}
em que $m$ é o número de observações, $y_i$ é o valor observado da variável de interesse para a $i$-ésima observação,
$x_{ij}$ é o valor da $j$-ésima variável de entrada para a $i$-ésima observação e $\beta$ é o vetor de coeficientes
do modelo.

Ainda que se espere que as relações entre os dados seja altamente não linear, o método é capaz de capturar algumas
das relações lineares entre as variáveis, além de servir como ponto de partida para as próximas análises.

\subsection{Modelos Não-lineares: Random Forest e Gradient Boosting}


% ----------------------------------------------------------------------------------------------------------------------
\section{Modelo Neural}
Antes de apresentar o modelo neural, será feita uma breve introdução a respeito de alguns conceitos importantes,
como o \textit{Perceptron} e os \textit{Multi Layer Perceptrons} (MLPs). Esses conceitos são fundamentais para compreender
a arquitetura do modelo utilizado.

\subsection{O Perceptron}
O \textit{Perceptron} é um modelo de rede neural artificial proposto por \citeonline{Rosenblatt1958}, inspirado no
funcionamento de neurônios biológicos. A figura \ref{fig:perceptron} ilustra a estrutura básica de um Perceptron.

\begin{figure}[!ht]
	\IBGEtab{\caption{Estrutura do Perceptron}
			 \label{fig:perceptron}}
	{\includesvg[scale=1]{figuras/perceptron}}
	{\fonte{o autor.}}
\end{figure}

Um perceptron recebe um conjunto $ X = [x_1, x_2, ..., x_n] $ de entradas,
com cada $ x_i $ associada a um peso aleatório $ w_i $. O Perceptron então calcula a soma ponderada das entradas e aplica uma 
função de ativação para produzir a saída. Matematicamente, isso é dado por:
\begin{equation}
\label{eq:perceptron}
\hat{y}(X) = f\left(\sum_{i=1}^{n} w_i x_i + b\right)
\end{equation}
em que $ \hat{y} $ é a saída do Perceptron e $ f $ a função de ativação que, para esse caso, usa-se a função degrau $ u(t) $.

O Perceptron é limitado a resolver problemas de classificação linearmente separáveis. Ou seja, problemas nos quais é
possível traçar uma linha (ou hiperplano) que separe as classes de forma clara. Para essas aplicações, o Perceptron
atualiza os pesos de modo iterativo durante o treinamento, através da taxa de aprendizado $ r $, conforme a equação 
\ref{eq:atualizacao_perceptron}, que demonstra o aprendizado através da atualização dos pesos no tempo:
\begin{equation}
\label{eq:atualizacao_perceptron}
w_i(t+1) \leftarrow w_i(t) + r (y(t) - \hat{y}(t)) x_i
\end{equation}
em que $ y $ é o valor real e $ \hat{y} $ a saída do Perceptron. A taxa de aprendizado representa o quão rápido
os pesos são atualizados durante o treinamento. A atualização é feita de forma a minimizar o erro entre a saída prevista
e a saída real.

Para problemas mais complexos, como aqueles que envolvem relações não-lineares entre as variáveis, outras abordagens
são necessárias.


\subsection{Multi Layer Perceptrons (MLPs)}
MLPs surgiram como uma evolução dos \textit{Perceptrons} simples, com a finalidade de permitir a modelagem de relações
não lineares, e são a base para o \textit{deep learning}, metodologia que utiliza redes neurais com inúmeras camadas para
resolução de tarefas complexas. A figura \ref{fig:mlp_1} ilustra a estrutura básica de um MLP de dupla camada.

\begin{figure}[!ht]
	\IBGEtab{\caption{MLP de dupla camada}
			 \label{fig:mlp_1}}
	{\includesvg[scale=1]{figuras/mlp_1}}
	{\fonte{o autor.}}
\end{figure}

Essas estruturas são compostas por neorônios (perceptrons) interconectados e organizados em camadas. Cada neurônio de uma camada está 
conectado a todos os neurônios das camadas adjacentes (\textit{fully connected}). A primeira e última camadas são chamadas de 
camada de entrada e camada de saída, respectivamente, enquanto as camadas intermediárias são chamadas de camadas ocultas.

MLPs são treinadas utilizando o algoritmo de \textit{backpropagation}, ou retropropagação, usado para atualizar os pesos
e viéses dos neurônios de forma a minimizar o erro entre a saída prevista e a saída real. O algoritmo de retropropagação
funciona da seguinte forma:
\begin{itemize}
	\item \textbf{Forward Pass:} Os dados de entrada são propagados através da rede, camada por camada, até a camada de saída.
	\item \textbf{Cálculo do Erro:} O erro é calculado comparando a saída prevista com a saída real.
	\item \textbf{Backward Pass:} O erro é propagado de volta através da rede, camada por camada, para calcular os gradientes dos pesos e viéses.
	\item \textbf{Atualização dos Pesos:} Os pesos e viéses são atualizados utilizando os gradientes calculados e uma taxa de aprendizado.
\end{itemize}

Esse processo é repetido por várias iterações (épocas) até que o modelo converja para uma solução satisfatória.


% ----------------------------------------------------------------------------------------------------------------------
\subsection{Arquitetura TSMixer}
A arquitetura proposta por \citeonline{Chen2023} é uma abordagem inovadora para previsão de séries temporais. Ela utiliza
camadas de \textit{Multi Layer Perceptrons} (MLPs) em cascata, denominadas de \textit{Mixing Layers}, para capturar as relações
na dimensão temporal, bem como na dimensão das características (variáveis). A figura \ref{fig:arquitetura_modelo_neural} ilustra 
a arquitetura do modelo.

A mistura no domínio do tempo permite ao modelo capturar os padrões temporais da série estudada. Essa etapa consiste
em uma MLP de camada única que opera ao longo da dimensão do tempo. Essa abordagem se mostrou eficaz para aprender
padrões temporais complexos, como sazonalidades e tendências, sem a necessidade de mecanismos de atenção, como os utilizados
na arquitetura \textit{Transformer}.

A mistura no domínio das características permite ao modelo capturar as relações entre as variáveis. Essa etapa consiste em
MLPs de camada dupla que operam ao longo da dimensão das características. Ou seja, as relações entre as váriaveis em um
determinado instante de tempo são capturadas. Essa abordagem é eficaz para aprender interações entre variáveis, como
correlações e dependências, sem a necessidade de mecanismos de atenção.

\begin{figure}[!ht]
	\IBGEtab{\caption{Arquitetura do modelo neural usado}
			 \label{fig:arquitetura_modelo_neural}}
	{\includesvg[scale=.3]{figuras/tsmixer-1_edit}}
	{\fonte{Adaptado de \citeonline{Chen2023}}}
\end{figure}





\subsection{Modelo Tiny Time Mixer}