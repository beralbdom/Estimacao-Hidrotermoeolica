% Apresentar estudos que contemple a temática abordada. Respeitar a autoria,
% nas citações diretas e indiretas. Evitar parágrafos muito longos. Evitar seções e
% subseções muito curtas.

\chapter{Fundamentação Teórica}
\section{Impacto do ENSO na Geração de Energia Elétrica}
O ENSO é um fenômeno que ocorre no Oceano Pacífico Equatorial, caracterizado por variações na temperatura da superfície
do mar (TSM) em regiões específicas, como ilustrado na Figura \ref{fig:regioes_enso}. Segundo \citeonline{Andreoli2016}, o 
fenômeno é um dos principais fatores que influenciam os padrões de vento e precipitação em diversas regiões da América 
do Sul e seus efeitos se estendem por todas as regiões do Brasil.

De acordo com \citeonline{Capozzoli2017}, diferentes fases do ENSO resultam em padrões distintos de precipitação sobre
o território brasileiro, variando de acordo com a região e a estação do ano. Sendo assim, há um impacto direto sobre a 
disponibilidade de recursos hídricos para a geração hidrelétrica.

A região Sul é uma das mais consistentemente afetadas. Eventos de El Niño tendem a causar precipitação acima da média,
particularmente durante a primavera e o verão, enquanto eventos de La Niña estão associados à condições de seca.

A região Sudeste apresenta uma resposta mais complexa, sendo consdierada uma zona de transição. A bacia do Rio Paraná,
em especial, apresenta sensibilidade aos fenômenos do ENSO, tendo apresentado tendência de aumento de vazão durante alguns
eventos de El Niño.

Para as regiões Norte e Nordeste, eventos de El Niño estão associados a períodos de seca, enquanto eventos de
La Niña tendem a trazer chuvas acima da média. No entanto, é importante destacar que outros fenômenos atmosféricos podem
interferir com esses padrões, modulando os efeitos do ENSO.

Sendo assim, verifica-se que as variações induzidas pelos fenômenos do ENSO traduzem-se diretamente em variações nas
vazões dos rios que alimentam as bacias, que por sua vez impactam diretamente o potencial de geração da fonte hidráulica.

\begin{figure}[!ht]
	\IBGEtab{\caption{Regiões do fenômeno El Niño-Oscilação Sul (ENSO)}
			 \label{fig:regioes_enso}}
	{\includesvg[scale=1]{figuras/regioes_enso_global}}
	{\fonte{o autor.}}
\end{figure}

\section{O Modelo NEWAVE}
\label{sec:newave}

Desenvolvido e mantido pelo Centro de Pesquisas de Energia Elétrica (CEPEL) e amplamente utilizado pelo setor elétrico 
brasileiro para definição de estratégias e tomada de decisão, o NEWAVE é um modelo de otimização que busca minimizar os 
custos de operação do sistema, considerando a incerteza das afluências futuras e a operação de um sistema 
hidro-térmico-eólico interligado. O modelo é utilizado para estudos como:
\begin{itemize}
	\item Elaboração do Plano Decenal de Expansão de Energia (PDE), pela EPE;
	\item Elaboração do Programa Mensal de Operação (PMO) e Plano de Operação Energética (PEN), pelo ONS;
	\item Formação de preços, como no cálculo do Preço de Liquidação das Diferenças (PLD) pelo CCEE;
	\item Cálculo de Garantia Física e da Energia Assegurada para empreendimentos de geração participantes nos leilões 
    de energia elétrica, pela EPE;
	\item Elaboração de diretrizes para os leilões de energia, pela EPE.
\end{itemize}

Em resumo, o modelo emprega a Programação Dinâmica Dual Estocástica (PDDE), uma técnica de otimização que permite lidar 
com as incertezas ligadas às afluências futuras sem que o modelo se torne computacionalmente impraticável, considerando 
múltiplos reservatórios, interconexões e o horizonte temporal de médio e longo prazos.

\subsection{Representação das Usinas}
O NEWAVE modela o sistema de geração hidrelétrico em Reservatórios Equivalentes de Energia (REEs), que são grupos de
usinas associadas a um subsistema ou submercado de energia. Cada subsistema pode conter mais de um REE, possibilitando
diferenciar bacias hidrográficas com regimes distintos, ainda que pertençam a um mesmo subsistema. 

Além disso, cada REE é definido por um conjunto de parâmetros que são calculados a partir das características indivuduais 
de cada usina. Nas versões mais recentes do modelo, também é possível considerar todas as usinas indivudalmente ou operar
de maneira híbrtida, ou seja, considerando alguns REEs e outras usinas individualmente.

As usinas termelétricas são representadas no modelo através de classes térmicas. Cada classe agrupa usinas com custos 
semelhantes e está associada a um subsistema. Cada classe também é definida por um conjunto de parâmetros calculados
a partir das características individuais de cada usina.

Nas versões mais recentes do modelo, a fonte eólica também é modelada. De maneira similar, os parques eólicos são agrupados
em Parques Eólicos Equivalentes (PEE). O agrupamento é feito a aprtir de dados de cadastro de cada prque eólico, estado,
submercado, função de produção (curva relacionando a velocidade do vento com a potência gerada), dados sobre torres de
medição e séries históricas de velocidade do vento.

\subsection{Dados de Entrada}
O modelo requer um conjunto de dados de entrada que inclui as características das usinas, dados dos subsistemas, demanda,
séries históricas de vazões e ventos, cronogramas de expansão, restrições operativas, dentre outros. Observa-se, portanto, 
que todos os dados de entrada são locais e, portanto, o modelo não considera variáveis externas, como fenomênos climáticos
como o EN e LN, que podem impactar a geração de energia elétrica. 

Ainda que as últimas versões do modelo apresentem campos previstos para a entrada de dados do ENSO, esses campos
estão marcados como "não implementados". Dessa forma, entende-se que o modelo não considera diretamente o impacto 
dessas variáveis. No entanto, vale destacar que essas variáveis externas podem ser utilizadas para elaborar as séries históricas de 
vazões e velocidade de ventos utilizadas como dados de entrada. 

\section{Modelos Linear e Não-linear}
fazer uma introdução sobre os modelos lineares e não lineares...

\subsection{Modelo Linear}

\subsection{Modelos Não-lineares: Random Forest e Gradient Boosting}

\section{Modelo Neural}
Atualmente, os modelos neurais mais avançados utilizam a arquitetura \textit{Transformer}, que são modelos neurais com um
mecanismo de atenção que permite ao modelo focar em diferentes partes da entrada de dados ao processar informações. Essa
arquitetura foi introduzida no artigo \textit{"Attention is All You Need"} de \citeonline{Vaswani2017} e hoje é a arquitetura
por trás dos principais \textit{Large Language Models} (LLMs) comercialmente disponíveis.

Inicialmente concebido para tarefas de tradução, modelos baseados nessa arquitetura demonstraram resultados superiores
em outras aplicações e hoje são amplamente utilizados em diversas áreas, incluindo previsão de séries temporais. No entanto,
esses modelos são altamente complexos e exigem um alto poder computacional para treinamento e inferência.

Além disso, conforme demonstrado por \citeonline{Zeng2022}, modelos baseados em \textit{Transformers} podem produzir
resultados inferiores quando comparados a modelos mais simples. 

Dessa forma, uma abordagem alternativa foi proposta por \citeonline{Chen2023}, uma arquitetura mais simples em comparação
aos \textit{Transformers}, mas com resultados bastante superiores

