% Apresentar estudos que contemple a temática abordada. Respeitar a autoria,
% nas citações diretas e indiretas. Evitar parágrafos muito longos. Evitar seções e
% subseções muito curtas.
\chapter{Trabalhos Relacionados}
Os impactos do fenômeno ENSO vêm sendo estudados extensamente em diversas áreas. Segundo \citeonline{Cirino2015}, eventos
de El Niño e La Niña influenciam significativamente a agricultura brasileira, especialmente nas regiões Sul e Nordeste.

Um estudo de \citeonline{Pirani2024} demonstrou que a ocorrência do fenômeno El Niño está associada a uma maior taxa de 
ocorrência de casos de dengue no estado de São Paulo, devido a um aumento na temperatura e precipitação, condições
favoráveis para a proliferação do mosquito Aedes aegypti.

Segundo \citeonline{Capozzoli2017}, o fenômeno ENSO tem uma relação direta sobre a disponibilidade de recursos hídricos
nas bacias hidrográficas brasileiras, o que sugere um impacto direto sobre a geração hidrelétrica no Brasil. Além disso,
os resultados são coerentes com a literatura, indicando impactos distintos nas diferentes regiões do país.

Naturalmente, o fenômeno ENSO também vem sendo estudado no contexto do setor elétrico brasileiro. Segundo um
estudo da \citeonline{epe2023}, a ocorrência do fenômeno La Niña em 2021 foi um fator determinante para a crise hídrica
que afetou o Brasil nesse período, em decorrência da redução das afluências, ou seja, a quantidade de água que chega aos
reservatórios das usinas hidrelétricas.

Segundo o relatório da Comissão Permanente para Análise de Metodologias e Programas Computacionais do Setor Elétrico (CPAMP),
constituída por instituições do setor elétrico como EPE, ONS e CEPEL, a incorporação de dados de variáveis climatológicas,
como o fenômeno ENSO, aos modelos computacionais é ativamente discutida devido a relação entre as séries históricas de 
vazões e dos ciclos de índices climáticos. \cite{cpamp2019}

De acordo com \citeonline{Resende2018}, o uso de modelos de aprendizado de máquina para previsão de carga do SIN tem o
potencial de aprimorar o resultado das previsões, reduzindo os desvios de previsão de carga e, consequentemente, uma redução 
significativa dos custos de operação do sistema elétrico. Infere-se, portanto, que essa abordagem também poderia
ser aplicada para estimar outros parâmetros, como a geração futura.

Para isso, é essencial selecionar modelos que sejam capazes de capturar as relações potencialmente complexas entre os
dados de geração e os fenômenos climáticos. Considerando que a literatura sugere que essa relação seja altamente não-linear,
modelos neurais seriam uma escolha natural, mas não necessariamente os modelos neurais mais avançados seriam os mais adequados.

Segundo \citeonline{Zeng2022}, modelos de previsão de séries temporais baseados na arquitetura \textit{Transformer}, introduzida 
por \citeonline{Vaswani2017}, podem produzir resultados inferiores quando comparados a modelos mais simples, como os baseados 
na arquitetura \textit(TSMixer), proposta por \citeonline{Chen2023}. Essa arquitetura, embora mais simples, produz resultados 
consideravelmente superiores com uma fração do custo computacional. 


\chapter{Fundamentação Teórica}
\section{Impacto do ENSO na Geração de Energia Elétrica}
O ENSO é um fenômeno que ocorre no Oceano Pacífico Equatorial, caracterizado por variações na temperatura da superfície
do mar (TSM) em regiões específicas, como ilustrado na Figura \ref{fig:regioes_enso}. Segundo \citeonline{Andreoli2016}, o 
fenômeno é um dos principais fatores que influenciam os padrões de vento e precipitação em diversas regiões da América 
do Sul e seus efeitos se estendem por todas as regiões do Brasil.

\begin{figure}[!ht]
	\IBGEtab{\caption{Regiões do fenômeno El Niño-Oscilação Sul (ENSO)}
			 \label{fig:regioes_enso}}
	{\includesvg[scale=1]{figuras/regioes_enso_global}}
	{\fonte{o autor.}}
\end{figure}

Os impactos em cada região estão resumidos a seguir:
\begin{itemize}
	\item \textbf{Sul:} A região Sul é uma das mais consistentemente afetadas. Eventos de El Niño tendem a causar precipitação acima da média,
particularmente durante a primavera e o verão, enquanto eventos de La Niña estão associados à condições de seca.
	\item \textbf{Sudeste:} A região Sudeste apresenta uma resposta mais complexa, sendo consdierada uma zona de transição. A bacia do Rio Paraná,
em especial, apresenta sensibilidade aos fenômenos do ENSO, tendo apresentado tendência de aumento de vazão durante alguns
eventos de El Niño.
	\item \textbf{Norte/Nordeste:} Para as regiões Norte e Nordeste, eventos de El Niño estão associados a períodos de seca, enquanto eventos de
La Niña tendem a trazer chuvas acima da média. No entanto, é importante destacar que outros fenômenos atmosféricos podem
interferir com esses padrões, modulando os efeitos do ENSO.
\end{itemize}

Sendo assim, verifica-se que as variações induzidas pelos fenômenos do ENSO traduzem-se diretamente em variações nas
vazões dos rios que alimentam as bacias, que por sua vez impactam diretamente o potencial de geração da fonte hidráulica.

\section{O Modelo NEWAVE}
\label{sec:newave}

Desenvolvido e mantido pelo Centro de Pesquisas de Energia Elétrica (CEPEL) e amplamente utilizado pelo setor elétrico 
brasileiro para definição de estratégias e tomada de decisão, o NEWAVE é um modelo de otimização que busca minimizar os 
custos de operação do sistema, considerando a incerteza das afluências futuras e a operação de um sistema 
hidro-térmico-eólico interligado. O modelo é utilizado para estudos como:
\begin{itemize}
	\item Elaboração do Plano Decenal de Expansão de Energia (PDE), pela EPE;
	\item Elaboração do Programa Mensal de Operação (PMO) e Plano de Operação Energética (PEN), pelo ONS;
	\item Formação de preços, como no cálculo do Preço de Liquidação das Diferenças (PLD) pelo CCEE;
	\item Cálculo de Garantia Física e da Energia Assegurada para empreendimentos de geração participantes nos leilões 
    de energia elétrica, pela EPE;
	\item Elaboração de diretrizes para os leilões de energia, pela EPE.
\end{itemize}

Em resumo, o modelo emprega a Programação Dinâmica Dual Estocástica (PDDE), uma técnica de otimização que permite lidar 
com as incertezas ligadas às afluências futuras sem que o modelo se torne computacionalmente impraticável, considerando 
múltiplos reservatórios, interconexões e o horizonte temporal de médio e longo prazos.

\subsection{Representação das Usinas}
O NEWAVE modela o sistema de geração hidrelétrico em Reservatórios Equivalentes de Energia (REEs), que são grupos de
usinas associadas a um subsistema ou submercado de energia. Cada subsistema pode conter mais de um REE, possibilitando
diferenciar bacias hidrográficas com regimes distintos, ainda que pertençam a um mesmo subsistema. 

Além disso, cada REE é definido por um conjunto de parâmetros que são calculados a partir das características indivuduais 
de cada usina. Nas versões mais recentes do modelo, também é possível considerar todas as usinas indivudalmente ou operar
de maneira híbrtida, ou seja, considerando alguns REEs e outras usinas individualmente.

As usinas termelétricas são representadas no modelo através de classes térmicas. Cada classe agrupa usinas com custos 
semelhantes e está associada a um subsistema. Cada classe também é definida por um conjunto de parâmetros calculados
a partir das características individuais de cada usina.

Nas versões mais recentes do modelo, a fonte eólica também é modelada. De maneira similar, os parques eólicos são agrupados
em Parques Eólicos Equivalentes (PEE). O agrupamento é feito a aprtir de dados de cadastro de cada prque eólico, estado,
submercado, função de produção (curva relacionando a velocidade do vento com a potência gerada), dados sobre torres de
medição e séries históricas de velocidade do vento.

\subsection{Dados de Entrada}
O modelo requer um conjunto de dados de entrada que inclui as características das usinas, dados dos subsistemas, demanda,
séries históricas de vazões e ventos, cronogramas de expansão, restrições operativas, dentre outros. Observa-se, portanto, 
que todos os dados de entrada são locais e, portanto, o modelo não considera variáveis externas, como fenomênos climáticos
como o EN e LN, que podem impactar a geração de energia elétrica. 

Ainda que as últimas versões do modelo apresentem campos previstos para a entrada de dados do ENSO, esses campos
estão marcados como "não implementados". Dessa forma, entende-se que o modelo não considera diretamente o impacto 
dessas variáveis. No entanto, vale destacar que essas variáveis externas podem ser utilizadas para elaborar as séries históricas de 
vazões e velocidade de ventos utilizadas como dados de entrada. 

\section{Modelos Linear e Não-linear}
fazer uma introdução sobre os modelos lineares e não lineares...

\subsection{Modelo Linear}

\subsection{Modelos Não-lineares: Random Forest e Gradient Boosting}

\section{Modelo Neural}

