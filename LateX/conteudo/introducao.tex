% Introdução e Trabalhos Relacionados (capítulos 1 e 2)
% ----------------------------------------------------------------------------------------------------------------------
\chapter{Introdução}
\sloppy																													% Para justificar o texto

\section{Contexto}
Historicamente, a matriz elétrica brasileira é considerada uma das mais limpas do mundo, com destaque para a fonte
hidráulica, que é responsável pela maior parte da geração de energia elétrica no país. Nos últimos anos, outras fontes
de geração vêm sendo incorporadas ao sistema, das quais destacam-se a eólica e solar fotovoltaica, conforme observado na
Figura \ref{fig:geracao_anual_por_fonte}, elaborada a partir de dados brutos de geração centralizada obtidos do Operador
Nacional do Sistema Elétrico (ONS), sem considerar a geração distribuída.

\begin{figure}[!ht]
	\IBGEtab{\caption{Geração centralizada anual por fonte}
			 \label{fig:geracao_anual_por_fonte}}
	{\includesvg[scale=1]{figuras/geracao_anual_percentual}}
	{\fonte{o autor.}}
\end{figure}

Nota-se, em especial, um crescimento significativo da geração eólica, observado a partir de 2015, e uma diminuição 
significativa da contribuição de geração térmica média no panorama geral nos anos seguintes. Em 2023, a fonte eólica 
foi responsável por 48\% da expansão da capacidade instalada total de 10,19 GW \cite{EPE2024}. Essa expansão se dá em 
função do maior número de empreendimentos participantes nos Leilões de Energia Elétrica do Ambiente de Contratação 
Regulada (ACR) realizados pela Empresa de Pesquisa Energética (EPE). Isso ocorre, dentre outros fatores, devido à queda 
nos custos de aerogeradores e painéis fotovoltaicos, além do fator "combustível zero" dessas fontes, o que torna novos 
empreendimentos mais atrativos economicamente para os agentes.

Embora essa expansão seja positiva, poupando recursos hídricos, contribuindo para a diversificação da matriz elétrica e
reduzindo o acionamento de usinas térmicas, essas fontes possuem características intrínsecas que as tornam intermitentes,
como a incidência solar e a velocidade do vento. Sendo assim, uma alta dependência dessas fontes tem o potencial
de tornar o sistema como um todo mais vulnerável.

Além disso, ao analisar a curva de carga do SIN, observa-se que, embora o seu pico ocorra no início da 
tarde, momento no qual a geração solar fotovoltaica apresenta significativa contribuição, o período noturno também 
apresenta carga considerável, conforme a Figura \ref{fig:carga_max_dia}, que mostra a curva de carga do SIN para o 
dia 15 de março de 2024, dia em que registrou-se um recorde de demanda máxima instantânea de 102.478 MW, segundo o ONS, 
e como pode ser observado na Figura \ref{fig:carga_anual}.

\begin{figure}[!ht]
	\IBGEtab{\caption{Curva de carga diária do SIN em base horária}
			 \label{fig:carga_max_dia}}
	{\includesvg[scale=1]{figuras/carga_max_dia_2024-03-15}}
	{\fonte{o autor.}}
\end{figure}

\section{Motivação}
Em um contexto no qual a implementação de sistemas de armazenamento de energia elétrica ainda é incipiente,
a matriz segue bastante dependente da fonte hidráulica e, de maneira complementar, das térmicas. A dependência da fonte
hidráulica, por sua vez, torna o sistema elétrico vulnerável a eventos climáticos extremos ocasionados pelas mudanças
climáticas. Por exemplo, em 2021, verificou-se um acionamento recorde de usinas térmicas e uma geração hidráulica 
percentual mínima. Isso se deve em razão da forte crise hídrica enfrentada pelo Brasil no período, a pior dos últimos 91 anos
até então. \cite{Soares2023}

\begin{figure}[!ht]
	\IBGEtab{\caption{Curva de carga do SIN em base mensal}
			 \label{fig:carga_anual}}
	{\includesvg[scale=1]{figuras/carga_anual}}
	{\fonte{o autor.}}
\end{figure}

Portanto, o estudo do sistema elétrico brasileiro, no contexto de cenários de eventos
climatológicos extremos é altamente relevante para a segurança energética do país, considerando uma estimativa de 
crescimento médio anual da carga do SIN de 3,2\%. \cite{pen2024}

Ao analisar a geração hidráulica bruta na Figura \ref{fig:geracao_hidraulica_bruta}, evidenciam-se pontos nos 
quais a geração é reduzida. Isso ocorre devido à sazonalidade das vazões nas bacias hidrográficas, responsáveis pelo 
abastecimento dos reservatórios. Considerando a amostragem em base mensal, observa-se que a geração é reduzida nos meses
de inverno, período caracterizado por menor ocorrência de precipitação e, consequentemente, menor vazão nos rios. Por
outro lado, nos meses de verão, a geração atinge seus maiores valores.

\begin{figure}[!ht]
	\IBGEtab{\caption{Geração hidráulica total em base mensal}
			 \label{fig:geracao_hidraulica_bruta}}
	{\includesvg[scale=1]{figuras/geracao_hidraulica_bruta}}
	{\fonte{o autor.}}
\end{figure}

Esse comportamento é esperado, uma vez que a geração hidráulica é diretamente influenciada pelas condições
que afetam a vazão dos rios. No entanto, a ocorrência de eventos climáticos como o El Niño-Oscilação Sul (ENSO) pode
favorecer condições que impactam diretamente no potencial de geração hidráulica. \cite{de2012influencia}

\begin{figure}[!ht]
	\IBGEtab{\caption{Índice ONI (Oceanic Niño Index)}
			 \label{fig:oni}}
	{\includesvg[scale=1]{figuras/oni}}
	{\fonte{o autor.}}
\end{figure}

Fenômenos como o ENSO são monitoradas por meio de índices como o ONI (Oceanic Niño Index), que classifica os 
eventos em três categorias: \textit{El-Niño} (EN), \textit{La-Niña} (LN) e neutro. A Figura \ref{fig:oni} mostra a 
classificação dos eventos de EN e LN ocorridos entre 2000 e 2024, na qual a escala de cores indica a intensidade do 
evento, representado pela anomalia, que é a diferença do valor observado para um período em relação à sua média histórica.


Fundamentalmente, no sistema elétrico brasileiro, cuja fonte hidráulica constitui a base da matriz, é essencial, para um
planejamento energético eficiente, otimizar o sistema de modo a considerar a incerteza associada à disponibilidade de recursos
hidrológicos futuros. Dessa forma, estima-se o valor da geração hidrelétrica que poderia substituir a geração térmica a curto
ou longo prazo, de modo a reduzir os custos do sistema e o risco de utilizar reservatórios de maneira desnecessária, garantindo
assim o atendimento à demanda futura, principalmente em casos de escassez hídrica.

Com a introdução de fontes como a eólica, a incerteza referente aos perfis de velocidade do vento também deve ser considerada
para o planejamento energético. Os estudos de planejamento são realizados por meio de modelos computacionais como o NEWAVE, 
DECOMP e DESSEM, do Centro de Pesquisas de Energia Elétrica (CEPEL), que consideram diferentes horizontes temporais: longo, 
médio e curto prazos, respectivamente. Também há outras soluções disponíveis no mercado, como o PSR SDDP, que engloba todos
os horizontes temporais. Considerando o escopo deste trabalho, o modelo NEWAVE será brevemente apresentado no capítulo seguinte.

Embora esses modelos sejam amplamente utilizados pelo setor elétrico brasileiro, sendo consolidados como ferramentas
confiáveis e essenciais para o planejamento energético, eles não consideram variáveis externas, como fenômenos climáticos
como o EN e LN, que podem impactar a geração de energia elétrica. Esses modelos são baseados em dados históricos de
vazões e velocidade do vento, que são obtidos a partir de medições locais, além de dados individuais de cada usina.

\section{Objetivo}
O projeto tem como objetivo investigar um possível impacto de variáveis climáticas externas na geração 
de energia elétrica no Brasil, com foco nas fontes hidráulica, térmica e eólica. Para tanto, foram empregadas técnicas 
computacionais para relacionar as séries históricas de geração com as séries de variáveis 
climáticas, através de modelos lineares, não lineares e neurais. 

Além disso, foram utilizadas séries históricas de geração, 
carga e vazões disponibilizados pelo ONS, bem como séries históricas de variáveis climáticas, como temperatura 
da superfície do mar, obtidas a partir de dados do ERA5, um projeto de reanálise atmosférica que combina resultados de
modelos computacionais com observações de diferentes fontes, como satélites e estações meteorológicas, resultando em
um \textit{dataset} global de alta resolução espacial e temporal. \citeonline{Hersbach2020}

A partir dessa investigação, espera-se poder avaliar o impacto dessas variáveis na geração de energia elétrica,
o que pode contribuir para o planejamento energético do país, especialmente em cenários de eventos climáticos extremos e
tendências climáticas.

É importante salientar que outras variáveis externas poderiam ser incorporadas ao estudo, ou até mesmo uma combinação entre
 variáveis locais e externas. Também poderiam ser considerados indicadores econômicos e outros dados relevantes. 
No entanto, este trabalho considera apenas as variáveis relacionadas ao fenômeno EN e LN, uma vez que estudos indicam
uma alta correlação entre esses fenômenos e o regime de chuvas no Brasil \cite{de2012influencia, Andreoli2016}, sendo a 
incorporação de outras variáveis reservada para trabalhos futuros.

% ----------------------------------------------------------------------------------------------------------------------
\section{Estrutura do Trabalho}
No capítulo 1, é feita uma breve introdução apresentando o contexto, motivação, objetivo e a estrutura do trabalho. Uma
breve análise da matriz elétrica é apresentada, com foco no histórico recente e no crescimento da geração eólica. Também
são apresentadas as curvas de carga do SIN e de geração hidráulica. Uma breve contextualização acerca dos fenômenos do
ENSO e a importância do modelo NEWAVE para o planejamento energético do país são apresentadas. Por fim, é apresentado
o objetivo do trabalho e a estrutura do documento. O capítulo 2 apresenta um panorama acerca de trabalhos relacionados ao tema.

No capítulo 3, a fundamentação teórica necessária para a compreensão do projeto é apresentada. São abordados os fatores
que fazem com que os fenômenos do EN e LN impactem o regime de chuvas no Brasil e a sua relação com a geração de 
energia elétrica. Também será feita uma breve introdução ao modelo NEWAVE,
que é amplamente utilizado pelo setor elétrico brasileiro para planejamento energético. Por fim, serão apresentados os modelos 
de previsão de séries temporais implementados, com foco no modelo neural, uma implementação baseada na arquitetura 
\textit{TSMixer} desenvolvida pela Google.

O capítulo 4 mostra a metodologia utilizada para a realização do projeto. Serão apresentados os conjuntos de dados
considerados e suas respectivas etapas de obtenção, tratamento e análise. Além disso, será apresentada a metodologia
utilizada para implementação dos modelos de previsão, incluindo seus parâmetros e métricas de avaliação. \textit{Snippets}
de códigos serão apresentados para facilitar a compreensão.

O capítulo 5 apresenta os resultados do projeto. Para cada modelo, serão apresentados os resultados de previsão, métricas 
de avaliação e uma análise crítica dos resultados obtidos. No capítulo 6, serão apresentadas as considerações finais.

% \chapter{Trabalhos Relacionados}
% Avaliação do Impacto de Secas Severas no Nordeste Brasileiro na Geração de Energia Elétrica Através do Modelo Newave: 
% Projeção das Energias Afluentes e Armazenadas \cite{Vilar2020}

% {https://files.abrhidro.org.br/Eventos/Trabalhos/60/PAP023274.pdf}

% {https://pantheon.ufrj.br/bitstream/11422/21709/1/887353.pdf}