% Introdução e Trabalhos Relacionados (capítulos 1 e 2)
% ----------------------------------------------------------------------------------------------------------------------
\chapter{Introdução}
\sloppy																													% Para justificar o texto

\section{Contexto}
Historicamente, a matriz elétrica brasileira é considerada uma das mais limpas do mundo, com destaque para a fonte
hidráulica, que é responsável pela maior parte da geração de energia elétrica no país. Nos últimos anos, outras fontes
de geração vêm sendo incorporadas ao sistema, das quais destacam-se a eólica e solar fotovoltaica, conforme observado na
Figura \ref{fig:geracao_anual_por_fonte}, elaborada a partir de dados brutos de geração centralizada obtidos do Operador
Nacional do Sistema Elétrico (ONS), sem considerar a geração distribuída.

\begin{figure}[!ht]
	\IBGEtab{\caption{Geração centralizada anual por fonte}
			 \label{fig:geracao_anual_por_fonte}}
	{\includesvg[scale=1]{figuras/geracao_anual_por_fonte}}
	{\fonte{o autor.}}
\end{figure}

Nota-se, em especial, um crescimento significativo da geração eólica, observado a partir de 2015, e uma diminuição 
significativa da contribuição de geração térmica média no panorama geral nos anos seguintes. Em 2023, a fonte eólica 
foi responsável por 48\% da expansão da capacidade instalada total de 10,19 GW \cite{EPE2024}. Essa expansão se dá em 
função do maior número de empreendimentos participantes nos Leilões de Energia Elétrica do Ambiente de Contratação 
Regulada (ACR) realizados pela Empresa de Pesquisa Energética (EPE). Isso ocorre, dentre outros fatores, devido à queda 
nos custos de aerogeradores e painéis fotovoltaicos, além do fator "combustível zero" dessas fontes, o que torna novos 
empreendimentos mais atrativos economicamente para os agentes.

Embora essa expansão seja positiva, poupando recursos hídricos, contribuindo para a diversificação da matriz elétrica e
reduzindo o acionamento de usinas térmicas, essas fontes possuem características intrínsecas que as tornam intermitentes,
como a incidência solar e a velocidade do vento. Essas variáveis possuem sazonalidades de curto e longo prazo, a 
depender da hora do dia e estação do ano, por exemplo. Sendo assim, uma alta dependência dessas fontes tem o potencial
de tornar o sistema como um todo mais vulnerável.

Além disso, ao analisar a curva de carga do SIN, observa-se que, embora o seu pico ocorra no início da 
tarde, momento no qual a geração solar fotovoltaica apresenta significativa contribuição, o período noturno também 
apresenta carga considerável, conforme a Figura \ref{fig:carga_max_dia}, que mostra a curva de carga do SIN para o 
dia 15 de março de 2024, dia em que registrou-se um recorde de demanda máxima instantânea de 102.478 MW, segundo o ONS, 
e como pode ser observado na Figura \ref{fig:carga_anual}.

\begin{figure}[!ht]
	\IBGEtab{\caption{Curva de carga diária do SIN em base horária}
			 \label{fig:carga_max_dia}}
	{\includesvg[scale=1]{figuras/carga_max_dia_2024-03-15}}
	{\fonte{o autor.}}
\end{figure}

\section{Motivação}
Fundamentalmente, em sistemas interligados cuja fonte hidráulica constitui a base da matriz elétrica, é essencial, 
para um planejamento energético eficiente, otimizar o sistema de modo a considerar a operação de todas as usinas, 
considerando a incerteza associada às afluências futuras. Dessa forma, estima-se o valor da geração hidrelétrica que 
poderia substituir a geração térmica a curto ou longo prazo, de modo a minimizar os custos de operação do sistema e o 
risco de utilizar reservatórios de maneira desnecessária, garantindo assim o atendimento à demanda futura, 
principalmente em casos de escassez hídrica.

\begin{figure}[!ht]
	\IBGEtab{\caption{Curva de carga do SIN em base mensal}
			 \label{fig:carga_anual}}
	{\includesvg[scale=1]{figuras/carga_anual}}
	{\fonte{o autor.}}
\end{figure}

Em um contexto no qual a implementação de sistemas de armazenamento de energia elétrica ainda é incipiente,
a matriz segue bastante dependente da fonte hidráulica e, de maneira complementar, das térmicas. A dependência da fonte
hidráulica, por sua vez, torna o sistema elétrico vulnerável a eventos climáticos extremos ocasionados pelas mudanças
climáticas. Por exemplo, em 2021, verificou-se um acionamento recorde de usinas térmicas e uma geração hidráulica 
percentual mínima. Isso se deve em razão da forte crise hídrica enfrentada pelo Brasil em 2021, a pior dos últimos 91 
anos. \cite{Soares2023}

Portanto, o estudo da operação do sistema elétrico brasileiro, no contexto de cenários de eventos
climatológicos extremos é altamente relevante para a segurança energética do país, considerando uma estimativa de 
crescimento médio anual da carga do SIN de 3,2\%. \cite{pen2024}

Ao analisar a geração hidráulica bruta na Figura \ref{fig:geracao_hidraulica_bruta}, evidenciam-se pontos nos 
quais a geração é reduzida. Isso ocorre devido à sazonalidade das vazões nas bacias hidrográficas, responsáveis pelo 
abastecimento dos reservatórios. Considerando a amostragem em base mensal, observa-se que a geração é reduzida nos meses
de inverno, período caracterizado por menor ocorrência de precipitação e, consequentemente, menor vazão nos rios. Por
outro lado, nos meses de verão, a geração atinge seus maiores valores.

\begin{figure}[!ht]
	\IBGEtab{\caption{Geração hidráulica total em base mensal}
			 \label{fig:geracao_hidraulica_bruta}}
	{\includesvg[scale=1]{figuras/geracao_hidraulica_bruta}}
	{\fonte{o autor.}}
\end{figure}

Esse comportamento é natural e esperado, uma vez que a geração hidráulica é diretamente influenciada pelas condições
que afetam a vazão dos rios. No entanto, a ocorrência de eventos climatológicos como o \textit{El-Niño} (EN) e 
\textit{La-Niña} (LN) pode favorecer condições que impactam diretamente no potencial de geração hidráulica. \cite{de2012influencia}

Fenômenos como o EN e LN são caracterizados por anomalias na temperatura da superfície do mar no Oceano Pacífico
Equatorial. Essas anomalias são monitoradas por meio de índices como o ONI (Oceanic Niño Index), que é calculado
pela NOAA e classifica os eventos em três categorias: EN, LN e neutro. A Figura \ref{fig:oni} mostra a classificação
dos eventos de EN e LN ocorridos entre 2000 e 2024, sendo que a escala de cores indica a intensidade do evento. Ao
analisar a geração hidráulica bruta no mesmo período, observa-se que a ocorrência de eventos de EN e LN está associada
a variações na geração.

\begin{figure}[!ht]
	\IBGEtab{\caption{Índice ONI (Oceanic Niño Index)}
			 \label{fig:oni}}
	{\includesvg[scale=1]{figuras/oni}}
	{\fonte{o autor.}}
\end{figure}

\section{Objetivo}
O projeto tem como objetivo analisar o impacto de variáveis climatológicas, como anomalias e outros índices na geração 
de energia elétrica no Brasil, com foco nas fontes hidráulica e térmica, adotando técnicas computacionais para realizar 
a análise de correlações entre variáveis climáticas e a geração de energia. Para tanto, serão utilizados séries 
históricas de geração, carga e vazões disponibilizados pelo ONS, bem como séries históricas de variáveis climatológicas,
 como temperatura da superfície do mar, pressão e precipitação, consolidadas por instituições de pesquisa como o
 \textit{National Oceanic and Atmospheric Administration} (NOAA) e 
 \textit{European Centre for Medium-Range WeatherForecasts} (ECMWF).

A partir dessas análises, espera-se identificar padrões de comportamento da geração em cenários de eventos 
climatológicos extremos, bem como propor medidas de mitigação de riscos e adaptação do sistema elétrico brasileiro a 
esses eventos. Além disso, o projeto visa estabelecer uma base de conhecimento para pesquisas futuras na área de
planejamento energético e clima, contribuindo para a segurança energética do país.

\section{Estrutura do Trabalho}

O presente trabalho está estruturado em cinco capítulos, sendo:
\begin{enumerate}
	\item Introdução, contendo o contexto, motivação, objetivo e estrutura do trabalho;
	\item Trabalhos Relacionados, que contém uma revisão bibliográfica sobre o tema;
	\item Metodologia, descrevendo a abordagem computacional adotada para a análise de dados;
	\item Resultados, que apresenta os resultados obtidos a partir da análise de dados;
	\item Conclusão e Trabalhos Futuros.
\end{enumerate}

\chapter{Trabalhos Relacionados}
Avaliação do Impacto de Secas Severas no Nordeste Brasileiro na Geração de Energia Elétrica Através do Modelo Newave: 
Projeção das Energias Afluentes e Armazenadas \cite{Vilar2020}

{https://files.abrhidro.org.br/Eventos/Trabalhos/60/PAP023274.pdf}

{https://pantheon.ufrj.br/bitstream/11422/21709/1/887353.pdf}