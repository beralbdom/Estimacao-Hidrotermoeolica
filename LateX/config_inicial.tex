% abnTeX2: Modelo de Trabalho Academico  em conformidade com ABNT NBR 14724:2011
% http://www.abntex.net.br/

% Customização Dept. Engenharia Elétrica UFF
% Autor: Bernardo Albuquerque (Adaptado da Prof. Malú Grave e Jonas Alessi)
% Versão: 15 de Janeiro de 2025.

% Edição: VSCode (LaTeX Workshop + LaTeX Utilities)
% Programas necessários: MiKTeX, Strawberry Perl, Inkscape
% Codificação: UTF-8
% LaTeX: abnTeX2
% ----------------------------------------------------------------------------------------------------------------------
\usepackage{cmap}                																						% Mapear caracteres especiais no PDF
\usepackage[scaled = .85]{zi4}
\usepackage[scaled = .82]{DejaVuSerif}
\usepackage[T1]{fontenc}         																						% Selecao de codigos de fonte
\usepackage[utf8]{inputenc}      																						% Codificacao do documento (conversão automática dos acentos)
\usepackage{lastpage}            																						% Usado pela Ficha catalográfica
\usepackage{indentfirst}         																						% Indenta o primeiro parágrafo de cada seção
\usepackage{color}               																						% Controle das cores
\usepackage{graphicx}            																						% Inclusão de gráficos
\usepackage{tocloft}
\usepackage[newfloat]{minted}
\usepackage{amsfonts}            																						% Símbolos
\usepackage[none]{hyphenat}      																						% Sem separação de sílabas
\usepackage{pdfpages}            																						% Acrescentar pdf ficha bibliográfica
\usepackage{caption}
\usepackage{titlesec}            																						% Modificar títulos
\usepackage{svg}                 																						% Pacote para incluir imagens svg
\usepackage{lipsum}              																						% para geração de dummy text
\usepackage[brazilian, hyperpageref]{backref}  																			% Paginas com as citações na bibliografia
\usepackage{enumitem}
	\setitemize[0]{itemindent=0.4cm, itemsep=0pt}
	\setenumerate[0]{itemindent=0.5cm, itemsep=0pt}
\usepackage[alf,
abnt-emphasize=bf,		 		 																						      % destaca o título da obra em negrito
abnt-url-package=none,			 																						    % Utiliza o pacote url
abnt-repeated-title-omit=yes,	 																						  % Retira a string "Repetição de título" nas referências
abnt-full-initials=yes,          																						% yes nome por extenso, no apenas iniciais
abnt-etal-list=3                 																						% abreviar com mais de 3 autores
]{abntex2/abntex2cite}           																						% Citações padrão ABNT

% \addto\captionsbrazilian{%
%   \renewcommand{\algorithmname}{Algoritmo}%
% }

% \usepackage{draftwatermark}
% \SetWatermarkText{PRÉVIA}
% \SetWatermarkScale{1}
% \SetWatermarkColor[gray]{0.9}

% Configurações de aparência do PDF final
% ----------------------------------------------------------------------------------------------------------------------
\DeclareFloatingEnvironment[
    listname={\protect\parbox[t]{\linewidth}{\raggedright Lista de Códigos}},
    name=Código,
    placement=tbhp,
    within=chapter,
]{codigo}

\captionsetup[codigo]{
  name=Código,
  position=above,              % coloca a legenda acima
  justification=raggedright, % alinha à esquerda
  singlelinecheck=false
}

\definecolor{blue}{RGB}{41,5,195}    																					% alterando o aspecto da cor azul
\titleformat{\chapter}[display]																							% Formatação do título do capítulo
    {\normalfont\huge\bfseries}
    {Capítulo \thechapter}{20pt}
    {}
    {}
\titlespacing*{\chapter}{0pt}{-20pt}{40pt}																				% Espaçamento entre o título do capítulo e o texto

\titleformat{name=\chapter,numberless}[display]
  {\normalfont\huge\bfseries\raggedright} % Use \raggedright for left alignment
  {}                                      % No label for unnumbered chapters
  {0pt}                                   % Separation (not applicable here)
  {}                                      % Code before title
\titlespacing*{name=\chapter,numberless}{0pt}{-20pt}{40pt} % Adjust spacing as for numbered chapters

\setlength{\parindent}{1.25cm}
\linespread{1.5}																										% Espaçamento entre linhas

\usemintedstyle{custommanni}
\setminted{
  frame=lines,
  framesep=1mm,
  baselinestretch=1,
  linenos,
  breaklines,
  style=colorful,
  tabsize=4,
  numbersep=5pt
}

\setlength{\afterchapskip}{\baselineskip}																				% Espaçamento entre o título do capítulo e o texto
\setlength{\parskip}{0cm}																								% Espaçamento entre parágrafos

\hangcaption																											% Alinhamento de legendas
\captionstyle[\raggedright]{}																							% Alinhamento de legendas																						% Alinhamento de legendas

\renewcommand{\backref}{}																								% Não gera texto de referência nas citações
\renewcommand*{\backrefalt}[4]{}																						% Define os textos da citação
\renewcommand{\listtablename}{Lista de Tabelas}																			% Nome da lista de tabelas

\captionnamefont{\ABNTEXfontereduzida}																					% Fonte das legendas
\captiontitlefont{\ABNTEXfontereduzida}																					% Fonte das legendas
\setlength\cftbeforechapterskip{0pt}																					% Espaçamento entre capítulos

% Compila o índice
% ----------------------------------------------------------------------------------------------------------------------
\makeindex