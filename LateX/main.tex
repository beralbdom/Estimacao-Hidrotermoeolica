% abnTeX2: Modelo de Trabalho Academico  em conformidade com ABNT NBR 14724:2011
% http://www.abntex.net.br/

% Customização Dept. Engenharia Elétrica UFF
% Autor: Bernardo Albuquerque (Adaptado da Prof. Malú Grave e Jonas Alessi)
% Versão: 15 de Janeiro de 2025.

% Edição: VSCode (LaTeX Workshop + LaTeX Utilities)
% Programas necessários: MiKTeX, Strawberry Perl, Inkscape
% Codificação: UTF-8
% LaTeX: abnTeX2
% ----------------------------------------------------------------------------------------------------------------------

\documentclass[
  12pt,				                                                                                                          % Tamanho da fonte	
  oneside,                                                                                                              % Não imprimir em verso e anverso, oposto do twoside 
  a4paper,			                                                                                                        % Tamanho do papel. 
  chapter = Title,		                                                                                                  % Títulos de capítulos convertidos em letras maiúsculas
  section = Title,		                                                                                                  % Títulos de seções convertidos em letras maiúsculas
  subsection = Title,	                                                                                                  % Títulos de subseções convertidos em letras maiúsculas
  english,			                                                                                                        % Idioma adicional para hifenização
  brazil,			                                                                                                          % O último idioma é o principal do documento
  sumario = tradicional
  ]{abntex2-uff}

% abnTeX2: Modelo de Trabalho Academico  em conformidade com ABNT NBR 14724:2011
% http://www.abntex.net.br/

% Customização Dept. Engenharia Elétrica UFF
% Autor: Bernardo Albuquerque (Adaptado da Prof. Malú Grave e Jonas Alessi)
% Versão: 15 de Janeiro de 2025.

% Edição: VSCode (LaTeX Workshop + LaTeX Utilities)
% Programas necessários: MiKTeX, Strawberry Perl, Inkscape
% Codificação: UTF-8
% LaTeX: abnTeX2
% ----------------------------------------------------------------------------------------------------------------------
\usepackage{cmap}                																						% Mapear caracteres especiais no PDF
\usepackage[scaled = .85]{zi4}
\usepackage[scaled = .82]{DejaVuSerif}
\usepackage[T1]{fontenc}         																						% Selecao de codigos de fonte
\usepackage[utf8]{inputenc}      																						% Codificacao do documento (conversão automática dos acentos)
\usepackage{lastpage}            																						% Usado pela Ficha catalográfica
\usepackage{indentfirst}         																						% Indenta o primeiro parágrafo de cada seção
\usepackage{color}               																						% Controle das cores
\usepackage{graphicx}            																						% Inclusão de gráficos
\usepackage{tocloft}
\usepackage[newfloat]{minted}
\usepackage{amsfonts}            																						% Símbolos
\usepackage[none]{hyphenat}      																						% Sem separação de sílabas
\usepackage{pdfpages}            																						% Acrescentar pdf ficha bibliográfica
\usepackage{caption}
\usepackage{titlesec}            																						% Modificar títulos
\usepackage{svg}                 																						% Pacote para incluir imagens svg
\usepackage{lipsum}              																						% para geração de dummy text
\usepackage[brazilian, hyperpageref]{backref}  																			% Paginas com as citações na bibliografia
\usepackage{enumitem}
	\setitemize[0]{itemindent=0.4cm, itemsep=0pt}
	\setenumerate[0]{itemindent=0.5cm, itemsep=0pt}
\usepackage[alf,
abnt-emphasize=bf,		 		 																						      % destaca o título da obra em negrito
abnt-url-package=none,			 																						    % Utiliza o pacote url
abnt-repeated-title-omit=yes,	 																						  % Retira a string "Repetição de título" nas referências
abnt-full-initials=yes,          																						% yes nome por extenso, no apenas iniciais
abnt-etal-list=3                 																						% abreviar com mais de 3 autores
]{abntex2/abntex2cite}           																						% Citações padrão ABNT

% \addto\captionsbrazilian{%
%   \renewcommand{\algorithmname}{Algoritmo}%
% }

% \usepackage{draftwatermark}
% \SetWatermarkText{PRÉVIA}
% \SetWatermarkScale{1}
% \SetWatermarkColor[gray]{0.9}

% Configurações de aparência do PDF final
% ----------------------------------------------------------------------------------------------------------------------
\DeclareFloatingEnvironment[
    listname={\protect\parbox[t]{\linewidth}{\raggedright Lista de Códigos}},
    name=Código,
    placement=tbhp,
    within=chapter,
]{codigo}

\captionsetup[codigo]{
  name=Código,
  position=above,              % coloca a legenda acima
  justification=raggedright, % alinha à esquerda
  singlelinecheck=false
}

\definecolor{blue}{RGB}{41,5,195}    																					% alterando o aspecto da cor azul
\titleformat{\chapter}[display]																							% Formatação do título do capítulo
    {\normalfont\huge\bfseries}
    {Capítulo \thechapter}{20pt}
    {}
    {}
\titlespacing*{\chapter}{0pt}{-20pt}{40pt}																				% Espaçamento entre o título do capítulo e o texto

\titleformat{name=\chapter,numberless}[display]
  {\normalfont\huge\bfseries\raggedright} % Use \raggedright for left alignment
  {}                                      % No label for unnumbered chapters
  {0pt}                                   % Separation (not applicable here)
  {}                                      % Code before title
\titlespacing*{name=\chapter,numberless}{0pt}{-20pt}{40pt} % Adjust spacing as for numbered chapters

\setlength{\parindent}{1.25cm}
\linespread{1.5}																										% Espaçamento entre linhas

\usemintedstyle{custommanni}
\setminted{
  frame=lines,
  framesep=1mm,
  baselinestretch=1,
  linenos,
  breaklines,
  style=colorful,
  tabsize=4,
  numbersep=5pt
}

\setlength{\afterchapskip}{\baselineskip}																				% Espaçamento entre o título do capítulo e o texto
\setlength{\parskip}{0cm}																								% Espaçamento entre parágrafos

\hangcaption																											% Alinhamento de legendas
\captionstyle[\raggedright]{}																							% Alinhamento de legendas																						% Alinhamento de legendas

\renewcommand{\backref}{}																								% Não gera texto de referência nas citações
\renewcommand*{\backrefalt}[4]{}																						% Define os textos da citação
\renewcommand{\listtablename}{Lista de Tabelas}																			% Nome da lista de tabelas

\captionnamefont{\ABNTEXfontereduzida}																					% Fonte das legendas
\captiontitlefont{\ABNTEXfontereduzida}																					% Fonte das legendas
\setlength\cftbeforechapterskip{0pt}																					% Espaçamento entre capítulos

% Compila o índice
% ----------------------------------------------------------------------------------------------------------------------
\makeindex                                                                                                % Pacotes fundamentais e configurações do documento
% Informações de dados para CAPA e FOLHA DE ROSTO
\titulo{Estimação da Geração Hidrotermoeólica Utilizando Redes Neurais e Variáveis do Fenômeno El Niño}
\autor{Bernardo Albuquerque Domingues da Silva}
\local{Niterói}
\data{2025}
\orientador{Prof. André da Costa Pinho, D.Sc.}
%\coorientador{Prof. Nome Completo do Orientador e Titulação}
%\supervisor{Supervisor}  % no caso de ser estágio

\instituicao{\textsc{Universidade Federal Fluminense \\
                            Escola de Engenharia \\
                            Curso de Graduação em Engenharia Elétrica}}
% \tipotrabalho{TRABALHO DE CONCLUSÃO DE CURSO}

% O preambulo deve conter o tipo do trabalho, o objetivo, 
% o nome da instituição e a área de concentração 
\preambulo{Projeto de Conclusão de Curso apresentado ao Corpo Docente do Departamento de Engenharia Elétrica da Escola 
de Engenharia da Universidade Federal Fluminense, como parte dos requisitos necessários à obtenção do título de 
Engenheiro Eletricista.}                                                                                              % Capa do documento deve ser antes do início do documento

\begin{document}
\frenchspacing                                                                                                          % Retira espaço extra obsoleto entre as frases.

% Elementos pré-textuais
% ----------------------------------------------------------------------------------------------------------------------
% Capa
{\Large \imprimircapa}

% Folha de rosto
\imprimirfolhaderosto

% Ficha catalográfica
\includepdf[pages=-]{pretextuais/ficha.pdf}                                                                             % https://bibliotecas.uff.br/bcg/fichacatalografica/

% Folha de aprovação
\begin{folhadeaprovacao}

  \begin{center}
  \vspace*{-1.2cm}
    {\imprimirautor}

    \vspace*{\fill}
    {\textbf{\imprimirtitulo}}
    \vspace*{\fill}
    
    \hspace{.45\textwidth}
    \begin{minipage}{.5\textwidth}
        \imprimirpreambulo
    \end{minipage}%
    \vspace*{\fill}
   \end{center}
    
  \begin{flushleft}
  	 Aprovado em: 11 de julho de 2025 com nota 9,5.
  \end{flushleft}
  \begin{center}
  \vspace*{2 cm}
  \textbf{BANCA EXAMINADORA}
   \assinatura{Prof. André da Costa Pinho, D.Sc. - UFF
   }
   \assinatura{Prof. Bruno Soares Moreira Cesar Borba, D.Sc. - UFF
   }
   \assinatura{Prof. Marcio Andre Ribeiro Guimaraens, D.Sc. - UFF
   }



\vspace*{\fill}

{\imprimirlocal}
\par
{\imprimirdata}
 \vspace*{-0.7cm}
 
 \end{center}
\end{folhadeaprovacao}

% Dedicatória
\include{pretextuais/dedicatoria}

% Agradecimentos
\begin{agradecimentos}
\sloppy	

Aos meus pais, Débora e Gilvan, por sempre me apoiarem nas minhas escolhas, terem me ajudado a superar os obstáculos 
que surgiram no caminho e sempre estarem presentes nos momentos mais importantes. Obrigado por acreditarem em mim, por 
terem abdicado de tantas coisas para me proporcionar uma educação de qualidade e me ensinado a importância da ética, esforço, 
estudo e trabalho. Serei eternamente grato pelos esforços e sacrifícios que fizeram por mim. Vocês sempre serão os meus 
maiores exemplos na vida. À minha irmã, Letícia, por ser meu alívio cômico por todos esses anos. À minha família, por compreender 
minhas ausências e por todos os aprendizados que me proporcionaram.

À minha companheira, Juliana, por ter me apoiado, incentivado e compreendido durante essa trajetória. Por ter me 
ajudado a manter a calma e acreditar que eu era capaz de superar qualquer obstáculo. Obrigado por ser a
minha maior incentivadora e por nunca me deixar desistir nos momentos de dificuldade. Seu amor e paciência foram
essenciais para que eu pudesse concluir este trabalho.

Ao meu amigo Diego e demais amigos que fiz durante a graduação, que sem dúvidas espero levar para a vida toda. Sem vocês o caminho
teria sido muito mais difícil. À Faraday E-Racing, que representou um marco na minha trajetória acadêmica e me proporcionou 
experiências essenciais para a minha formação. À equipe de Tecnologias da Superintendência de Geração de 
Energia (SGR) da Empresa de Pesquisa Energética (EPE), pelas oportunidades e aprendizado durante o meu período de estágio 
que, além de terem sido fundamentais para a minha formação como engenheiro, possibilitaram o desenvolvimento deste trabalho. 

Ao professor André Pinho, pela sua excelência, maestria em ensinar e por ter me orientado de maneira exemplar durante 
o desenvolvimento deste projeto. Agradeço também aos professores Flávio Martins, Felipe Sass e Marcio Guimaraens, por me 
lembrarem em cada aula do motivo pelo qual escolhi a Engenharia Elétrica.

\end{agradecimentos}

% Epígrafe
%\include{pretextuais/epigrafe}

% Resumos
% resumo em português
\begin{resumo}
\noindent
A matriz elétrica brasileira, caracterizada pela forte dependência da geração hidráulica, apresenta vulnerabilidade a fenômenos
climáticos como o El Niño-Oscilação Sul (ENSO), cujos impactos não são diretamente considerados nos modelos de 
planejamento energético do país. Este trabalho investiga o impacto de variáveis exógenas associadas ao ENSO na estimação da geração 
das fontes hidráulica, eólica e térmica no Sistema Interligado Nacional (SIN). Para tal, foram desenvolvidos e comparados modelos de 
regressão linear, não-linear (Random Forest) e uma arquitetura de rede neural pré-treinada (Tiny Time Mixer). O modelo 
neural foi avaliado em sua aplicação direta, utilizando os pesos pré-treinados, e então foi realizado o processo de 
ajuste fino (fine-tuning) para especializar o modelo com os dados climáticos. Os resultados indicam que os modelos de regressão 
tradicionais não são capazes de capturar as dinâmicas não-lineares das fontes eólica e térmica. O modelo neural, por sua vez,
demonstrou alta capacidade de generalização, e o processo de fine-tuning provou ser a abordagem mais eficaz, resultando em uma diminuição 
do erro médio quadrático (MSE) de até 21,38\% para a fonte hidráulica. Conclui-se que a incorporação de variáveis climáticas externas 
em modelos neurais avançados aprimora significativamente a acurácia da estimação de geração, apresentando-se como uma metodologia promissora
para compor as ferramentas de planejamento energético do setor elétrico.

\vspace{0.2cm}
\noindent
Palavras-chave: Geração de energia. Clima. Planejamento energético. Machine learning.
\end{resumo}

% resumo em inglês
\begin{resumo}[Abstract]	
\begin{otherlanguage*}{english}
\noindent 
\textit{The Brazilian electrical matrix, characterized by its strong dependence on hydroelectric generation, is vulnerable to climate 
phenomena such as the El Niño-Southern Oscillation (ENSO). This work investigates the impact of exogenous variables associated with 
ENSO on the generation forecasting of hydro, wind, and thermal power sources within the National Interconnected System (SIN). To this end, 
linear regression, non-linear (Random Forest), and a pre-trained neural network architecture (Tiny Time Mixer) models were developed and compared. 
The neural network was evaluated both in its direct application and through a fine-tuning process to specialize the model with climate data. The 
results, evaluated by the Mean Squared Error (MSE), indicated that traditional regression models were inadequate for the wind and thermal sources. 
The neural approach demonstrated superiority, and the fine-tuning process proved to be the most effective, achieving an MSE reduction of up to 
21.38\% for the hydraulic source compared to the non-adjusted model. It is concluded that incorporating external climate variables into advanced 
neural networks significantly improves the accuracy of generation forecasting, validating the methodology as a promising tool to complement the 
energy planning frameworks of the electrical sector.}

\vspace{0.2cm}
\noindent
Key-words: Energy generation. Climate. Risk mitigation. Machine learning.
\end{otherlanguage*}
\end{resumo}

% Lista de ilustrações, tabelas e algoritmos
\renewcommand{\listfigurename}{}                                                                                        % Retira o nome da lista
\pretextualchapter{Lista de Figuras}                                                                                    % Capítulo sem numeração
\listoffigures*                                                                                                         % * indica que a lista não vai para o sumário
\cleardoublepage

\renewcommand{\listtablename}{}
\pretextualchapter{Lista de Tabelas}
\listoftables*
\cleardoublepage

\cleardoublepage

% Lista de abreviaturas e siglas
\begin{siglas}
  \item[ACR] Ambiente de Contratação Regulada
  \item[API] Aplication Programming Interface (Interface de Programação de Aplicações)
  \item[CDS] Climate Data Store
  \item[CEPEL] Centro de Pesquisas de Energia Elétrica
  \item[CPAMP] Comissão Permanente para Análise de Metodologias e Programas Computacionais do Setor Elétrico
  \item[CSV] Comma Separated Values (Valores Separados por Vírgula)
  \item[ECMWF] European Centre for Medium-Range Weather Forecasts
  \item[EN] El-Niño
  \item[ENSO] El Niño-Oscilação Sul
  \item[ERA5] European Centre for Medium-Range Weather Forecasts Reanalysis 5
  \item[EPE] Empresa de Pesquisa Energética
  \item[GPU] Graphics Processing Unit (Unidade de Processamento Gráfico)
  \item[IBGE] Instituto Brasileiro de Geografia e Estatística
  \item[LN] La-Niña
  \item[LOESS] Local Polynomial Regression Fitting (Ajuste de Regressão Polinomial Local)
  \item[MLP] Multi Layer Perceptrons
  \item[MSE] Erro Quadrático Médio
  \item[MSTL] Multiple Seasonal-Trend decomposition using LOESS
  \item[MWh] Megawatt-hora
  \item[MWmed] Megawatt médio
  \item[NEWAVE] Modelo de Otimização Energética
  \item[ONI] Oceanic Niño Index
  \item[ONS] Operador Nacional do Sistema Elétrico
  \item[PDDE] Programação Dinâmica Dual Estocástica
  \item[PDE] Plano Decenal de Expansão de Energia
  \item[PEE] Parques Eólicos Equivalentes
  \item[PEN] Plano de Operação Energética
  \item[PLD] Preço de Liquidação das Diferenças
  \item[PMO] Programa Mensal de Operação
  \item[R²] Coeficiente de Determinação
  \item[REE] Reservatórios Equivalentes de Energia
  \item[SIN] Sistema Interligado Nacional
  \item[SST] Sea Surface Temperature (Temperatura da Superfície do Mar)
  \item[TTM] Tiny Time Mixer
  \item[UFF] Universidade Federal Fluminense
\end{siglas}

% Sumário
\renewcommand{\contentsname}{}	% Retira o nome da lista
\pretextualchapter{Sumário}
\tableofcontents*
\cleardoublepage

% Elementos textuais
% ----------------------------------------------------------------------------------------------------------------------
\textual

% Introdução
% Introdução e Trabalhos Relacionados (capítulos 1 e 2)
% ----------------------------------------------------------------------------------------------------------------------
\chapter{Introdução}
\sloppy																													% Para justificar o texto

\section{Contexto}
Historicamente, a matriz elétrica brasileira é considerada uma das mais limpas do mundo, com destaque para a fonte
hidráulica, que é responsável pela maior parte da geração de energia elétrica no país. Nos últimos anos, outras fontes
de geração vêm sendo incorporadas ao sistema, das quais destacam-se a eólica e solar fotovoltaica, conforme observado na
Figura \ref{fig:geracao_anual_por_fonte}, elaborada a partir de dados brutos de geração centralizada obtidos do Operador
Nacional do Sistema Elétrico (ONS), sem considerar a geração distribuída.

\begin{figure}[!ht]
	\IBGEtab{\caption{Geração centralizada anual por fonte}
			 \label{fig:geracao_anual_por_fonte}}
	{\includesvg[scale=1]{figuras/geracao_anual_por_fonte}}
	{\fonte{o autor.}}
\end{figure}

Nota-se, em especial, um crescimento significativo da geração eólica, observado a partir de 2015, e uma diminuição 
significativa da contribuição de geração térmica média no panorama geral nos anos seguintes. Em 2023, a fonte eólica 
foi responsável por 48\% da expansão da capacidade instalada total de 10,19 GW \cite{EPE2024}. Essa expansão se dá em 
função do maior número de empreendimentos participantes nos Leilões de Energia Elétrica do Ambiente de Contratação 
Regulada (ACR) realizados pela Empresa de Pesquisa Energética (EPE). Isso ocorre, dentre outros fatores, devido à queda 
nos custos de aerogeradores e painéis fotovoltaicos, além do fator "combustível zero" dessas fontes, o que torna novos 
empreendimentos mais atrativos economicamente para os agentes.

Embora essa expansão seja positiva, poupando recursos hídricos, contribuindo para a diversificação da matriz elétrica e
reduzindo o acionamento de usinas térmicas, essas fontes possuem características intrínsecas que as tornam intermitentes,
como a incidência solar e a velocidade do vento. Essas variáveis possuem sazonalidades de curto e longo prazo, a 
depender da hora do dia e estação do ano, por exemplo. Sendo assim, uma alta dependência dessas fontes tem o potencial
de tornar o sistema como um todo mais vulnerável.

Além disso, ao analisar a curva de carga do SIN, observa-se que, embora o seu pico ocorra no início da 
tarde, momento no qual a geração solar fotovoltaica apresenta significativa contribuição, o período noturno também 
apresenta carga considerável, conforme a Figura \ref{fig:carga_max_dia}, que mostra a curva de carga do SIN para o 
dia 15 de março de 2024, dia em que registrou-se um recorde de demanda máxima instantânea de 102.478 MW, segundo o ONS, 
e como pode ser observado na Figura \ref{fig:carga_anual}.

\begin{figure}[!ht]
	\IBGEtab{\caption{Curva de carga diária do SIN em base horária}
			 \label{fig:carga_max_dia}}
	{\includesvg[scale=1]{figuras/carga_max_dia_2024-03-15}}
	{\fonte{o autor.}}
\end{figure}

\section{Motivação}
Em um contexto no qual a implementação de sistemas de armazenamento de energia elétrica ainda é incipiente,
a matriz segue bastante dependente da fonte hidráulica e, de maneira complementar, das térmicas. A dependência da fonte
hidráulica, por sua vez, torna o sistema elétrico vulnerável a eventos climáticos extremos ocasionados pelas mudanças
climáticas. Por exemplo, em 2021, verificou-se um acionamento recorde de usinas térmicas e uma geração hidráulica 
percentual mínima. Isso se deve em razão da forte crise hídrica enfrentada pelo Brasil em 2021, a pior dos últimos 91 
até então. \cite{Soares2023}

\begin{figure}[!ht]
	\IBGEtab{\caption{Curva de carga do SIN em base mensal}
			 \label{fig:carga_anual}}
	{\includesvg[scale=1]{figuras/carga_anual}}
	{\fonte{o autor.}}
\end{figure}

Portanto, o estudo da operação do sistema elétrico brasileiro, no contexto de cenários de eventos
climatológicos extremos é altamente relevante para a segurança energética do país, considerando uma estimativa de 
crescimento médio anual da carga do SIN de 3,2\%. \cite{pen2024}

Ao analisar a geração hidráulica bruta na Figura \ref{fig:geracao_hidraulica_bruta}, evidenciam-se pontos nos 
quais a geração é reduzida. Isso ocorre devido à sazonalidade das vazões nas bacias hidrográficas, responsáveis pelo 
abastecimento dos reservatórios. Considerando a amostragem em base mensal, observa-se que a geração é reduzida nos meses
de inverno, período caracterizado por menor ocorrência de precipitação e, consequentemente, menor vazão nos rios. Por
outro lado, nos meses de verão, a geração atinge seus maiores valores.

\begin{figure}[!ht]
	\IBGEtab{\caption{Geração hidráulica total em base mensal}
			 \label{fig:geracao_hidraulica_bruta}}
	{\includesvg[scale=1]{figuras/geracao_hidraulica_bruta}}
	{\fonte{o autor.}}
\end{figure}

Esse comportamento é natural e esperado, uma vez que a geração hidráulica é diretamente influenciada pelas condições
que afetam a vazão dos rios. No entanto, a ocorrência de eventos climatológicos como o \textit{El-Niño} (EN) e 
\textit{La-Niña} (LN) pode favorecer condições que impactam diretamente no potencial de geração hidráulica. \cite{de2012influencia}

Fenômenos como o EN e LN são caracterizados por anomalias na temperatura da superfície do mar no Oceano Pacífico
Equatorial. Essas anomalias são monitoradas por meio de índices como o ONI (Oceanic Niño Index), que classifica os 
eventos em três categorias: EN, LN e neutro. A Figura \ref{fig:oni} mostra a classificação dos eventos de EN e LN 
ocorridos entre 2000 e 2024, sendo que a escala de cores indica a intensidade do evento. Ao analisar a geração 
hidráulica bruta no mesmo período, observa-se que a ocorrência de eventos de EN e LN pode estar associada a variações na geração.

\begin{figure}[!ht]
	\IBGEtab{\caption{Índice ONI (Oceanic Niño Index)}
			 \label{fig:oni}}
	{\includesvg[scale=1]{figuras/oni}}
	{\fonte{o autor.}}
\end{figure}

Fundamentalmente, em sistemas interligados cuja fonte hidráulica constitui a base da matriz elétrica, é essencial, 
para um planejamento energético eficiente, otimizar o sistema de modo a considerar a operação de todas as usinas, 
considerando a incerteza associada às afluências futuras. Dessa forma, estima-se o valor da geração hidrelétrica que 
poderia substituir a geração térmica a curto ou longo prazo, de modo a minimizar os custos de operação do sistema e o 
risco de utilizar reservatórios de maneira desnecessária, garantindo assim o atendimento à demanda futura, 
principalmente em casos de escassez hídrica.

Os estudos de planejamento energético são realizados por meio de modelos computacionais como o NEWAVE, DECOMP e DESSEM, 
que consdieram diferentes horizontes temporais: longo, médio e curto prazos, respectivamente. Também há outras soluções 
disponíveis, como o PSR SDDP, que engloba todos os horizontes temporais. Considerando o escopo deste trabalho, o modelo 
NEWAVE será brevemente apresentado a seguir.

\subsection{O modelo NEWAVE}
\label{sec:newave}

Desenvolvido e mantido pelo Centro de Pesquisas de Energia Elétrica (CEPEL) e amplamente utilizado pelo setor elétrico 
brasileiro para definição de estratégias e tomada de decisão, o NEWAVE é um modelo de otimização que busca minimizar os 
custos de operação do sistema, considerando a incerteza das afluências futuras e a operação de um sistema 
hidro-térmico-eólico interligado. O modelo é utilizado para estudos como:
\begin{itemize}
	\item Elaboração do Plano Decenal de Expansão de Energia (PDE), pela EPE;
	\item Elaboração do Programa Mensal de Operação (PMO) e Plano de Operação Energética (PEN), pelo ONS;
	\item Formação de preços, como no cálculo do Preço de Liquidação das Diferenças (PLD) pelo CCEE;
	\item Cálculo de Garantia Física e da Energia Assegurada para empreendimentos de geração participantes nos leilões de energia elétrica, pela EPE;
	\item Elaboração de diretrizes para os leilões de energia, pela EPE.
\end{itemize}

Em resumo, o modelo emprega a Programação Dinâmica Dual Estocástica (PDDE), uma técnica de otimização que permite lidar 
com as incertezas ligadas às afluências futuras sem que o modelo se torne computacionalmente impraticável, considerando 
múltiplos reservatórios, interconexões e o horizonte temporal de médio e longo prazos.

O NEWAVE modela o sistema de geração hidrelétrico em Reservatórios Equivalentes de Energia (REE), que são grupos de
usinas associadas a um subsistema ou submercado de energia. Cada subsistema pode conter mais de um REE, possibilitando
diferenciar bacias hidrográficas com regimes distintos, ainda que pertençam a um mesmo subsistema. Além disso, cada REE
é definido por um conjunto de parâmetros que são calculados a partir das características indivuduais de cada usina. Nas
versões mais recentes do modelo, também é possível considerar todas as usinas indivudalmente ou operar de maneira
híbrtida, ou seja, considerando alguns REEs e outras usinas individualmente.

As usinas termelétricas são representadas no modelo através de classes térmicas. Cada classe agrupa usinas com custos 
semelhantes e está associada a um subsistema. Cada classe também é definida por um conjunto de parâmetros calculados
a partir das características individuais de cada usina.

Nas versões mais recentes do modelo, a fonte eólica também é modelada. De maneira similar, os parques eólicos são agrupados
em Parques Eólicos Equivalentes (PEE). O agrupamento é feito a aprtir de dados de cadastro de cada prque eólico, estado,
submercado, função de produção (curva relacionando a velocidade do vento com a potência gerada), dados sobre torres de
medição e séries históricas de velocidade do vento.

O modelo requer um conjunto de dados de entrada que inclui as características das usinas, dados dos subsistemas, demanda,
séries históricas de vazões e ventos, cronogramas de expansão, restriçòes operativas, entre outros. Observa-se que todos
os dados de entrada são locais e, portanto, o modelo não considera variáveis externas, como fenomênos climáticos como o 
EN e LN, que podem impactar a geração de energia elétrica. No entanto, vale ressaltar que as últimas versões do modelo
apresentam campos previstos para a entrada de dados do ENSO, mas estão marcados como "não implementados".

Vale destacar que essas variáveis externas podem ser utilizadas para elaborar as séries históricas de vazões e velocidade
de ventos utilizadas como dados de entrada. Dessa forma, entende-se que o modelo não considera diretamente o impacto 
dessas variáveis.

\section{Objetivo}
Sendo assim, o projeto tem como objetivo investigar um possível impacto de variáveis climatológicas externas na geração 
de energia elétrica no Brasil, com foco nas fontes hidráulica, térmica e eólica. 

Para tanto, foram empregadas técnicas computacionais para relacionar as séries históricas de geração com as séries de variáveis 
climáticas, através de modelos lineares, não lineares e neurais. Além disso, foram utilizadas séries históricas de geração, 
carga e vazões disponibilizados pelo ONS, bem como também séries históricas de variáveis climatológicas, como temperatura 
da superfície do mar, obtidas a partir de dados do ERA5, um projeto de reanálise atmosférica do \textit{European Centre for Medium-Range Weather Forecasts} (ECMWF).

A partir dessa investigação, espera-se poder avaliar o impacto de variáveis climáticas na geração de energia elétrica,
o que pode contribuir para o planejamento energético do país, especialmente em cenários de eventos climáticos extremos e
tendências climáticas.

É importante salientar que outras variáveis externas poderiam ser incorporadas ao estudo, ou até mesmo uma combinação entre
 variáveis locais e externas. Também poderiam ser considerados indicadores econômicos e outros dados relevantes. 
No entanto, este trabalho considera apenas as variáveis relacionadas ao fenômeno EN e LN, uma vez que estudos indicam
uma alta correlação entre esses fenômenos e o regime de chuvas no Brasil \cite{de2012influencia, Andreoli2016}, sendo a 
incorporação de outras variáveis reservada para trabalhos futuros.

\section{Estrutura do Trabalho}
No capítulo 1, é feita uma breve introdução apresentando o contexto, motivação, objetivo e a estrutura do trabalho. Uma
breve análise da matriz elétrica é apresentada, com foco no histórico recente e no crescimento da geração eólica. Também
são apresentadas as curvas de carga do SIN e de geração hidráulica. Uma breve contextualização acerca dos fenômenos do
ENSO e a importância do modelo NEWAVE para o planejamento energético do país são apresentadas. Por fim, é apresentado
o objetivo do trabalho e a estrutura do documento.

No capítulo 2, a fundamentação teórica necessária para a compreensão do projeto é apresentada. São abordados os fatores
que fazem com que os fenômenos do EN e LN impactem o regime de chuvas no Brasil e a sua relação com a geração de 
energia elétrica. Também será feita uma breve introdução ao modelo NEWAVE, que é amplamente utilizado pelo setor elétrico
brasileiro para planejamento energético. Por fim, serão apresentados os modelos de previsão de séries temporais 
implementados, com foco no modelo neural, uma implementação baseada na arquitetura \textit{TSMixer} desenvolvida pela Google.

O capítulo 3 apresenta a metodologia utilizada para a realização do projeto. Serão apresentados os conjuntos de dados
considerados e suas respectivas etapas de obtenção, tratamento e análise. Além disso, será apresentada a metodologia
utilizada para implementação dos modelos de previsão, incluindo seus parâmetros e métricas de avaliação. \textit{Snippets}
de códigos serão apresentados para facilitar a compreensão.

O capítulo 4 apresenta os resultados obtidos a partir da implementação de cada modelo de previsão. Inicia-se com os resultados
do modelo linear, seguido pelo modelo não linear e, por fim, o modelo neural. Para cada modelo, serão apresentados os
resultados de previsão, métricas de avaliação e uma análise crítica dos resultados obtidos.

Por fim, no capítulo 5, são apresentadas as considerações finais do trabalho, incluindo uma discussão final dos resultados
obtidos, limitações observadas e sugestões para trabalhos futuros.

% \chapter{Trabalhos Relacionados}
% Avaliação do Impacto de Secas Severas no Nordeste Brasileiro na Geração de Energia Elétrica Através do Modelo Newave: 
% Projeção das Energias Afluentes e Armazenadas \cite{Vilar2020}

% {https://files.abrhidro.org.br/Eventos/Trabalhos/60/PAP023274.pdf}

% {https://pantheon.ufrj.br/bitstream/11422/21709/1/887353.pdf}

% Fundamentação Teórica
% Apresentar estudos que contemple a temática abordada. Respeitar a autoria,
% nas citações diretas e indiretas. Evitar parágrafos muito longos. Evitar seções e
% subseções muito curtas.

\chapter{Fundamentação Teórica}
\section{Impacto do ENSO na Geração de Energia Elétrica}
O ENSO é um fenômeno que ocorre no Oceano Pacífico Equatorial, caracterizado por variações na temperatura da superfície
do mar (TSM) em regiões específicas, como ilustrado na Figura \ref{fig:regioes_enso}. Segundo \citeonline{Andreoli2016}, o 
fenômeno é um dos principais fatores que influenciam os padrões de vento e precipitação em diversas regiões da América 
do Sul e seus efeitos se estendem por todas as regiões do Brasil.

De acordo com \citeonline{Capozzoli2017}, diferentes fases do ENSO resultam em padrões distintos de precipitação sobre
o território brasileiro, variando de acordo com a região e a estação do ano. Sendo assim, há um impacto direto sobre a 
disponibilidade de recursos hídricos para a geração hidrelétrica.

A região Sul é uma das mais consistentemente afetadas. Eventos de El Niño tendem a causar precipitação acima da média,
particularmente durante a primavera e o verão, enquanto eventos de La Niña estão associados à condições de seca.

A região Sudeste apresenta uma resposta mais complexa, sendo consdierada uma zona de transição. A bacia do Rio Paraná,
em especial, apresenta sensibilidade aos fenômenos do ENSO, tendo apresentado tendência de aumento de vazão durante alguns
eventos de El Niño.

Para as regiões Norte e Nordeste, eventos de El Niño estão associados a períodos de seca, enquanto eventos de
La Niña tendem a trazer chuvas acima da média. No entanto, é importante destacar que outros fenômenos atmosféricos podem
interferir com esses padrões, modulando os efeitos do ENSO.

Sendo assim, verifica-se que as variações induzidas pelos fenômenos do ENSO traduzem-se diretamente em variações nas
vazões dos rios que alimentam as bacias, que por sua vez impactam diretamente o potencial de geração da fonte hidráulica.

\begin{figure}[!ht]
	\IBGEtab{\caption{Regiões do fenômeno El Niño-Oscilação Sul (ENSO)}
			 \label{fig:regioes_enso}}
	{\includesvg[scale=1]{figuras/regioes_enso_global}}
	{\fonte{o autor.}}
\end{figure}

\section{O Modelo NEWAVE}
\label{sec:newave}

Desenvolvido e mantido pelo Centro de Pesquisas de Energia Elétrica (CEPEL) e amplamente utilizado pelo setor elétrico 
brasileiro para definição de estratégias e tomada de decisão, o NEWAVE é um modelo de otimização que busca minimizar os 
custos de operação do sistema, considerando a incerteza das afluências futuras e a operação de um sistema 
hidro-térmico-eólico interligado. O modelo é utilizado para estudos como:
\begin{itemize}
	\item Elaboração do Plano Decenal de Expansão de Energia (PDE), pela EPE;
	\item Elaboração do Programa Mensal de Operação (PMO) e Plano de Operação Energética (PEN), pelo ONS;
	\item Formação de preços, como no cálculo do Preço de Liquidação das Diferenças (PLD) pelo CCEE;
	\item Cálculo de Garantia Física e da Energia Assegurada para empreendimentos de geração participantes nos leilões 
    de energia elétrica, pela EPE;
	\item Elaboração de diretrizes para os leilões de energia, pela EPE.
\end{itemize}

Em resumo, o modelo emprega a Programação Dinâmica Dual Estocástica (PDDE), uma técnica de otimização que permite lidar 
com as incertezas ligadas às afluências futuras sem que o modelo se torne computacionalmente impraticável, considerando 
múltiplos reservatórios, interconexões e o horizonte temporal de médio e longo prazos.

\subsection{Representação das Usinas}
O NEWAVE modela o sistema de geração hidrelétrico em Reservatórios Equivalentes de Energia (REEs), que são grupos de
usinas associadas a um subsistema ou submercado de energia. Cada subsistema pode conter mais de um REE, possibilitando
diferenciar bacias hidrográficas com regimes distintos, ainda que pertençam a um mesmo subsistema. 

Além disso, cada REE é definido por um conjunto de parâmetros que são calculados a partir das características indivuduais 
de cada usina. Nas versões mais recentes do modelo, também é possível considerar todas as usinas indivudalmente ou operar
de maneira híbrtida, ou seja, considerando alguns REEs e outras usinas individualmente.

As usinas termelétricas são representadas no modelo através de classes térmicas. Cada classe agrupa usinas com custos 
semelhantes e está associada a um subsistema. Cada classe também é definida por um conjunto de parâmetros calculados
a partir das características individuais de cada usina.

Nas versões mais recentes do modelo, a fonte eólica também é modelada. De maneira similar, os parques eólicos são agrupados
em Parques Eólicos Equivalentes (PEE). O agrupamento é feito a aprtir de dados de cadastro de cada prque eólico, estado,
submercado, função de produção (curva relacionando a velocidade do vento com a potência gerada), dados sobre torres de
medição e séries históricas de velocidade do vento.

\subsection{Dados de Entrada}
O modelo requer um conjunto de dados de entrada que inclui as características das usinas, dados dos subsistemas, demanda,
séries históricas de vazões e ventos, cronogramas de expansão, restrições operativas, dentre outros. Observa-se, portanto, 
que todos os dados de entrada são locais e, portanto, o modelo não considera variáveis externas, como fenomênos climáticos
como o EN e LN, que podem impactar a geração de energia elétrica. 

Ainda que as últimas versões do modelo apresentem campos previstos para a entrada de dados do ENSO, esses campos
estão marcados como "não implementados". Dessa forma, entende-se que o modelo não considera diretamente o impacto 
dessas variáveis. No entanto, vale destacar que essas variáveis externas podem ser utilizadas para elaborar as séries históricas de 
vazões e velocidade de ventos utilizadas como dados de entrada. 

\section{Modelos Linear e Não-linear}
fazer uma introdução sobre os modelos lineares e não lineares...

\subsection{Modelo Linear}

\subsection{Modelos Não-lineares: Random Forest e Gradient Boosting}

\section{Modelo Neural}
Atualmente, os modelos neurais mais avançados utilizam a arquitetura \textit{Transformer}, que são modelos neurais com um
mecanismo de atenção que permite ao modelo focar em diferentes partes da entrada de dados ao processar informações. Essa
arquitetura foi introduzida no artigo \textit{"Attention is All You Need"} de \citeonline{Vaswani2017} e hoje é a arquitetura
por trás dos principais \textit{Large Language Models} (LLMs) comercialmente disponíveis.

Inicialmente concebido para tarefas de tradução, modelos baseados nessa arquitetura demonstraram resultados superiores
em outras aplicações e hoje são amplamente utilizados em diversas áreas, incluindo previsão de séries temporais. No entanto,
esses modelos são altamente complexos e exigem um alto poder computacional para treinamento e inferência.

Além disso, conforme demonstrado por \citeonline{Zeng2022}, modelos baseados em \textit{Transformers} podem produzir
resultados inferiores quando comparados a modelos mais simples. 

Dessa forma, uma abordagem alternativa foi proposta por \citeonline{Chen2023}, uma arquitetura mais simples em comparação
aos \textit{Transformers}, mas com resultados bastante superiores



% Desenvolvimento
\chapter{Metodologia} % ============================================================================================== %
\section{Abordagem Computacional} % ================================================================================== %

O projeto foi desenvolvido em \textit{Python} versão 3.12. Toda as etapas, da obtenção dos dados de entrada à implementação
dos modelos computacionais foram organizadas em módulos. Todas as etapas de processamento intensivo foram realizadas em
paralelo, utilizando todas as \textit{threads} disponíveis do sistema. As etapas referentes ao modelo neural foram 
realizadas com aceleração da \textit{Graphics Processing Unit} (GPU).

Considerando que o suporte a aceleração por GPU é limitado no Windows, foi necessário realizar as etapas de treinamento do modelo 
neural em Linux, utilizando a distribuição Fedora 42. A aceleração por GPU foi necessária para reduzir o tempo de treinamento do modelo. 
O sistema utilizado possui um processador AMD Ryzen 5900X, 32 GB de memória RAM e placa de vídeo AMD RX 6800 XT. A tabela 
\ref{tab:bibliotecas} mostra as bibliotecas utilizadas no projeto, suas finalidades e versões.

\begin{table}[htb]
  \centering
   \IBGEtab{
    \caption{Bibliotecas utilizadas no projeto}
    \label{tab:bibliotecas}
    }{
    \begin{tabular}{llll}
  		  \hline
    	  \textbf{Biblioteca} & \textbf{Descrição} & \textbf{Versão} \\ \hline
        numpy & Cálculos numéricos e manipulação de arrays & 1.26.4 \\
        pandas & Manipulação e análise de dados (DataFrames) & 2.2.3 \\
        requests & Requisições HTTP & 2.32.3 \\
        urllib3 & Gerenciamento de conexões HTTP & 2.2.3 \\
        alive\_progress & Barra de progresso para loops & 3.2.0 \\
        netCDF4 & Leitura de arquivos NetCDF & 1.7.2 \\
        cdsapi & API para download de dados do ECMWF & 0.7.5 \\
        geopandas & Manipulação de dados geoespaciais & 1.0.1 \\
        matplotlib & Visualização de dados & 3.9.2 \\
        scikit-learn & Aplicação de modelos iniciais & 1.5.2 \\
        scipy & Ferramentas e algoritmos matemáticos & 1.14.1 \\
        transformers & Modelos Neurais Pré-treinadis & 4.52.3 \\
        torch & Processamento de Redes Neurais & 2.7.0 \\
        \hline
    \end{tabular}
    }{
      \fonte{o autor.}}
\end{table}

\section{Obtenção e Pré Processamento dos Dados} % =================================================================== %
\subsection{Obtenção dos Dados Energéticos e Hidrológicos} % ========================================================= %

A primeira etapa do projeto consiste na consolidação das séries históricas de geração, carga e variáveis hidrológicas,
que são disponibilizadas publicamente no portal Dados Abertos do ONS, a partir do ano 2000. As séries referentes 
às variáveis hidrológicas são disponibilizadas em base diária, e os dados de geração e carga são disponibilizados em 
base horária. As Tabelas \ref{tab:hidrologicos}, \ref{tab:carga} e \ref{tab:geracao} mostram os parâmetros dos dados hidrológicos, 
carga e geração, respectivamente.

Os dados de geração são disponibilizados em Mega Watt médio (MWmed) por fonte de energia, subsistema, estado, 
modalidade de operação, entre outras variáveis. Os dados de carga também são disponibilizados em MWmed e contêm 
informações sobre a carga em cada subsistema do SIN.

\begin{table}[htb]
  \centering
   \IBGEtab{
    \caption{Parâmetros dos dados hidrológicos}
    \label{tab:hidrologicos}
    }{
      \begin{tabular}{llll}
          \hline
          \textbf{Parâmetro} & \textbf{Descrição} & \textbf{Tipo} \\ \hline
          din\_instante & Instante de aferição & Datetime\\
          nom\_subsistema & Subsistema & String\\
          tip\_reservatorio & Tipo de reservatório & String\\
          nom\_bacia  & Bacia hidrográfica & String\\
          nom\_ree & Nome do REE & String\\
          val\_nivelmontante & Valor do nível montante (m) & Float\\
          val\_niveljusante & Valor do nível jusante (m) & Float\\
          val\_volumeutilcon & Volume útil consistido (\%) & Float\\
          val\_vazaoafluente & Vazão afluente (m³/s) & Float\\
          val\_vazaoturbinada & Vazão turbinada (m³/s) & Float\\
          val\_vazaovertida & Vazão vertida (m³/s) & Float\\
          val\_vazaodefluente & Vazão defluente (m³/s) & Float\\
          val\_vazaoevaporacaoliquida & Vazão de evaporação líquida (m³/s) & Float\\ \hline
      \end{tabular}
    } {
      \fonte{\citeonline{pen2024}}
      \nota{Variáveis não utilizadas foram omitidas.}
    }
\end{table}

Para as séries de geração, os dados de 2000 a 2021 são agrupados pelos respectivos anos, e a partir de 2022,
as informações estão agrupadas em arquivos por mês e ano. Para as séries de carga, os dados são disponibilizados por
ano. Como o ONS não disponibiliza \textit{Aplication Programming Interface} (API) para a obtenção dos dados diretamente, foi 
necessário uma outra abordagem, a fim de evitar o \textit{download} manual dos dados. 

\begin{table}[htb]
  \centering
   \IBGEtab{
    \caption{Parâmetros dos dados de carga}
    \label{tab:carga}
    }{
    \begin{tabular}{llll}
  		  \hline
    	  \textbf{Parâmetro} & \textbf{Descrição} & \textbf{Tipo} \\ \hline
        din\_instante & Instante de aferição & Datetime\\
        nom\_subsistema & Subsistema da usina & String\\
        val\_cargaenergiahomwmed & Carga de energia (MWmed) & Float\\ \hline
    \end{tabular}
    }{
      \fonte{\citeonline{pen2024}}}
\end{table}

Um script foi desenvolvido para fazer o download dos dados por meio de requisições HTTP, utilizando as bibliotecas 
\code{requests} e \code{urllib3} para gerenciar as conexões. Todos os downloads foram realizados em paralelo, utilizando 
todas as threads disponíveis do sistema. Ao todo, cerca de 10 GB de dados em arquivos Comma Separated Values (CSV) foram consolidados.

\begin{table}[htb]
  \centering
   \IBGEtab{
    \caption{Parâmetros dos dados de geração}
    \label{tab:geracao}
    }{
      \begin{tabular}{llll}
          \hline
          \textbf{Parâmetro} & \textbf{Descrição} & \textbf{Tipo} \\ \hline
          id\_subsistema & Instante de aferição & Datetime\\
          nom\_subsistema & Subsistema da usina & String\\
          id\_estado & Estado onde a usina está localizada & String\\ 
          nom\_tipousina & Tipo de usina & String\\
          nom\_tipocombustivel & Tipo de combustível & String\\
          nom\_usina & Nome da usina & String\\
          val\_geracao & Geração de energia (MWmed) & Float\\ \hline
      \end{tabular}
    } {
      \fonte{\citeonline{pen2024}}
      \nota{Variáveis não utilizadas foram omitidas.}
    }
\end{table}

Os dados de geração contém informações que permitem uma análise detalhada em diferentes níveis de granulidade. Dessa maneira, 
possíveis impactos em diferentes escalas geográficas e temporais poderão ser avaliados.

\subsection{Pré Processamento dos Dados de Geração e Carga} % ======================================================== %
Com as séries históricas de geração e carga consolidadas, se faz necessário preparar os dados para que possam ser utilizados 
nos modelos computacionais. Essa etapa envolve a verificação, transformação e limpeza dos dados. Utilizando a biblioteca
\code{pandas}, todos os arquivos com o histórico de geração e carga foram lidos e consolidados em dois \code{DataFrames} 
distintos, estrutura de dados tabulares da biblioteca. Com isso, as seguintes operações foram realizadas:
\begin{itemize}
    \item Seleção das colunas relevantes;
    \item Verificação de valores inválidos e tratamento de valores ausentes;
    \item Agrupamento dos tipos de usinas por classes: hidráulica, térmica, eólica, fotovoltaica e outras;
    \item Reamostragem para diferentes bases temporais, considerando a geração média (MWmed) e energia gerada (MWh);
    \item Agrupamento dos dados de geração por subsistema e classe.
\end{itemize}

\begin{figure}[!ht]
	\IBGEtab{\caption{Subsistemas do SIN segundo o ONS}
			 \label{fig:subsistemas_brasil}}
	{\includesvg[scale=1]{figuras/subsistemas_brasil}}
	{\fonte{o autor.}}
\end{figure}

Optou-se por fazer o agrupamento dos dados de geração por subsistema e fonte para permitir avaliar os impactos das variáveis
do ENSO em diferentes regiões do Brasil. Além disso, a representação por subsistema também é utilizada pelo modelo NEWAVE. 
A figura \ref{fig:subsistemas_brasil} mostra os subsistemas do SIN.

Diferentes arquivos CSV foram gerados, considerando a geração média (MWmed) e energia gerada (MWh) para as escalas diária e mensal.
Os dados de geração contém o histórico de geração por subsistema e fonte, enquanto que os dados de carga contém o histórico de carga
por subsistema. Considerando o período de 2000 a 2024, foram consolidados ao todo 9132 amostras diárias e 300 amostras mensais.

\subsection{Pré Processamento dos Dados Hidrológicos} % ============================================================== %
Os dados hidrológicos contém informações sobre o nível dos reservatórios e vazões. Para o contexto do projeto, espera-se que
essas informações sejam altamente correlacionados com a geração hidrelétrica, o que será verificado a seguir na seção de análise 
exploratória.

Sendo assim, os dados hidrológicos foram utilizados para verificar a performance dos modelos quando as variáveis de entrada
possuem alta correlação com a variável de saída. Dessa forma, é estabelecida uma referência para comparação entre os
resultados dos modelos.

As séries históricas foram consolidadas em um único \code{DataFrame}, contendo as vazões totais em cada subsistema do SIN,
considerando a soma das vazões de todos os reservatórios em cada subsistema, em base diária.

\subsection{Obtenção dos Dados do ENSO} % ============================================================================ %
Os dados do ENSO foram obtidos a partir do \textit{Climate Data Store} (CDS) do ECMWF. O CDS é um banco de dados com diversos
\textit{datasets} de variáveis climáticas de diferentes regiões do mundo. Para esse projeto, o dataset utilizado foi o 
\textit{ERA5 post-processed daily statistics on single levels from 1940 to present}, que contém dados diários de diversas 
variáveis, incluindo a temperatura da superfície do mar (SST).

Considerando que ENSO é um fenômeno definido pela temperatura da superfície do mar em regiões 
específicas do Oceano Pacífico, e que os dados de geração, carga e hidrológicos são disponibilizados em base diária,
optou-se por utilizar a temperatura absoluta da superfície do mar em base diária para cada uma das regiões do ENSO. As
regiões do ENSO e suas coordenadas geográficas são mostradas na Tabela \ref{tab:regioes_enso}.

\begin{table}[htb]
  \centering
   \IBGEtab{
    \caption{Regiões do ENSO}
    \label{tab:regioes_enso}
    }{
    \begin{tabular}{lll}
  		  \hline
    	  \textbf{Região} & \textbf{Latitude} & \textbf{Longitude} \\ \hline
        Niño 1+2 & -10° a 0° & -90° a -80° \\
        Niño 3 & -5° a 5° & -150° a -90° \\
        Niño 3.4 & -5° a 5° & -170° a -120° \\
        Niño 4 & -5° a 5° & -160° a -150° \\ \hline
    \end{tabular}
    }{
      \fonte{\citeonline{TheDefinitionofElNio}}}
\end{table}

Os dados foram obtidos usando a biblioteca \code{cdsapi}, que permite acessar o CDS através da
API do ECMWF. O script desenvolvido para essa etapa realiza o download dos dados de temperatura da superfície do mar
para cada uma das regiões do ENSO, considerando o período de 2000 a 2024. Os dados são obtidos em formato NetCDF, que é 
um formato de arquivo utilizado para armazenar dados científicos multidimensionais.

\subsection{Pré Processamento dos Dados do ENSO} % =================================================================== %
Após o download, os arquivos NetCDF são processados para extrair as informações relevantes para cada região do ENSO. 
Utilizando a biblioteca \code{netCDF4}, cada arquivo anual é lido para extrair as dimensões de tempo, latitude e longitude, 
além da variável de interesse, a temperatura da superfície do mar (SST).

O processamento segue as seguintes etapas:
\begin{itemize}
\item Conversão do formato das coordenadas de longitude, de 0 a 360 graus, para -180 a 180 graus;
\item Para cada uma das regiões do ENSO, um subconjunto geográfico dos dados globais é criado, selecionando os pontos
de latitude e longitude que se encontram dentro dos limites de cada região;
\item A média espacial da variável é calculada para cada dia sobre este subconjunto. Esse processo resulta em uma única série
temporal diária, que representa o valor médio da variável para aquela região específica;
\item Ao final, as séries temporais anuais de cada região são consolidadas, formando um conjunto de dados único que abrange
todo o período de análise, de 2000 a 2024.
\end{itemize}

Em resumo, o processo transforma os dados brutos multidimensionais em séries temporais diárias para cada região
do ENSO, que agora estão prontas para serem utilizadas como variáveis exógenas nos modelos computacionais.


\section{Análise Exploratória dos Dados} % =========================================================================== %
A análise exploratória dos dados é uma etapa crucial para entender as características e padrões dos dados antes de aplicar modelos computacionais.
Essa etapa envolve a visualização e análise estatística, permitindo identificar tendências, sazonalidades, correlações e possíveis outliers. 
Além das bibliotecas \code{pandas} e \code{matplotlib}, a biblioteca \code{statsmodels} foi utilizada para realizar a decomposição
das séries temporais, permitindo avaliar as sazonalidades e tendências presentes nos dados.

Para a decomposição das séries temporais, a função \code{MSTL} foi utilizada, que realiza a decomposição em componentes de 
tendência, sazonalidade e resíduo. Essa função utiliza o método de \textit{local polynomial regression fitting}, ou ajuste
de regressão polinomial local (LOESS) para decompor a série, sendo capaz de capturar sazonalidades de séries temporais
não estacionárias, que é o caso das séries de geração e carga do SIN. Para todas as análises a seguir, os dados estão
em base temporal mensal e foram normalizados utilizando o \code{StandardScaler} da biblioteca \code{scikit-learn}.

\subsection{Dados de Geração e Carga} % ============================================================================== %
\subsubsection{Dados de Geração} % =================================================================================== %
A Figura \ref{fig:decomposicao_Hidráulica} mostra a decomposição da série temporal de geração hidráulica, que revela uma 
tendência de aumento dos anos de 2000 a 2012, seguida por uma período de estabilidade de 2012 a 2024. Ao mesmo tempo,
o regime de sazonalidade apresenta maior amplitude após o ano de 2012, sugerindo uma maior variabilidade da contribuição
hidráulica para a geração do SIN. Esse comportamento pode ser explicado pela introdução em larga escala das fontes renováveis 
variáveis, como a eólica e a solar fotovoltaica, na matriz elétrica. \cite{Silva2016}
\begin{figure}[!ht]
  \IBGEtab{\caption{Decomposição da série temporal de geração hidráulica}
       \label{fig:decomposicao_Hidráulica}}
  {\includesvg[scale=1]{figuras/decomposicao_Hidráulica}}
  {\fonte{o autor.}}
\end{figure}

\begin{figure}[!ht]
  \IBGEtab{\caption{Decomposição da série temporal de geração eólica}
       \label{fig:decomposicao_Eólica}}
  {\includesvg[scale=1]{figuras/decomposicao_Eólica}}
  {\fonte{o autor.}}
\end{figure}

Conforme descrito no capítulo 1, a geração dessas fontes é intermitente e não despachável, ou seja, dependente da 
disponibilidade de vento e sol. Nesse cenário, as usinas hidrelétricas, especialmente aquelas com reservatórios, passaram 
a atuar como a principal ferramenta de flexibilidade do sistema, compensando a variabilidade das fontes intermitentes. 

Em momentos de alta geração eólica e solar, o ONS reduz a produção hidrelétrica para absorver a energia renovável, criando vales 
de geração mais profundos. Inversamente, em períodos de baixa geração renovável, como durante a noite, a geração hidrelétrica 
é acionada para suprir a demanda, resultando em picos mais elevados. Portanto, o aumento da amplitude da sazonalidade após 
2012 é a assinatura visual desse novo papel operacional, no qual a fonte hidráulica não apenas segue o ciclo hidrológico, 
mas também responde dinamicamente à intermitência e situações de pico de demanda para garantir a estabilidade do SIN. \cite{Wang2025}

A Figura \ref{fig:decomposicao_Eólica} mostra a decomposição da série temporal de geração eólica. A tendência
evidencia o crescimento exponencial da capacidade instalada a partir de 2012, reflexo das políticas de incentivo e dos leilões 
de energia. Diferentemente da fonte hidráulica, a sazonalidade da geração eólica é ditada exclusivamente 
pela disponibilidade do recurso, com picos correspondentes à "safra dos ventos", que ocorre tipicamente no segundo semestre. 

A amplitude crescente da sazonalidade e a maior variância do resíduo são consequências diretas da expansão 
da capacidade instalada. Em conjunto, a sazonalidade e o resíduo da geração eólica são as principais causas da mudança no 
perfil operacional da geração hidráulica, que ajusta sua produção para compensar a intermitência da fonte eólica.

\begin{figure}[!ht]
  \IBGEtab{\caption{Decomposição da série temporal de geração térmica}
       \label{fig:decomposicao_Térmica}}
  {\includesvg[scale=1]{figuras/decomposicao_Térmica}}
  {\fonte{o autor.}}
\end{figure}

A Figura \ref{fig:decomposicao_Térmica} apresenta a decomposição da geração térmica. A tendência atua 
como um termômetro do risco hidrológico do país, com picos proeminentes que coincidem diretamente com períodos de crise hídrica, 
como em 2014-2015 e 2021, quando o despacho térmico é massivo para garantir o suprimento. 

A sazonalidade opera de forma inversa ao ciclo hidrológico, com maior geração no período seco para poupar os reservatórios. 
O resíduo, por sua vez, exibe alta volatilidade e representa o papel da fonte térmica como recurso de ponta e de emergência, 
acionada para cobrir a intermitência de outras fontes, atender picos de demanda e responder a indisponibilidades no sistema. 


\subsubsection{Dados de Carga} % ===================================================================================== %
\begin{figure}[!ht]
  \IBGEtab{\caption{Decomposição da série temporal de carga}
       \label{fig:decomposicao_carga}}
  {\includesvg[scale=1]{figuras/decomposicao_carga}}
  {\fonte{o autor.}}
\end{figure}
A Figura \ref{fig:decomposicao_carga} mostra a decomposição da série temporal de carga. A tendência
reflete diretamente a atividade econômica do país, com um crescimento contínuo até 2014, seguido por uma estagnação durante 
a recessão de 2014-2016 e uma queda abrupta em 2020 devido à pandemia de COVID-19. \cite{Magazzino2021}

A sazonalidade é a mais regular entre todas as séries, ditada pelo clima, com picos de consumo consistentes no verão devido 
ao uso de sistemas de refrigeração. O resíduo captura variações de curto prazo, como ondas de temperatura e feriados, e 
possui uma volatilidade relativamente baixa.


\section{Implementação dos Modelos de Regressão} % =================================================================== %
Para todos os casos, os códigos referentes às implementações dos modelos estão disponíveis no Apêncice A. A seguir, será
apresentada uma visão geral das etapas de implementação, métricas de avaliação e os modelos computacionais utilizados.

\subsection{Métricas de Avaliação} % ================================================================================= %
Para todos os casos, duas métricas de avaliação foram utilizadas: o erro quadrático médio ($MSE$) e o coeficiente de determinação ($R^2$).
O $MSE$ é uma medida que quantifica a média dos erros quadráticos entre os valores reais e as previsões do modelo. O $R^2$, por outro lado, 
é uma medida que indica a proporção da variabilidade dos dados que é explicada pelo modelo, variando de 0 a 1, onde valores mais 
próximos de 1 indicam um modelo mais preciso. As equações para o cálculo dessas métricas são apresentadas a seguir:
\begin{equation}
    \label{eq:mse}
    MSE = \frac{1}{n} \sum_{i=1}^{n} (y_i - \hat{y}_i)^2
\end{equation}
\begin{equation}
    \label{eq:r2}
    R^2 = 1 - \frac{\sum_{i=1}^{n} (y_i - \hat{y}_i)^2}{\sum_{i=1}^{n} (y_i - \bar{y})^2}
\end{equation}
em que $ y_i $ são os valores reais, $ \hat{y}_i $ são os valores previstos pelo modelo, $ \bar{y} $ é a média dos valores 
reais e $ n $ é o número de amostras. Essas métricas estão disponíveis na biblioteca \code{scikit-learn} através das
funções \code{mean\_squared\_error} e \code{r2\_score}, respectivamente.

\subsection{Etapas Comuns} % ========================================================================================= %
As etapas iniciais de implementação são comuns a todos os modelos, consistindo em: 
\begin{enumerate}
    \item \textbf{Entrada de dados:} Leitura dos arquivos CSV contendo os dados de geração, carga e variáveis do ENSO;
    \item \textbf{Consolidação do dataset:} Alinhamento temporal dos dados, garantindo que as séries estejam sincronizadas;
    \item \textbf{Definição das variáveis:} Seleção das clunas contendo as variáveis dependentes (geração por fonte e subsistema) e 
independentes (carga e variáveis do ENSO);
    \item \textbf{Normalização:} Normalização das variáveis independentes utilizando o \code{StandardScaler} do \code{scikit-learn};
    \item \textbf{Definição dos conjuntos para treino:} Divisão dos dados em conjuntos de treino, teste e validação (para o modelo neural);
    \item \textbf{Avaliação:} Avaliação através das métricas escolhidas e visual dos resultados através de gráficos de dispersão e linha.
\end{enumerate}

A normalização é importante para evitar que o modelo atribua maior peso às variáveis com maior amplitude,
o que poderia resultar em resultados não representativos. Para a etapa de treinamento, na qual os modelos atualizam seus parâmetros 
para minimizar o erro, os dados são divididos em conjuntos de treino, teste e validação (para o caso do modelo neural). 

O conjunto de treino é utilizado para ajustar os parâmetros 
do modelo, enquanto o conjunto de teste é utilizado para avaliar a performance do modelo em dados não vistos. Para o modelo neural, 
o conjunto de validação é utilizado para monitorar o desempenho do modelo durante o treinamento e evitar o sobreajuste. 
Diferentes tamanhos da janela de treinamento foram avaliados e serão apresentados no capítulo 5.

Os gráficos de dispersão mostram os valores reais e estimados junto a uma linha ideal, de modo que valores próximos à linha indicam 
resultados mais precisos. O outro gráfico demonstra a separação dos períodos de treino e teste, permitindo uma avaliação mais direta 
entre os resultados reais e estimados. Para facilitar a visualização, os gráficos são apresentados em base temporal mensal, 
considerando a média mensal dos valores.

\subsection{Modelo Linear} % ========================================================================================= %
O modelo \code{LinearRegression} da biblioteca \code{scikit-learn} foi implementado com a abordagem de regressão linear múltipla, 
uma técnica estatística que busca modelar a relação entre uma variável dependente e várias variáveis independentes, conforme descrito pela 
equação \ref{eq:regressao_linear}. Nesse contexto, a variável dependente é a geração de uma determinada fonte em um subsistema do 
SIN, e as variáveis independentes são as variáveis auxiliares (ou exógenas), os dados de SST e carga.

Foi considerado apenas o período de 2010 a 2024, já que a fonte eólica era pouco presente
no SIN até então. Considerando as limitações do modelo, utilizar todo o dataset poderia resultar em resultados não representativos, 
uma vez que o modelo assume que a relação entre as variáveis é linear e constante ao longo do tempo. Assim, a escolha do período 
de 2010 a 2024 é justificada pela necessidade de garantir que os dados utilizados sejam representativos do comportamento atual 
do sistema, permitindo uma análise mais precisa e confiável.

Após as etapas comuns, de carregamento de dados e tratamento inicial, os datasets de treino e teste foram definidos, de modo que
o conjunto de treino contenha 70\% dos dados e o conjunto de teste 30\%. O modelo é então instanciado com paralelismo do
CPU habilitado e treinado no conjunto de treino. A seguir, o modelo é avaliado no conjunto de teste, e os resultados são
apresentados através dos gráficos.

O modelo linear permite uma interpretação direta dos resultados, uma vez que é possível obter os coeficientes da equação
linear que descreve a relação entre as variáveis dependentes e independentes. Os demais modelos não permitem essa análise diretamente.

\subsection{Modelo Não Linear} % ===================================================================================== %
O modelo \code{RandomForestRegressor} da biblioteca \code{scikit-learn} foi implementado com a abordagem de previsão multivariada.
O modelo é um dos poucos da biblioteca que permite prever múltiplas variáveis dependentes simultaneamente, sem considerar os modelos
neurais, e foi selecionado por este motivo.

Para este caso, também foi considerado o período de 2010 a 2024, e as etapas comuns foram realizadas. Após a consolidação 
dos dados de geração, carga e ENSO, foram criados atributos temporais adicionais, como ano, mês e dia do ano, para enriquecer 
o conjunto de dados. As variáveis exógenas (carga e ENSO) foram normalizadas com o \code{StandardScaler}. O conjunto de dados 
foi então dividido cronologicamente, com 70\% dos dados para treino e 30\% para teste, utilizando a função \code{train\_test\_split} 
com o parâmetro \code{shuffle = False}.

Para definir os melhores hiperparâmetros do modelo, foi utilizado o \code{GridSearchCV}, que realiza uma busca exaustiva por 
meio de validação cruzada em grade. A validação cruzada foi adaptada para séries temporais com o uso do \code{TimeSeriesSplit}, 
que garante que os dados de treino sempre ocorram antes dos dados de validação em cada divisão. O modelo foi treinado no conjunto 
de treino, e os melhores parâmetros foram selecionados com base na métrica de erro quadrático médio ($MSE$).

Após a aplicação do \code{GridSearchCV}, o modelo foi instanciado com os parâmetros otimizados, com número de árvores de decisão 
\code{n\_estimators = 1000} e profundidade máxima \code{max\_depth = 30}, com o paralelismo habilitado \code{n\_jobs = -1}. O modelo 
treinado foi então utilizado para realizar as previsões no conjunto de teste. Os resultados foram avaliados com as métricas $R^2$ e $MSE$. 
Por fim, para cada variável alvo, foram gerados e salvos gráficos de dispersão e de série temporal para a análise visual dos resultados.


\section{Implementação do Modelo Neural} % =========================================================================== %
O modelo neural utilizado foi o \textit{TinyTimeMixer}, versão 2.1, disponível na biblioteca \code{transformers} do 
\textit{Hugging Face}. Ele é um modelo de previsão de séries temporais pré-treinado baseado na arquitetura \textit{TSMixer}, 
apresentada no capítulo 3. Modelos pré treinados são modelos que já foram treinados em grandes conjuntos de dados. Ou seja, 
os pesos das camadas de \textit{perceptrons} já foram ajustados para capturar padrões gerais em séries temporais.

Além disso, o modelo permite o processo de \textit{fine tuning}, que é a atualização dos pesos para adequar-se
a um conjunto de dados específico, considerando variáveis dependentes e independentes, que no contexto do projeto são a geração de energia
por subsistema e fonte, carga e variáveis do ENSO.

O modelo possui diferentes variações, com diferentes tamanhos de janelas de contexto e previsão e tem suporte para escalas 
de tempo semanal, diária, horária e de minutos. Para o projeto, foram utilizados as variantes com janelas de contexto e 
previsão de 512 e 96 dias, bem como a de 90 e 30 semanas.

A biblioteca \code{tsfm\_public}, que contém funções auxiliares para a implementação dos modelos publicados pela IBM, foi
utilizada para definir os conjuntos de treino, teste e validação, além de realizar o pré processamento dos dados, incluindo o
processo de normalização.

Vale destacar que, embora \citeonline{Ekambaram2024} sugira o congelamento de todo o \textit{backbone} do modelo, ou seja, 
não atualizar todos os pesos das camadas escondidas durante o ajuste fino, de modo a preservar o conhecimento prévio do modelo,
isso não foi seguido para esse projeto, dado que resultados superiores foram obtidos ao permitir que todas as camadas 
fossem atualizadas. Dessa forma, o modelo é capaz de aprender padrões mais complexos e específicos do conjunto de dados, 
o que pode levar a uma melhor performance nas previsões.

\subsection{Sem Variáveis Exógenas} % ================================================================================ %
Inicialmente, foi implementado o modelo sem considerar as variáveis exógenas, ou seja, apenas com as variáveis dependentes, a 
geração de energia por subsistema e fonte. Essa abordagem, conhecida como \textit{one-shot forecasting}, consiste em aplicar 
um modelo já treinado diretamente sobre a série temporal de interesse para realizar a inferência, sem uma nova etapa de treinamento.

Para esta análise, os dados de 2000 a 2024 foram considerados, devido a capacidade do modelo capturar relações mais complexas. 
O conjunto de dados foi então dividido em frações de treino (70\%), validação (10\%) e teste (20\%), utilizando a função 
\code{prepare\_data\_splits}. Diferentes variantes do modelo foram avaliadas para diferentes amostragens temporais, e os resultados 
serão apresentados no capítulo 5.

O pré-processamento foi realizado pela classe \code{TimeSeriesPreprocessor}, configurada para as variáveis de geração como 
alvo \code{target\_columns}. A classe ajusta os dados para o formato esperado pelo modelo, com janelas de contexto definidas
considerando a variante selecionada, além de aplicar a normalização \code{StandardScaler}, cujos parâmetros são aprendidos 
a partir do conjunto de treino.

Os modelos \code{TinyTimeMixerForPrediction} utilizados nesta etapa foram diferentes variantes pré treinadas, carregadas a partir
de suas respectivas identificações do \textit{Hugging Face}. Apenas as camadas de entrada e saída do modelo foram adaptada 
para corresponder ao número de variáveis de geração.

A previsão é executada através de uma \code{TimeSeriesForecastingPipeline}, que aplica o modelo pré-processado sobre o conjunto 
de teste. Como as previsões são geradas em frequência semanal, um pós-processamento é realizado para agregar os resultados em 
médias mensais, permitindo uma comparação direta com os demais modelos. As métricas $R^2$ e $MSE$ são calculadas sobre os dados 
com a mesma base temporal. Por fim, para cada variável, são gerados e salvos gráficos de dispersão e de série temporal para 
análise visual dos resultados.


\subsection{Com Variáveis Exógenas} % ================================================================================ %
Para a implementação do modelo com variáveis exógenas, faz-se necessário realizar o processo de \textit{fine tuning}, que 
é o processo de atualização dos pesos do modelo pré treinado para adequar-se ao conjunto de dados específico, considerando
a presença de variáveis dependentes e independentes.

Após a consolidação dos dados de geração, carga e ENSO, o conjunto de dados é dividido em frações de treino (70\%), validação 
(10\%) e teste (20\%), utilizando a função \code{prepare\_data\_splits}. A biblioteca \code{tsfm\_public} é novamente empregada 
para o pré-processamento através da classe \code{TimeSeriesPreprocessor}, que desta vez é configurada para tratar tanto as variáveis 
alvo \code{target\_columns} quanto as exógenas \code{control\_columns}, aplicando a normalização \code{StandardScaler} em ambas.

O modelo \code{TinyTimeMixerForPrediction} é carregado, e sua arquitetura é ajustada para o novo conjunto de dados, especificando 
os canais de entrada para as variáveis dependentes e independentes. Uma característica central do modelo é a sua capacidade de 
\textit{mixing}, ou mistura de canais, que permite aprender as interdependências entre as múltiplas séries temporais. Isso é 
habilitado pelos parâmetros \code{decoder\_mode = 'mix\_channel'} e \code{enable\_forecast\_channel\_mixing = True}. Adicionalmente, 
o parâmetro \code{fcm\_prepend\_past = True} é utilizado para que os valores passados das séries também sejam considerados no processo 
de mistura, enriquecendo o contexto disponível para a previsão.

Para o treinamento, a taxa de aprendizado \code{learning\_rate} é definida dinamicamente pela função \code{optimal\_lr\_finder}, 
que busca um valor otimizado para a convergência do modelo. O otimizador utilizado foi o \code{AdamW}, uma variante do otimizador 
Adam que desacopla a regularização de decaimento de peso (\textit{weight decay}) da atualização do gradiente.

O treinamento é gerenciado pela classe \code{Trainer} da biblioteca \code{transformers}, que recebe os hiperparâmetros, 
como o número máximo de épocas, definido em 500, e \textit{batch size} (tamanho do lote) \code{batch\_size = 64}. O tamanho
do lote define o número de amostras que são propagadas através da rede antes da atualização dos pesos. A escolha do tamanho
do lote envolve um compromisso entre a estabilidade da convergência e a capacidade de generalização do modelo. Em tese, 
lotes maiores podem resultar em um treinamento mais rápido, mas podem fazer com que o modelo convirja para um mínimo local, 
enquanto lotes menores podem levar a uma convergência lenta, mas com maior chance de encontrar um mínimo global.

Adicionalmente, o \code{OneCycleLR}, é empregado para variar a taxa de aprendizado de forma cíclica durante o treinamento, 
começando com um valor baixo, aumentando até um máximo e depois diminuindo novamente, o que pode acelerar a convergência.

Uma estratégia de parada antecipada \code{EarlyStoppingCallback} é implementada para monitorar a perda no conjunto de validação 
e interromper o treinamento caso não haja melhora significativa de ao menos 0,001 por 50 épocas consecutivas, 
evitando o sobreajuste (\textit{overfitting}). O modelo com o menor erro de validação é salvo ao final do processo.

% Resultados
\chapter{Resultados}
Este capítulo apresenta os resultados obtidos na estimação da geração das fontes hidráulica, eólica e térmica. A análise 
se inicia com modelos de regressão linear e não linear (Random Forest) para estabelecer um baseline de performance e 
avaliar a linearidade das relações (Seções 5.1 e 5.2). Em seguida, na Seção 5.3, avalia-se o desempenho do modelo 
neural \textit{Tiny Time Mixer} (TTM) em sua aplicação direta. Por fim, na Seção 5.3.2, demonstra-se o impacto e a 
eficácia do ajuste fino (\textit{finetuning}) deste modelo com a inclusão de variáveis do fenômeno ENSO, quantificando o ganho de 
acurácia obtido com a especialização do modelo. Serão apresentados os gráficos de geração total das fontes. Demais gráficos 
dos resultados para os subsistemas estão disponíveis no Apêndice A.

\section{Modelo Linear}
Nesta seção, são apresentados e analisados os resultados obtidos com a aplicação do modelo de regressão linear. A avaliação busca 
estabelecer um desempenho base (baseline) e verificar a capacidade do modelo em capturar as relações entre a geração de energia e 
as variáveis climáticas do ENSO e de carga. A análise é dividida por fonte, começando pela hidráulica e, em seguida, pelas fontes 
eólica e térmica.
\subsection{Fonte Hidráulica}
A Figura \ref{fig:linear_hidraulica} apresenta os resultados da regressão linear para a fonte hidráulica. Para esse caso,
foram consdierados os dados de 2010 em diante. O conjunto de treino foi constituído pelos dados de 2010 a 2021, e o 
conjunto de teste pelos dados restantes. 

\begin{figure}[!ht]
  \IBGEtab{\caption{Regressão linear para fonte hidráulica (carga + ENSO) em base diária}
       \label{fig:linear_hidraulica}}
  {\includesvg[scale=.9]{figuras/linear/linear_Hidráulica}}
  {\fonte{o autor.}}
\end{figure}
Fica evidente que, para a geração hidráulica total, o modelo é capaz de estimar os valores de geração do conjunto de teste
com um valor de R² de 0,589, o que indica uma correlação moderada entre os valores estimados e os valores reais, considerando
as limitações do modelo linear. 

Para verificar a utilidade dos dados do ENSO, uma nova regressão foi realizada, dessa vez
considerando apenas a carga como variável independente, conforme a Figura \ref{fig:linear_hidraulica_carga}. Observa-se 
uma queda no valor de R² de 0,589 para 0,522, uma redução de 12,06\% na capacidade explicativa do modelo, indicando que a 
inclusão do ENSO melhora significativamente a predição. Além disso, o resultado é compatível com a fundamentação teórica 
apresentada no Capítulo 3, na qual a influência do ENSO na geração hidráulica é direta e significativa.

\begin{figure}[!ht]
  \IBGEtab{\caption{Regressão linear para fonte hidráulica (apenas carga) em base diária}
       \label{fig:linear_hidraulica_carga}}
  {\includesvg[scale=.9]{figuras/linear/linear_Hidráulica_CARGA}}
  {\fonte{o autor.}}
\end{figure}

Embora o modelo linear atinja um R² de 0,589, o gráfico de dispersão (à esquerda) mostra que os pontos se 
afastam da linha ideal especialmente nos extremos. Nota-se que o modelo tende a subestimar os picos de geração 
(pontos acima da linha) e a superestimar os vales (pontos abaixo da linha), uma limitação característica da regressão 
linear ao tentar capturar a amplitude total de uma série com forte sazonalidade.

A fim de melhor avaliar os resultados, os demais gráficos serão apresentados em base temporal mensal, indicando a média 
dos valores de geração para cada mês em MWmed, considerando os valores médios das variáveis independentes para o mesmo período. 
A Figura \ref{fig:linear_hidraulica_mensal} apresenta os resultados mensais da regressão linear para a fonte hidráulica, 
considerando a carga e o ENSO. 

\begin{figure}[!ht]
  \IBGEtab{\caption{Regressão linear para fonte hidráulica (carga + ENSO) em base mensal}
       \label{fig:linear_hidraulica_mensal}}
  {\includesvg[scale=.9]{figuras/linear/linear_Hidráulica_mensal}}
  {\fonte{o autor.}}
\end{figure}
\begin{figure}[!ht]
  \IBGEtab{\caption{Regressão linear para fonte hidráulica (apenas carga) em base mensal}
       \label{fig:linear_hidraulica_mensal_carga}}
  {\includesvg[scale=.9]{figuras/linear/linear_Hidráulica_mensal_CARGA}}
  {\fonte{o autor.}}
\end{figure}

Novamente, é realizado uma avaliação do modelo considerando apenas a carga, conforme a Figura 
\ref{fig:linear_hidraulica_mensal_carga}. Observa-se que, para a regressão mensal, a queda na performance do modelo é ainda
maior: o valor de R² cai de 0,515 para 0,417. Isso pode indicar que a inclusão de variáveis do ENSO é ainda mais significante 
para a base temporal mensal.

\subsection{Fonte Eólica e Térmica}
As Figuras \ref{fig:linear_eolica_mensal} e \ref{fig:linear_termica_mensal} apresentam os resultados para as fontes eólica 
e térmica, nos quais pode-se observar que o modelo não foi capaz de apresentar resultados satisfatórios, sugerindo
que a relação entre as variáveis independentes e dependentes pode ser não linear, validando a hipótese levantada nos capítulos
anteriores.

\begin{figure}[!ht]
  \IBGEtab{\caption{Regressão linear para fonte eólica (carga + ENSO) em base mensal}
       \label{fig:linear_eolica_mensal}}
  {\includesvg[scale=.9]{figuras/linear/linear_Eólica_mensal}}
  {\fonte{o autor.}}
\end{figure}
\begin{figure}[!ht]
  \IBGEtab{\caption{Regressão linear para fonte térmica (carga + ENSO) em base mensal}
       \label{fig:linear_termica_mensal}}
  {\includesvg[scale=.9]{figuras/linear/linear_Térmica_mensal}}
  {\fonte{o autor.}}
\end{figure}

A regressão para a fonte eólica resultou em um valor de R² de -0,005, o que indica que o modelo não é capaz de explicar a 
variabilidade dos dados, sendo similar a uma média simples dos dados; enquanto que para a fonte térmica, o valor de R² foi negativo, 
indicando que o modelo é pior do que uma média simples dos dados. Esses resultados corroboram a hipótese de que a relação entre as 
variáveis independentes e dependentes é não linear, o que justifica a escolha de modelos mais complexos para essas fontes.


\section{Modelo Não Linear (Random Forest Regressor)}
A seguir, são explorados os resultados do modelo não linear Random Forest. O objetivo desta seção é avaliar se um modelo com maior 
capacidade de capturar relações complexas e não lineares oferece uma melhoria de performance em relação à abordagem linear. A estrutura 
da análise permanece a mesma, focando sequencialmente nas fontes hidráulica, eólica e térmica.
\subsection{Fonte Hidráulica}
\begin{figure}[!ht]
  \IBGEtab{\caption{Random Forest para fonte hidráulica (carga + ENSO)}
       \label{fig:rf_hidraulica}}
  {\includesvg[scale=.9]{figuras/nlinear/rf_Hidráulica}}
  {\fonte{o autor.}}
\end{figure}
A Figura \ref{fig:rf_hidraulica} apresenta os resultados do modelo não linear para a fonte hidráulica, considerando os dados 
de carga e ENSO como variáveis independentes. O dataset utilizado contém dados a partir de 2010. Os resultados são
ligeiramente superiores aos do modelo de regressão linear, com um valor de R² de 0,521.

\subsection{Fonte Eólica e Térmica}
\begin{figure}[!ht]
  \IBGEtab{\caption{Random Forest para fonte eólica (carga + ENSO)}
       \label{fig:rf_eolica}}
  {\includesvg[scale=.9]{figuras/nlinear/rf_Eólica}}
  {\fonte{o autor.}}
\end{figure}
\begin{figure}[!ht]
  \IBGEtab{\caption{Random Forest para fonte térmica (carga + ENSO)}
       \label{fig:rf_termica}}
  {\includesvg[scale=.9]{figuras/nlinear/rf_Térmica}}
  {\fonte{o autor.}}
\end{figure}

As Figuras \ref{fig:rf_eolica} e \ref{fig:rf_termica} apresentam os resultados do modelo não linear para as fontes eólica 
e térmica, respectivamente. De maneira similar ao modelo linear, o modelo não linear também não é capaz de produzir resultados
satisfatórios para essas fontes.

Ainda que o modelo Random Forest apresente uma capacidade superior para modelar não-linearidades, os resultados para as 
fontes eólica e térmica permaneceram insatisfatórios. Essa dificuldade sugere que as relações temporais são complexas e 
podem ser mais bem capturadas por modelos como o avaliado a seguir.

\section{Modelo Neural TTM}
\subsection{Modelo Pré-treinado}
Ao fazer a implementação \textit{oneshot}, todos os pesos do \textit{backbone}, as camadas escondidas, permanecem inalterados
em relação ao modelo pré treinado, e apenas a \textit{head}, ou cabeça de previsão, é ajustada para gerar a saída desejada,
considerando o número de variáveis dependentes.

A seguir serão apresentados os resultados da previsão \textit{oneshot} para cada fonte de geração, considerando os dados de carga e ENSO
como variáveis exógenas. Para cada fonte, serão apresentados os resultados considerando as janelas de contexto e previsão
de 90/30 semanas e 512/96 dias. Demais gráficos dos resultados para os subsistemas estão disponíveis no Apêndice B.

\subsubsection{Fonte Hidráulica}
\begin{figure}[!ht]
  \IBGEtab{\caption{Oneshot para fonte hidráulica (contexto/previsão: 90/30 semanas)}
       \label{fig:oneshot_hidraulica_w90-30}}
  {\includesvg[scale=.9]{figuras/oneshot/os_Hidráulica_W90-30}}
  {\fonte{o autor.}}
\end{figure}
\begin{figure}[!ht]
  \IBGEtab{\caption{Oneshot para fonte hidráulica (contexto/previsão: 512/96 dias)}
       \label{fig:oneshot_hidraulica_d512-96}}
  {\includesvg[scale=.9]{figuras/oneshot/os_Hidráulica_D512-96}}
  {\fonte{o autor.}}
\end{figure}
As Figuras \ref{fig:oneshot_hidraulica_w90-30} e \ref{fig:oneshot_hidraulica_d512-96} apresentam os resultados da previsão
\textit{oneshot} para a fonte hidráulica, considerando as janelas de contexto e previsão de 90/30 semanas e 512/96 dias, respectivamente.
Observa-se que as métricas de avaliação foram superiores para o caso de 512/96 dias. Isso pode ser explicado devido à
taxa de amostragem diária, que permite uma maior quantidade de dados para o treinamento do modelo e maior número de 
janelas de contexto e previsão.

Para o caso de 90/30 semanas, o valor de R² foi de 0,573, enquanto que para o caso de 512/96 dias, o valor de R² foi de 0,581.
Esses valores são razoáveis, considerando a implementação \textit{oneshot}, sem ajuste fino para o dataset. Além disso, os
resultados são comparáveis aos resultados do modelo linear, o que sugere que a relação da geração hidráulica com as variáveis 
consideradas pode ser mais linear do que as demais fontes de geração.

\subsubsection{Fonte Eólica}
\begin{figure}[!ht]
  \IBGEtab{\caption{Oneshot para fonte eólica (contexto/previsão: 90/30 semanas)}
       \label{fig:oneshot_eolica_w90-30}}
  {\includesvg[scale=.9]{figuras/oneshot/os_Eólica_W90-30}}
  {\fonte{o autor.}}
\end{figure}
\begin{figure}[!ht]
  \IBGEtab{\caption{Oneshot para fonte eólica (contexto/previsão: 512/96 dias)}
       \label{fig:oneshot_eolica_d512-96}}
  {\includesvg[scale=.9]{figuras/oneshot/os_Eólica_D512-96}}
  {\fonte{o autor.}}
\end{figure}
As Figuras \ref{fig:oneshot_eolica_w90-30} e \ref{fig:oneshot_eolica_d512-96} apresentam os resultados da previsão
\textit{oneshot} para a fonte eólica, considerando as janelas de contexto e previsão de 90/30 semanas e 512/96 dias, respectivamente.
Observa-se que, para o caso de 90/30 semanas, o valor de R² foi de 0,683, enquanto que para o caso de 512/96 dias, o valor de R² foi de 0,826.

Esses valores são superiores aos resultados da fonte hidráulica, sugerindo que o modelo consegue capturar melhor a relação
entre as variáveis. Além disso, as métricas de avaliação foram superiores em comparação com os modelos
linear e não linear, sugerindo que, de fato, a relação entre as variáveis é extremamente não linear.

\subsubsection{Fonte Térmica}
\begin{figure}[!ht]
  \IBGEtab{\caption{Oneshot para fonte térmica (contexto/previsão: 90/30 semanas)}
       \label{fig:oneshot_termica_w90-30}}
  {\includesvg[scale=.9]{figuras/oneshot/os_Térmica_W90-30}}
  {\fonte{o autor.}}
\end{figure}
\begin{figure}[!ht]
  \IBGEtab{\caption{Oneshot para fonte térmica (contexto/previsão: 512/96 dias)}
       \label{fig:oneshot_termica_d512-96}}
  {\includesvg[scale=.9]{figuras/oneshot/os_Térmica_D512-96}}
  {\fonte{o autor.}}
\end{figure}
As Figuras \ref{fig:oneshot_termica_w90-30} e \ref{fig:oneshot_termica_d512-96} apresentam os resultados da previsão
\textit{oneshot} para a fonte térmica, considerando as janelas de contexto e previsão de 90/30 semanas e 512/96 dias, respectivamente.
A geração térmica apresentou o pior resultado entre as fontes, com um valor de R² de 0,183 para o caso de 90/30 semanas
e 0,531 para o caso de 512/96 dias. Ainda assim, os valores são superiores em comparação com os os
modelos linear e não linear.

\subsection{Impacto e Análise do Ajuste Fino}
\subsubsection{Fonte Hidráulica}
\begin{figure}[!ht]
  \IBGEtab{\caption{Finetune para fonte hidráulica (contexto/previsão: 90/30 semanas)}
       \label{fig:finetune_hidraulica_w90-30}}
  {\includesvg[scale=.9]{figuras/finetune/fn_Hidráulica_W90-30}}
  {\fonte{o autor.}}
\end{figure}
\begin{figure}[!ht]
  \IBGEtab{\caption{Finetune para fonte hidráulica (contexto/previsão: 512/96 dias)}
       \label{fig:finetune_hidraulica_d512-96}}
  {\includesvg[scale=.9]{figuras/finetune/fn_Hidráulica_D512-96}}
  {\fonte{o autor.}}
\end{figure}
As Figuras \ref{fig:finetune_hidraulica_w90-30} e \ref{fig:finetune_hidraulica_d512-96} apresentam os resultados do ajuste fino
para a fonte hidráulica, considerando as janelas de contexto e previsão de 90/30 semanas e 512/96 dias, respectivamente.

Observa-se que o ajuste fino melhorou em 4,53\% o valor de R² para o caso de 90/30 semanas, passando de 0,573 para 0,599.
Para o caso de 512/96 dias, o ajuste fino melhorou em 15,49\% o valor de R², passando de 0,581 para 0,671. Os resultados 
indicam que o processo de ajuste dos pesos do modelo é eficaz em melhorar a performance do modelo.

\subsubsection{Fonte Eólica}
\begin{figure}[!ht]
  \IBGEtab{\caption{Finetune para fonte eólica (contexto/previsão: 90/30 semanas)}
       \label{fig:finetune_eolica_w90-30}}
  {\includesvg[scale=.9]{figuras/finetune/fn_Eólica_W90-30}}
  {\fonte{o autor.}}
\end{figure}
\begin{figure}[!ht]
  \IBGEtab{\caption{Finetune para fonte eólica (contexto/previsão: 512/96 dias)}
       \label{fig:finetune_eolica_d512-96}}
  {\includesvg[scale=.9]{figuras/finetune/fn_Eólica_D512-96}}
  {\fonte{o autor.}}
\end{figure}
As Figuras \ref{fig:finetune_eolica_w90-30} e \ref{fig:finetune_eolica_d512-96} apresentam os resultados do ajuste fino
para a fonte eólica, considerando as janelas de contexto e previsão de 90/30 semanas e 512/96 dias, respectivamente.

Observa-se que o ajuste fino melhorou em 19,76\% o valor de R² para o caso de 90/30 semanas, passando de 0,683 para 0,818.
Para o caso de 512/96 dias, o ajuste fino melhorou em 1,09\% o valor de R², passando de 0,826 para 0,835. Os resultados
também indicam que o processo de ajuste dos pesos foi eficaz para a fonte eólica, em especial para o caso de 90/30 semanas.

\subsubsection{Fonte Térmica}
\begin{figure}[!ht]
  \IBGEtab{\caption{Finetune para fonte térmica (contexto/previsão: 90/30 semanas)}
       \label{fig:finetune_termica_w90-30}}
  {\includesvg[scale=.9]{figuras/finetune/fn_Térmica_W90-30}}
  {\fonte{o autor.}}
\end{figure}
\begin{figure}[!ht]
  \IBGEtab{\caption{Finetune para fonte térmica (contexto/previsão: 512/96 dias)}
       \label{fig:finetune_termica_d512-96}}
  {\includesvg[scale=.9]{figuras/finetune/fn_Térmica_D512-96}}
  {\fonte{o autor.}}
\end{figure}
As Figuras \ref{fig:finetune_termica_w90-30} e \ref{fig:finetune_termica_d512-96} apresentam os resultados do ajuste fino
para a fonte térmica, considerando as janelas de contexto e previsão de 90/30 semanas e 512/96 dias, respectivamente.

A fonte térmica apresentou melhoria de 37,15\% no valor de R² para o caso de 90/30 semanas, passando de 0,183 para 0,251.
Para o caso de 512/96 dias, o ajuste fino não apresentou melhoria significativa. 

\subsection{Análise Geral}
Os resultados demonstram uma clara hierarquia de performance entre os modelos testados. Foi evidenciado que modelos 
lineares e não-lineares tradicionais são insuficientes para capturar a complexa dinâmica das gerações eólica e térmica, sendo 
amplamente superados pelo modelo TTM. A Tabela \ref{tab:performance_modelos} resume as métricas de performance dos modelos.

\begin{table}[htb]
  \centering
    \IBGEtab{
      \caption{Métricas de performance dos modelos}
      \label{tab:performance_modelos}
      }{
        \begin{tabular}{lcccccc}
          \toprule
          \multirow{2}{*}{\textbf{Modelo}} & \multicolumn{2}{c}{\textbf{Hidráulica}} & \multicolumn{2}{c}{\textbf{Eólica}} & \multicolumn{2}{c}{\textbf{Térmica}} \\
          \cmidrule(lr){2-3} \cmidrule(lr){4-5} \cmidrule(lr){6-7}
           & \textbf{R² Ajust.} & \textbf{MSE} & \textbf{R² Ajust.} & \textbf{MSE} & \textbf{R² Ajust.} & \textbf{MSE} \\ \midrule
          Linear & 0,515 & 2,22$\times$ 10$^7$ & -0,005 & 7,78$\times$ 10$^6$ & -2,352 & 4,84$\times$ 10$^7$ \\
          % \addlinespace
          Não Linear & 0,521 & 2,19$\times$ 10$^7$ & -2,349 & 2,59$\times$ 10$^7$ & -0,078 & 1,56$\times$ 10$^7$ \\
          % \addlinespace
          Neural Oneshot & 0,581 & 1,87$\times$ 10$^7$ & 0,826 & 1,69$\times$ 10$^6$ & 0,531 & 6,63$\times$ 10$^6$ \\
          % \addlinespace
          Neural Finetune & 0,671 & 1,47$\times$ 10$^7$ & 0,835 & 1,61$\times$ 10$^6$ & 0,535 & 6,35$\times$ 10$^6$ \\ \bottomrule
        \end{tabular}
      }{
        \fonte{o autor.}}
\end{table}

Para as Tabelas \ref{tab:matriz_mse_hidraulica_correta}, \ref{tab:matriz_mse_eolica_corrigida} e \ref{tab:matriz_mse_termica_corrigida},
são apresentadas as diferenças percentuais de MSE entre os modelos. As tabelas indicam a diferença percentual do valor de
MSE do modelo na coluna para o modelo na linha, de acordo com a equação:

\begin{equation}
\Delta_\% = \frac{\text{MSE}_{linha} - \text{MSE}_{coluna}}{\text{MSE}_{coluna}}
\end{equation}
em que $\Delta_\%$ é a diferença percentual, $\text{MSE}_{linha}$ é o valor de MSE do modelo na linha e $\text{MSE}_{coluna}$ 
é o valor de MSE do modelo na coluna.

\begin{table}[htb]
  \centering
  \IBGEtab{
    \caption{Variação de MSE para a fonte hidráulica}
    \label{tab:matriz_mse_hidraulica_correta}
  }{
    \begin{tabular}{lcccc}
      \toprule
      \textbf{Modelo} & \textbf{Linear} & \textbf{Não Linear} & \textbf{Neural Oneshot} & \textbf{Neural Finetune} \\
      \midrule
      Linear & - & +1,37\% & +18,72\% & +51,02\% \\
      Não Linear & --1,35\% & - & +17,11\% & +48,98\% \\
      Neural Oneshot & --15,77\% & --14,61\% & - & +27,21\% \\
      Neural Finetune & --33,78\% & --32,88\% & --21,39\% & - \\
      \bottomrule
    \end{tabular}
  }{\fonte{o autor.}}
\end{table}

Evidencia-se que as menores diferenças percentuais observadas entre os modelos foram para a fonte hidráulica. Naturalmente,
isso é explicado pela natureza da fonte e pela relação entre as variáveis do ENSO com a geração hidráulica, que é diretamente
influenciada pelo fenômeno, conforme discutido no Capítulo 3.

\begin{table}[htb]
  \centering
  \IBGEtab{
    \caption{Variação de MSE para a fonte eólica}
    \label{tab:matriz_mse_eolica_corrigida}
  }{
    \begin{tabular}{lcccc}
      \toprule
      \textbf{Modelo} & \textbf{Linear} & \textbf{Não Linear} & \textbf{Neural Oneshot} & \textbf{Neural Finetune} \\
      \midrule
      Linear & - & --69,96\% & +360,36\% & +383,23\% \\
      Não Linear & +232,91\% & - & +1432,54\% & +1508,7\% \\
      Neural Oneshot & --78,28\% & --93,47\% & - & +4,97\% \\
      Neural Finetune & --79,31\% & --93,78\% & --4,73\% & - \\
      \bottomrule
    \end{tabular}
  }{\fonte{o autor.}}
\end{table}

A fonte eólica apresenta a maior variação percentual de MSE entre os modelos. O modelo não linear, por exemplo, apresenta
um valor de MSE 1508,7\% maior que o modelo neural ajustado. A fonte térmica apresenta um meio termo entre as demais fontes.
Em tese, caso fossem consideradas variáveis como velocidade do vento, temperatura e PLD, essas fontes poderiam apresentar resultados
mais satisfatórios. 

A velocidade do vento está diretamente associada à geração eólica, enquanto que a temperatura e o PLD poderiam atuar como
uma \textit{proxy} para a demanda de geração térmica, uma vez que a temperatura influencia o consumo de energia e o PLD poderia
representar in diretamente a disponibilidade de recursos hídricos.

\begin{table}[htb]
  \centering
  \IBGEtab{
    \caption{Variação de MSE para a fonte térmica}
    \label{tab:matriz_mse_termica_corrigida}
  }{
    \begin{tabular}{lcccc}
      \toprule
      \textbf{Modelo} & \textbf{Linear} & \textbf{Não Linear} & \textbf{Neural Oneshot} & \textbf{Neural Finetune} \\
      \midrule
      Linear & - & +210,26\% & +630,02\% & +662,2\% \\
      Não Linear & --67,77\% & - & +135,29\% & +145,67\% \\
      Neural Oneshot & --86,3\% & --57,5\% & - & +4,41\% \\
      Neural Finetune & --86,88\% & --59,29\% & --4,22\% & - \\
      \bottomrule
    \end{tabular}
  }{\fonte{o autor.}}
\end{table}

% Conclusão
\chapter{Conclusão}
Este trabalho teve como objetivo central investigar e quantificar o impacto de variáveis climáticas externas, associadas ao 
fenômeno ENSO, na estimação da geração de energia das fontes hidráulica, térmica e eólica no SIN. 
Para tal, foram implementados e comparados modelos computacionais de complexidade crescente, desde regressões lineares e 
não-lineares até uma arquitetura de rede neural pré-treinada, avaliando o ganho de performance ao especializar o modelo com 
os dados do ENSO.

Os resultados obtidos permitiram extrair conclusões importantes. Foi demonstrado que modelos de regressão tradicionais, tanto 
lineares quanto o Random Forest, são insuficientes para capturar a dinâmica complexa das fontes eólica e térmica, apresentando 
valores de R² insatisfatórios e até negativos, o que valida a hipótese de não-linearidade das relações envolvidas. A introdução 
do modelo neural pré-treinado (TinyTimeMixer) representou um salto qualitativo significativo, especialmente para a fonte eólica, 
que atingiu um R² de 0,826, evidenciando o poder de generalização dessa arquitetura.

Foram observados ganhos de performance substanciais ao realizar o ajuste fino do modelo neural, o que responde afirmativamente 
à questão central da pesquisa: a incorporação de variáveis climáticas externas, aliada a uma arquitetura neural adequada, 
aprimora de forma mensurável e significativa a estimação da geração energética.

Reconhece-se, contudo, as limitações deste estudo. A análise se restringiu às variáveis do fenômeno ENSO, enquanto outras 
variáveis climáticas e indicadores econômicos poderiam ser incorporados para enriquecer os modelos. Adicionalmente, foi 
utilizada uma arquitetura neural específica (TSMixer), e a exploração de outras arquiteturas, como as baseadas em Transformers 
ou LSTMs, poderia trazer resultados distintos.

Em trabalhos futuros, será feita a expansão do conjunto de variáveis exógenas, incluindo outros índices 
climáticos relevantes para o território brasileiro e indicadores macroeconômicos. Além disso, a aplicação da metodologia de 
fine-tuning de modelos pré-treinados a outras tarefas, como a previsão de carga ou de preços de energia (PLD), poderá ser estudada.

% Elementos pós-textuais
% ----------------------------------------------------------------------------------------------------------------------
\postextual

% Referências
\bibliographystyle{abntex2-alf}
\bibliography{referencias}

% Apêndices
\begin{apendicesenv}
\chapter*{Apêndice A: Resultados por Subsistema}
\section*{Modelo Linear}

\begin{figure}[H]
  \IBGEtab{\caption*{Regressão linear para fonte hidráulica -- subsistema Sul}
       \label{fig:linear_hidraulica_s}}
  {\includesvg[scale=.9]{figuras/linear/linear_Hidráulica_S_mensal}}
  {\fonte{o autor.}}
\end{figure}
\begin{figure}[H]
  \IBGEtab{\caption*{Regressão linear para fonte hidráulica -- subsistema Sudeste}
       \label{fig:linear_hidraulica_se}}
  {\includesvg[scale=.9]{figuras/linear/linear_Hidráulica_SE_mensal}}
  {\fonte{o autor.}}
\end{figure}
\begin{figure}[H]
  \IBGEtab{\caption*{Regressão linear para fonte hidráulica -- subsistema Nordeste}
       \label{fig:linear_hidraulica_ne}}
  {\includesvg[scale=.9]{figuras/linear/linear_Hidráulica_NE_mensal}}
  {\fonte{o autor.}}
\end{figure}
\begin{figure}[H]
  \IBGEtab{\caption*{Regressão linear para fonte hidráulica -- subsistema Norte}
       \label{fig:linear_hidraulica_n}}
  {\includesvg[scale=.9]{figuras/linear/linear_Hidráulica_N_mensal}}
  {\fonte{o autor.}}
\end{figure}

\begin{figure}[H]
  \IBGEtab{\caption*{Regressão linear para fonte térmica -- subsistema Sul}
       \label{fig:linear_termica_s}}
  {\includesvg[scale=.9]{figuras/linear/linear_Térmica_S_mensal}}
  {\fonte{o autor.}}
\end{figure}
\begin{figure}[H]
  \IBGEtab{\caption*{Regressão linear para fonte térmica -- subsistema Sudeste}
       \label{fig:linear_termica_se}}
  {\includesvg[scale=.9]{figuras/linear/linear_Térmica_SE_mensal}}
  {\fonte{o autor.}}
\end{figure}
\begin{figure}[H]
  \IBGEtab{\caption*{Regressão linear para fonte térmica -- subsistema Nordeste}
       \label{fig:linear_termica_ne}}
  {\includesvg[scale=.9]{figuras/linear/linear_Térmica_NE_mensal}}
  {\fonte{o autor.}}
\end{figure}
\begin{figure}[H]
  \IBGEtab{\caption*{Regressão linear para fonte térmica -- subsistema Norte}
       \label{fig:linear_termica_n}}
  {\includesvg[scale=.9]{figuras/linear/linear_Térmica_N_mensal}}
  {\fonte{o autor.}}
\end{figure}
\begin{figure}[H]
  \IBGEtab{\caption*{Regressão linear para fonte eólica -- subsistema Sul}
       \label{fig:linear_eolica_s}}
  {\includesvg[scale=.9]{figuras/linear/linear_Eólica_S_mensal}}
  {\fonte{o autor.}}
\end{figure}
\begin{figure}[H]
  \IBGEtab{\caption*{Regressão linear para fonte eólica -- subsistema Sudeste}
       \label{fig:linear_eolica_se}}
  {\includesvg[scale=.9]{figuras/linear/linear_Eólica_SE_mensal}}
  {\fonte{o autor.}}
\end{figure}
\begin{figure}[H]
  \IBGEtab{\caption*{Regressão linear para fonte eólica -- subsistema Nordeste}
       \label{fig:linear_eolica_ne}}
  {\includesvg[scale=.9]{figuras/linear/linear_Eólica_NE_mensal}}
  {\fonte{o autor.}}
\end{figure}
\begin{figure}[H]
  \IBGEtab{\caption*{Regressão linear para fonte eólica -- subsistema Norte}
       \label{fig:linear_eolica_n}}
  {\includesvg[scale=.9]{figuras/linear/linear_Eólica_N_mensal}}
  {\fonte{o autor.}}
\end{figure}

\section*{Modelo Random Forest}

\begin{figure}[H]
  \IBGEtab{\caption*{Random Forest para fonte hidráulica -- subsistema Sul}
       \label{fig:rf_hidraulica_s}}
  {\includesvg[scale=.9]{figuras/nlinear/rf_Hidráulica_S}}
  {\fonte{o autor.}}
\end{figure}
\begin{figure}[H]
  \IBGEtab{\caption*{Random Forest para fonte hidráulica -- subsistema Sudeste}
       \label{fig:rf_hidraulica_se}}
  {\includesvg[scale=.9]{figuras/nlinear/rf_Hidráulica_SE}}
  {\fonte{o autor.}}
\end{figure}
\begin{figure}[H]
  \IBGEtab{\caption*{Random Forest para fonte hidráulica -- subsistema Nordeste}
       \label{fig:rf_hidraulica_ne}}
  {\includesvg[scale=.9]{figuras/nlinear/rf_Hidráulica_NE}}
  {\fonte{o autor.}}
\end{figure}
\begin{figure}[H]
  \IBGEtab{\caption*{Random Forest para fonte hidráulica -- subsistema Norte}
       \label{fig:rf_hidraulica_n}}
  {\includesvg[scale=.9]{figuras/nlinear/rf_Hidráulica_N}}
  {\fonte{o autor.}}
\end{figure}
\begin{figure}[H]
  \IBGEtab{\caption*{Random Forest para fonte térmica -- subsistema Sul}
       \label{fig:rf_termica_s}}
  {\includesvg[scale=.9]{figuras/nlinear/rf_Térmica_S}}
  {\fonte{o autor.}}
\end{figure}
\begin{figure}[H]
  \IBGEtab{\caption*{Random Forest para fonte térmica -- subsistema Sudeste}
       \label{fig:rf_termica_se}}
  {\includesvg[scale=.9]{figuras/nlinear/rf_Térmica_SE}}
  {\fonte{o autor.}}
\end{figure}
\begin{figure}[H]
  \IBGEtab{\caption*{Random Forest para fonte térmica -- subsistema Nordeste}
       \label{fig:rf_termica_ne}}
  {\includesvg[scale=.9]{figuras/nlinear/rf_Térmica_NE}}
  {\fonte{o autor.}}
\end{figure}
\begin{figure}[H]
  \IBGEtab{\caption*{Random Forest para fonte térmica -- subsistema Norte}
       \label{fig:rf_termica_n}}
  {\includesvg[scale=.9]{figuras/nlinear/rf_Térmica_N}}
  {\fonte{o autor.}}
\end{figure}
\begin{figure}[H]
  \IBGEtab{\caption*{Random Forest para fonte eólica -- subsistema Sul}
       \label{fig:rf_eolica_s}}
  {\includesvg[scale=.9]{figuras/nlinear/rf_Eólica_S}}
  {\fonte{o autor.}}
\end{figure}
\begin{figure}[H]
  \IBGEtab{\caption*{Random Forest para fonte eólica -- subsistema Sudeste}
       \label{fig:rf_eolica_se}}
  {\includesvg[scale=.9]{figuras/nlinear/rf_Eólica_SE}}
  {\fonte{o autor.}}
\end{figure}
\begin{figure}[H]
  \IBGEtab{\caption*{Random Forest para fonte eólica -- subsistema Nordeste}
       \label{fig:rf_eolica_ne}}
  {\includesvg[scale=.9]{figuras/nlinear/rf_Eólica_NE}}
  {\fonte{o autor.}}
\end{figure}
\begin{figure}[H]
  \IBGEtab{\caption*{Random Forest para fonte eólica -- subsistema Norte}
       \label{fig:rf_eolica_n}}
  {\includesvg[scale=.9]{figuras/nlinear/rf_Eólica_N}}
  {\fonte{o autor.}}
\end{figure}

\section*{Modelo Neural TTM - Oneshot}

\begin{figure}[H]
  \IBGEtab{\caption*{TTM para fonte hidráulica -- subsistema Sul}
       \label{fig:ttm_hidraulica_s}}
  {\includesvg[scale=.9]{figuras/oneshot/os_Hidráulica_S_D512-96}}
  {\fonte{o autor.}}
\end{figure}
\begin{figure}[H]
  \IBGEtab{\caption*{TTM para fonte hidráulica -- subsistema Sudeste}
       \label{fig:ttm_hidraulica_se}}
  {\includesvg[scale=.9]{figuras/oneshot/os_Hidráulica_SE_D512-96}}
  {\fonte{o autor.}}
\end{figure}
\begin{figure}[H]
  \IBGEtab{\caption*{TTM para fonte hidráulica -- subsistema Nordeste}
       \label{fig:ttm_hidraulica_ne}}
  {\includesvg[scale=.9]{figuras/oneshot/os_Hidráulica_NE_D512-96}}
  {\fonte{o autor.}}
\end{figure}
\begin{figure}[H]
  \IBGEtab{\caption*{TTM para fonte hidráulica -- subsistema Norte}
       \label{fig:ttm_hidraulica_n}}
  {\includesvg[scale=.9]{figuras/oneshot/os_Hidráulica_N_D512-96}}
  {\fonte{o autor.}}
\end{figure}
\begin{figure}[H]
  \IBGEtab{\caption*{TTM para fonte térmica -- subsistema Sul}
       \label{fig:ttm_termica_s}}
  {\includesvg[scale=.9]{figuras/oneshot/os_Térmica_S_D512-96}}
  {\fonte{o autor.}}
\end{figure}
\begin{figure}[H]
  \IBGEtab{\caption*{TTM para fonte térmica -- subsistema Sudeste}
       \label{fig:ttm_termica_se}}
  {\includesvg[scale=.9]{figuras/oneshot/os_Térmica_SE_D512-96}}
  {\fonte{o autor.}}
\end{figure}
\begin{figure}[H]
  \IBGEtab{\caption*{TTM para fonte térmica -- subsistema Nordeste}
       \label{fig:ttm_termica_ne}}
  {\includesvg[scale=.9]{figuras/oneshot/os_Térmica_NE_D512-96}}
  {\fonte{o autor.}}
\end{figure}
\begin{figure}[H]
  \IBGEtab{\caption*{TTM para fonte térmica -- subsistema Norte}
       \label{fig:ttm_termica_n}}
  {\includesvg[scale=.9]{figuras/oneshot/os_Térmica_N_D512-96}}
  {\fonte{o autor.}}
\end{figure}
\begin{figure}[H]
  \IBGEtab{\caption*{TTM para fonte eólica -- subsistema Sul}
       \label{fig:ttm_eolica_s}}
  {\includesvg[scale=.9]{figuras/oneshot/os_Eólica_S_D512-96}}
  {\fonte{o autor.}}
\end{figure}
\begin{figure}[H]
  \IBGEtab{\caption*{TTM para fonte eólica -- subsistema Sudeste}
       \label{fig:ttm_eolica_se}}
  {\includesvg[scale=.9]{figuras/oneshot/os_Eólica_SE_D512-96}}
  {\fonte{o autor.}}
\end{figure}
\begin{figure}[H]   
  \IBGEtab{\caption*{TTM para fonte eólica -- subsistema Nordeste}
       \label{fig:ttm_eolica_ne}}
  {\includesvg[scale=.9]{figuras/oneshot/os_Eólica_NE_D512-96}}
  {\fonte{o autor.}}
\end{figure}
\begin{figure}[H]
  \IBGEtab{\caption*{TTM para fonte eólica -- subsistema Norte}
       \label{fig:ttm_eolica_n}}
  {\includesvg[scale=.9]{figuras/oneshot/os_Eólica_N_D512-96}}
  {\fonte{o autor.}}
\end{figure}

\section*{Modelo Neural TTM - Finetune}

\begin{figure}[H]
  \IBGEtab{\caption*{TTM para fonte hidráulica -- subsistema Sul}
       \label{fig:ttm_hidraulica_s_finetune}}
  {\includesvg[scale=.9]{figuras/finetune/fn_Hidráulica_S_D512-96}}
  {\fonte{o autor.}}
\end{figure}
\begin{figure}[H]
  \IBGEtab{\caption*{TTM para fonte hidráulica -- subsistema Sudeste}
       \label{fig:ttm_hidraulica_se_finetune}}
  {\includesvg[scale=.9]{figuras/finetune/fn_Hidráulica_SE_D512-96}}
  {\fonte{o autor.}}
\end{figure}
\begin{figure}[H]
  \IBGEtab{\caption*{TTM para fonte hidráulica -- subsistema Nordeste}
       \label{fig:ttm_hidraulica_ne_finetune}}
  {\includesvg[scale=.9]{figuras/finetune/fn_Hidráulica_NE_D512-96}}
  {\fonte{o autor.}}
\end{figure}
\begin{figure}[H]
  \IBGEtab{\caption*{TTM para fonte hidráulica -- subsistema Norte}
       \label{fig:ttm_hidraulica_n_finetune}}
  {\includesvg[scale=.9]{figuras/finetune/fn_Hidráulica_N_D512-96}}
  {\fonte{o autor.}}
\end{figure}
\begin{figure}[H]
  \IBGEtab{\caption*{TTM para fonte térmica -- subsistema Sul}
       \label{fig:ttm_termica_s_finetune}}
  {\includesvg[scale=.9]{figuras/finetune/fn_Térmica_S_D512-96}}
  {\fonte{o autor.}}
\end{figure}
\begin{figure}[H]
  \IBGEtab{\caption*{TTM para fonte térmica -- subsistema Sudeste}
       \label{fig:ttm_termica_se_finetune}}
  {\includesvg[scale=.9]{figuras/finetune/fn_Térmica_SE_D512-96}}
  {\fonte{o autor.}}
\end{figure}
\begin{figure}[H]
  \IBGEtab{\caption*{TTM para fonte térmica -- subsistema Nordeste}
       \label{fig:ttm_termica_ne_finetune}}
  {\includesvg[scale=.9]{figuras/finetune/fn_Térmica_NE_D512-96}}
  {\fonte{o autor.}}
\end{figure}
\begin{figure}[H]
  \IBGEtab{\caption*{TTM para fonte térmica -- subsistema Norte}
       \label{fig:ttm_termica_n_finetune}}
  {\includesvg[scale=.9]{figuras/finetune/fn_Térmica_N_D512-96}}
  {\fonte{o autor.}}
\end{figure}
\begin{figure}[H]
  \IBGEtab{\caption*{TTM para fonte eólica -- subsistema Sul}
       \label{fig:ttm_eolica_s_finetune}}
  {\includesvg[scale=.9]{figuras/finetune/fn_Eólica_S_D512-96}}
  {\fonte{o autor.}}
\end{figure}
\begin{figure}[H]
  \IBGEtab{\caption*{TTM para fonte eólica -- subsistema Sudeste}
       \label{fig:ttm_eolica_se_finetune}}
  {\includesvg[scale=.9]{figuras/finetune/fn_Eólica_SE_D512-96}}
  {\fonte{o autor.}}
\end{figure}
\begin{figure}[H]
  \IBGEtab{\caption*{TTM para fonte eólica -- subsistema Nordeste}
       \label{fig:ttm_eolica_ne_finetune}}
  {\includesvg[scale=.9]{figuras/finetune/fn_Eólica_NE_D512-96}}
  {\fonte{o autor.}}
\end{figure}
\begin{figure}[H]   
  \IBGEtab{\caption*{TTM para fonte eólica -- subsistema Norte}
       \label{fig:ttm_eolica_n_finetune}}
  {\includesvg[scale=.9]{figuras/finetune/fn_Eólica_N_D512-96}}
  {\fonte{o autor.}}
\end{figure}


\end{apendicesenv}

% Anexos
% \begin{anexosenv}
\chapter*{Título do Anexo A}




\end{anexosenv}

% Índice remissivo
\printindex
\end{document}